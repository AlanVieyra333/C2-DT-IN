% !TEX root = ../../../integrado.tex
\begin{UseCase}{IN-DES-CU1.1}{Registrar Oferta Educativa}{
	% V 0.1 Ok.
	La oferta educativa registrada servirá para establecer los cupos a ofrecer a los aspirantes para un ciclo escolar y modalidad seleccionados.\\
	
	Para realizar esta oferta se deben seleccionar los Programas Académicos de la Unidad Académica que se deseen ofertar para luego definir el número de Oferta Educativa propuesta por Área de conocimiento que ofrezca la Unidad Académica.
}
	\UCccitem{Versión}{0.1}
	
	\UCccsection{Datos para el control Interno}	
	\UCccitem{Elaboró}{Eduardo Espino Maldonado}
	\UCccitem{Supervisó}{Ulises Vélez Saldaña}
	\UCccitem{Operación}{Registro}
	\UCccitem{Prioridad}{Alta} 
	\UCccitem{Complejidad}{Baja}
	\UCccitem{Volatilidad}{Baja} 
	\UCccitem{Madurez}{Media} 
	\UCccitem{Estatus}{Revisado}
	\UCccitem{Dificultades}{¿Cual es el rango de oferta educativa? Falta regla y mensaje}
	\UCccitem{Fecha del último estatus}{5 de diciembre de 2017}
	
	\UCccsection{Revisión Version 0.1}
	\UCccitem{Fecha}{5 de diciembre de 2017}
	\UCccitem{Evaluador}{Ulises Vélez Saldaña}
	\UCccitem{Resultado}{aprobado}
	\UCccitem{Observaciones}{Ninguna}

	\UCsection{Atributos}
	% V 0.1 Ok.
	\UCitem{Actor}{%
	\begin{Titemize}
		\Titem \refElem{DESEncargadoDeOfertaEducativa}
	\end{Titemize}
} 
	% V 0.1 Ok.
	\UCitem{Propósito}{Saber el número estudiantes esperados por Programa Académico ofertado, Modalidad y unidad Académica en cada Ciclo Escolar para facilitar el monitoreo del proceso de Inscripciones.}
	% V 0.1 Ok.
	\UCitem{Entradas}{\begin{Titemize}
		\Titem Para cada Programa Académico se indica si se oferta o no en el Ciclo Escolar.
		\Titem Cantidad de \refElem{tOfertaEducativa} para cada Programa Académico ofertado.
	\end{Titemize}}
	% V 0.1 Ok.
	\UCitem{Origen}{\begin{Titemize}
			\Titem \ioSeleccionar
	\end{Titemize}}
	% V 0.1 Ok.
	\UCitem{Salidas}{%
		\begin{Titemize}
			\Titem Los nombres de los Niveles Educativos de la Unidad Académica (ver \refElem{NivelEducativo.nombre})
			\Titem Los nombres de las Áreas de Conocimiento de la Unidad Académica (ver \refElem{AreaDeConocimiento.nombre})
			\Titem Las \refElem{UnidadAcademica.siglas} de la Unidad Académica.
			\Titem Los nombres de los Programas Académicos (ver \refElem{ProgramaAcademico.nombre}.
			\Titem \refIdElem{MSG1}
		\end{Titemize}
	}
	% V 0.1 Ok.
	\UCitem{Destino}{Pantalla}
	% V 0.1 Ok.
	\UCitem{Precondiciones}{%
		\begin{Titemize}
			
			\Titem \textbf{Sistematizada:} Que se haya seleccionado un \textbf{Ciclo Escolar} y \textbf{Modalidad}.
			
			\Titem \textbf{Sistematizada:} Que exista al menos un \textbf{Programa Académico con Plan de Estudios} Vigente registrada en el sistema.
			
			\Titem \textbf{Sistematizada:} Que exista al menos una \textbf{Área de conocimiento} registrada en el sistema.
		
		\end{Titemize}}
	% V 0.1 Ok.
	\UCitem{Postcondiciones}{Se registrará la oferta educativa de los Programas Académicos de la Unidad Académica seleccionada.} 
	% V 0.1 Ok.
	\UCitem{Reglas de Negocio}{%
		\begin{Titemize}
			\Titem \refIdElem{BR-S002}
		\end{Titemize} 
	}
	% V 0.1 Ok.
	\UCitem{Errores}{%
		\begin{Titemize}
						
			\Titem \UCerr{Uno}{Cuando no existe al menos un \textbf{Programa Académico con Plan de Estudios} registrado,}{se muestra el mensaje \refIdElem{MSG3} y termina el caso de uso.}
			
			\Titem \UCerr{Dos}{Cuando no existe al menos una \textbf{Área de conocimiento} registrada,}{se muestra el mensaje \refIdElem{MSG3} y termina el caso de uso.}
			
			\Titem \UCerr{Tres}{Cuando los campos ingresados no cumplen con el tipo de dato solicitado,}{se muestra el mensaje \refIdElem{MSG7} y continua en el paso \ref{IN-DES-CU1.1:Registrar} de la trayectoria principal.}
						
			\Titem \UCerr{Cuatro}{Cuando la operación no se pudo llevar a cabo,}{se muestra el mensaje \refIdElem{MSG2} y continua en el paso \ref{IN-DES-CU1.1:Registrar} de la trayectoria principal.}
			
		\end{Titemize} 
	}
	% V 0.1 Ok.
	\UCitem{Viene de}{\refIdElem{IN-DES-CU1}}
	% V 0.1 Ok.
	\UCitem{Disparador}{%
		\begin{Titemize}
			\Titem Se determino el número de lugares a ofertar de los Programas Académicos de la Unidad Académica.
		\end{Titemize} 
	}
	% V 0.1 Ok.
	\UCitem{Condiciones de Término}{Se registra la oferta educativa asignada a cada Programa Académico marcado como ofertado.}
	% V 0.1 Ok.
	\UCitem{Efectos Colaterales}{El número de oferta educativa por Programa Académico, Unidad Académico, Nivel Educativo y Modalidad serán actualizados.}
	% V 0.1 Ok.
	\UCitem{Referencia Documental}{C1-PF Proceso Fortalecido}
	% V 0.1 Ok.
	\UCitem{Auditable}{Si, se guarda el nombre del actor, la fecha y hora de registro y la información registrada.}
	% V 0.1 Ok.
	\UCitem{Datos sensibles}{Ninguno identificado.}
	
\end{UseCase}

%Trayectoria Principal : Happy Path
\begin{UCtrayectoria}
	% V 0.1 Ok.
	\UCpaso [\UCactor] Solicita registrar la oferta educativa de los Programas Académicos de una Unidad Académica presionando el botón \IUbutton{Registrar Oferta Educativa} de la pantalla \refIdElem{IN-DES-IU1}.
	% V 0.1 Ok.
	\UCpaso Obtiene las \textbf{Áreas de conocimiento} asociadas a la Unidad Académica seleccionada. \refErr{Uno}
	% V 0.1 Ok.
	\UCpaso \label{IN-DES-CU1.1:ObtieneNA} Obtiene los \textbf{Programas Académicos} asociados al \textbf{Nivel Educativo}, {\bf Modalidad} y \textbf{Unidad Académica} seleccionados que tengan un {\bf Plan de Estudios} Activo con estado ``Activo'' seleccionada. \refErr{Dos}
	% V 0.1 Ok.
	\UCpaso Filtra los Programas Académicos obtenidos en el paso \ref{IN-DES-CU1.1:ObtieneNA} por Área de conocimiento, la cual por defecto es ``Todos''.
	% V 0.1 Ok.
	\UCpaso Muestra la pantalla \refIdElem{IN-DES-IU1.1} con la información obtenida.
	% V 0.1 Ok.
	\UCpaso [\UCactor] Selecciona los Programas Académicos a ofertar en el Ciclo Escolar seleccionado en el caso de uso \refIdElem{IN-DES-CU1}.
	% V 0.1 Ok.
	\UCpaso [\UCactor] Ingresa las cantidades de Oferta Educativa de los Programas Académicos a ofertar para ambos semestres del ciclo escolar.
	% V 0.1 Ok.
	\UCpaso [\UCactor] \label{IN-DES-CU1.1:Registrar} Solicita registrar la oferta educativa presionando el botón \IUbutton{Aceptar}. \refTray{A}
	% V 0.1 Ok.
	\UCpaso Verifica que los datos ingresados cumplan con el tipo de definido en el diccionario de datos con base en la regla de negocio \refIdElem{BR-S002}. \refErr{Dos}
	% V 0.1 Ok.
	\UCpaso Registra la Oferta Educativa de los Programas Académicos a ofertar registrados. \refErr{Cuatro}.
	% V 0.1 Ok.
	\UCpaso Muestra el mensaje \refIdElem{MSG1} en la pantalla \refIdElem{IN-DES-IU1} indicando que el registro de la Oferta Educativa se realizo de manera exitosa.
\end{UCtrayectoria}

%Trayectoria Alternativas

%----------------- Trayectoria A ----------------- 
\begin{UCtrayectoriaA}[Fin de la trayectoria]{A}{Cuando el actor no desea llevar a cabo la operación.}
	% V 0.1 Ok.
	\UCpaso [\UCactor] Solicita cancelar la operación presionando el botón \IUbutton{Cancelar}.
	% V 0.1 Ok.
	\UCpaso Muestra la pantalla \refIdElem{IN-DES-IU1}.
\end{UCtrayectoriaA}
