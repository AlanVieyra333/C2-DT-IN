\begin{UseCase}{IN-DAE-CU2.3}{Eliminar Calendario Académico}{		
		
		Permite eliminar un \refElem{CalendarioAcademico} con lo cual se quitará la asociación de las unidades académicas a las fechas establecidas en dicho calendario, las Unidades Académicas que se encontraban asociadas al Calendario Académico no podrán realizar las actividades relacionadas al Calendario Académico. 
	}
	\UCccitem{Versión}{0.1}
	\UCccsection{Datos para el control Interno}	
	\UCccitem{Elaboró}{Bruno Suárez Cruz }
	\UCccitem{Supervisó}{Ulises Vélez Saldaña}
	\UCccitem{Operación}{Eliminar}
	\UCccitem{Prioridad}{Media}
	\UCccitem{Complejidad}{Baja}
	\UCccitem{Volatilidad}{Alta}
	\UCccitem{Madurez}{Baja}
	\UCccitem{Estatus}{Revisado por análisis}
	\UCccitem{Dificultades}{}
	\UCccitem{Fecha del último estatus}{19 de Diciembre  de 2017}
	%\UCccs{}
	\UCccitem{Fecha}{}
	\UCccitem{Evaluador}{}
	\UCccitem{Resultado}{}
	\UCccitem{Observaciones}{}
	%	\UCsection{Atributos}{}
	\UCitem{Actor}{%
		\begin{Titemize}
			\Titem \refElem{DAEJefeDeRegistro}
			\Titem \refElem{DAEAdministradorDeRegistro}
		\end{Titemize}
	}
	\UCitem{Propósito}{Eliminar un \textbf{Calendario Académico} con lo cual las Unidades Académicas que se encontraban asociadas al mismo, quedaran temporalmente sin calendario asignado con esto las Unidades Académicas no podan realizar sus actividades.}
	\UCitem{Entradas}{Selección de la opción para notificar al \refElem{UAJefeDeControlEscolar} de cada unidad académica asociada al Calendario Académico.
	}
	\UCitem{Origen}{%
		\begin{Titemize}
			\Titem \ioObtener
			\Titem  Se selecciona con el mouse.
		\end{Titemize}
	}	
	\UCitem{Salidas}{%
		\begin{Titemize}
			\Titem 	Acrónimos de las Unidades Académicas (ver \refElem{UnidadAcademica.acronimo}).
			\Titem  Muestra en la pantalla  el mensaje  \refIdElem{MSG136} y la redaccion \refIdElem{MSG133} cuando se elimina el Calendario Académico.
		\end{Titemize}
	}
	\UCitem{Destino}{%
		\begin{Titemize}
			\Titem Correo Electrónico
			\Titem Pantalla
		\end{Titemize}	
	}
	\UCitem{Precondición}{%
		\begin{Titemize}
			\Titem \textbf{Sistematizada:} Que se haya seleccionado previamente un Ciclo Escolar y una Modalidad en el caso de uso \refIdElem{IN-DAE-CU1}.
			\Titem \textbf{Sistematizada:} Que haya al menos un Calendario Académico registrado.
			\Titem \textbf{Sistematizada:} Que el Ciclo escolar no haya iniciado (ver \refIdElem{BR-IN-N004}).
		\end{Titemize}	
	}
	\UCitem{Postcondiciones}{%
		\begin{Titemize}
			\Titem Se eliminará la asociación de  las unidades académicas y fechas establecidas al Calendario Académico definido previamente en el caso de uso \refIdElem{IN-DAE-UI2.1}.
			\Titem Se enviarán notificaciones la los Jefes de Control escolar.
		\end{Titemize}	
	}
	\UCitem{Reglas de Negocio}{%
      Ninguna}
	\UCitem{Errores}{Ninguno}
	\UCitem{Viene de}{\refIdElem{IN-DAE-CU2}}
	\UCitem{Disparadores}{ El actor requiere eliminar un calendario creado y de esta manera le eliminará la asociación de las Unidades Académicas a las fechas establecidas.
	}
	\UCitem{Condiciones de Término}{Se quitara la asociación de las Unidades Académicas de un Calendario Académico.}
	\UCitem{Efectos colaterales}{Las Unidades Académicas no contaran con un calendario definido, por lo cual estas no podrán trabajar hasta que se les sea asignado un nuevo Calendario Académico.}
	\UCitem{Referencia Documental}{}
	\UCitem{Auditable}{Sí, se debe guardar: Fecha, usuario y actividad realizada. Indicando las escuelas a las que afecta dicho cambio.}
	\UCitem{Datos sensibles}{Ninguno}
\end{UseCase}


%Trayectoria Principal : Happy Path

\begin{UCtrayectoria}
	\UCpaso [\UCactor] Presiona el botón \IUbutton{Eliminar} de la pantalla \refIdElem{IN-DAE-UI2} del calendario operativo que se requiere eliminar.
	\UCpaso Obtiene los acrónimos de las Unidades Académicas que estén asociados al Calendario Académico. 
	\UCpaso Obtiene el Mensaje \refIdElem{MSG136} y la redaccion del correo \refIdElem{MSG133} para la eliminación de un Calendario Académico.
	\UCpaso \label{IN-DAE-CU2.3:noti} Muestra la pantalla \refIdElem{IN-DAE-UI2.3} con la información obtenida.\refTray{A} 
	\UCpaso [\UCactor] Selecciona la opción requerida para el envío de la notificación.
	%\UCpaso [\UCactor] Modifica la redacción del correo a enviar.
	\UCpaso [\UCactor] Presiona el botón \IUbutton{Aceptar}.\refTray{B} 
	\UCpaso Elimina las asociaciones de las Unidades Académicas asociadas al calendario.
	\UCpaso Elimina las Actividades que componen el calendario académico.
	\UCpaso Elimina los calendarios especializados que se hayan registrado de las Unidades Académicas afectadas.
   	\UCpaso Envía las notificaciones pertinentes al Jefe de control escolar, dependiendo de las modificaciones generadas en el Calendario Académico.
    \UCpaso El sistema registra en la bitácora la hora, fecha y usuario con la descripción de la acción especificando, las Unidades Académicas afectadas.
	\UCpaso Muestra la pantalla \refIdElem{IN-DAE-UI2} o \refIdElem{IN-DAE-UI2a}.
\end{UCtrayectoria}



%-------------------------- Trayectoria Alternativa A --------------------------------- 

\begin{UCtrayectoriaA}{A}{No se requiere enviar la notificación a los Jefes de Control Escolar }
	
 	
	\UCpaso [\UCactor] Quita la selección de envío de notificación.
	\UCpaso Sigue en el paso \ref{IN-DAE-CU2.3:noti} de la trayectoria principal.
 
\end{UCtrayectoriaA}


%-------------------------- Trayectoria Alternativa B --------------------------------- 

\begin{UCtrayectoriaA}[Termina caso de uso]{B}{Se requiere Cancelar la operción}
	
	\UCpaso [\UCactor] 	Presiona el botón \IUbutton{Cancelar} de la pantalla \refIdElem{IN-DAE-UI2.3}.
	
	\UCpaso Muestra la pantalla \refIdElem{IN-DAE-UI2} o \refIdElem{IN-DAE-UI2a} 
	
\end{UCtrayectoriaA}

