% !TEX root = ../../../integrado.tex

\begin{UseCase}{IN-DAE-CU4.3}{Eliminar evaluación de E.T.S.}{
		Permite eliminar el Periodo de E.T.S. asignado a una unidad académica durante un ciclo escolar y modalidad determinada.
	}
	\UCccitem{Versión}{0.1}
	\UCccsection{Datos para el control Interno}	
	\UCccitem{Elaboró}{Alan Fernando Rincón Vieyra}
	\UCccitem{Supervisó}{Eduardo Espino Maldonado}
	\UCccitem{Operación}{Eliminar}
	\UCccitem{Prioridad}{Alta}
	\UCccitem{Complejidad}{Media}
	\UCccitem{Volatilidad}{Media}
	\UCccitem{Madurez}{Baja}
	\UCccitem{Estatus}{Por revisar}
	\UCccitem{Dificultades}{
		\begin{Titemize}
			\Titem Regla de negocio de periodo válido.
			\Titem Modelo de información de los periodos de E.T.S.
			\Titem Especificar cómo se usará el tipo de E.T.S.
			\Titem ¿Quienes son los actores?
		\end{Titemize}
	}
	\UCccitem{Fecha del último estatus}{12 de Enero del 2018}
	\UCccsection{Revisión Versión 0.1}
	\UCccitem{Fecha}{}
	\UCccitem{Evaluador}{}
	\UCccitem{Resultado}{}
	\UCccitem{Observaciones}{}
	\UCsection{Atributos}
	\UCitem{Actores}{
		\begin{Titemize}
			\Titem \refElem{Actor}
		\end{Titemize}
	}
	\UCitem{Propósito}{Eliminar las fechas de los periodos de registro, aplicación y evaluación de E.T.S. que cada \refElem{tUnidadAcademica} realiza durante un \refElem{tCicloEscolar} y \refElem{tModalidad} determinada.}
	
	\UCitem{Entradas}{
		\begin{Titemize}
			\Titem \refElem{ActividadCalendario.fechaInicio} y \refElem{ActividadCalendario.fechaFin} del periodo de E.T.S.
			\Titem \refElem{ActividadCalendario.fechaInicio} y \refElem{ActividadCalendario.fechaFin} del \refElem{Actividad.registroDeETS}.
			\Titem \refElem{ActividadCalendario.fechaInicio} y \refElem{ActividadCalendario.fechaFin} de la \refElem{Actividad.aplicacionDeETS}.
			\Titem \refElem{ActividadCalendario.fechaInicio} y \refElem{ActividadCalendario.fechaFin} del \refElem{Actividad.registroDeEvaluacion}.
		\end{Titemize}
	}
	\UCitem{Origen}{
		\begin{Titemize}
			\Titem Se selecciona con el mouse.
		\end{Titemize}
	}
	\UCitem{Salidas}{
		\begin{Titemize}
			\Titem Periodo de E.T.S. eliminado.
		\end{Titemize}
	}
	\UCitem{Destino}{Pantalla}
	\UCitem{Precondiciones}{
		\begin{Titemize}
			\Titem \textbf{Sistematizada:} Que se haya seleccionado previamente un \refElem{CicloEscolar}, una \refElem{tModalidad} y una \refElem{tUnidadAcademica}.
			\Titem \textbf{Sistematizada:} Que exista al menos un Periodo de E.T.S. registrado en el sistema.
		\end{Titemize}
	}
	\UCitem{Postcondiciones}{
		\begin{Titemize}
			\Titem Se actualiza en el sistema la eliminación de un Periodo de E.T.S., impidiendo a los alumnos inscribirse a los mismos.
		\end{Titemize}
	}
	\UCitem{Reglas de Negocio}{
		\begin{Titemize}
			\Titem \refIdElem{BR-IN-S015}
			\Titem \refIdElem{BR-IN-S016}
			\Titem \refIdElem{BR-IN-S017}
		\end{Titemize}
	}
	\UCitem{Errores}{
		\begin{Titemize}
			\Titem \UCerr{Uno}{Cuando no hay conexión con el sistema}{se muestra el mensaje \refIdElem{MSG2} y regresa al paso \ref{IN-DAE-CU4.3:solicitarEliminar} de la trayectoria principal.}
		\end{Titemize}
	}
	
	\UCitem{Viene de}{\refIdElem{IN-DAE-CU4}}
	\UCitem{Disparador}{
		\begin{Titemize}
			\Titem El actor requiere Eliminar un periodo de E.T.S.
		\end{Titemize}
	}
	
	\UCitem{Condiciones de Término}{
		\begin{Titemize}
			\Titem Se muestra la información actualizada al ya no mostrar el Periodo de E.T.S. seleccionado.
		\end{Titemize}
	}
	\UCitem{Efectos Colaterales}{Ninguno}
	\UCitem{Referencia Documental}{C1-PF Proceso Fortalecido}
	\UCitem{Auditable}{Si, se registra el usuario, la fecha y la acción que realizó.}
	\UCitem{Datos sensibles}{Ninguno}
\end{UseCase}


%Trayectoria Principal : Happy Path


\begin{UCtrayectoria}
	\UCpaso[\UCactor]  \ref{IN-DAE-CU4.3:solicitarEliminar}Solicita eliminar un Periodo de E.T.S. dando clic en el ícono {\IUMenos} en la pantalla \refIdElem{IN-DAE-IU4}. \refTray{A}
	\UCpaso Solicita la confirmación de la eliminación mostrando el mensaje \refIdElem{MSG25}.
	\UCpaso[\UCactor] Presiona el botón \IUbutton{Si}. \refTray{B}
	\UCpaso Elimina la información del Periodo de E.T.S. seleccionado. \refErr{Uno}
	\UCpaso Muestra el mensaje \refIdElem{MSG1} indicando que la eliminación del Periodo de E.T.S. se realizo de manera exitosa.
\end{UCtrayectoria}

%Trayectorias Alternativas

\begin{UCtrayectoriaA}[Fin del caso de uso.]{A}{El actor desea cancelar la operación.}
	\UCpaso[\UCactor] Solicita cancelar la operación presionando el botón \IUbutton{Regresar}.
	\UCpaso Muestra la pantalla \refIdElem{IN-DAE-UI2.5}.
\end{UCtrayectoriaA}

\begin{UCtrayectoriaA}[Fin del caso de uso.]{B}{El actor desea cancelar la operación.}
	\UCpaso[\UCactor] Solicita cancelar la operación presionando el botón \IUbutton{No}.
	\UCpaso Muestra la pantalla \refIdElem{IN-DAE-IU4}.
\end{UCtrayectoriaA}