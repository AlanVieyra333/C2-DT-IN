% !TEX root = ../../../integrado.tex
\begin{UseCase}{IN-DAE-CU3.2}{Modificar configuración de proceso de inscripción}{

	Permite al \refElem{DAEJefeDeRegistro} modificar una configuración de procesos que ofrece el módulo de inscripciones. \\

	La ejecución de los procesos es necesaria para realizar las validaciones que se requieren sobre la información de los aspirantes y de esta manera evitar posibles complicaciones en el traslado de información ofrecida por el proceso.
	}
	\UCccitem{Versión}{0.2}
	\UCccsection{Datos para el control Interno}	
	\UCccitem{Elaboró}{Eduardo Espino Maldonado}
	\UCccitem{Supervisó}{Eduardo Espino Maldonado / Ulises Vélez Saldaña}
	\UCccitem{Operación}{Modificar}
	\UCccitem{Prioridad}{Alta}
	\UCccitem{Complejidad}{Media}
	\UCccitem{Volatilidad}{Media}
	\UCccitem{Madurez}{Baja}
	\UCccitem{Estatus}{Por corregir}
	\UCccitem{Dificultades}{}
	\UCccitem{Fecha del último estatus}{14 de Diciembre de 2017}
	
	\UCccsection{Revisión Versión 0.1}
	\UCccitem{Fecha}{14 de Diciembre del 2017}
	\UCccitem{Evaluador}{Ulises Vélez Saldaña}
	\UCccitem{Resultado}{Aplicar correcciones}
	\UCccitem{Observaciones}{}
	
	\UCsection{Atributos}

	\UCitem{Actor}{\begin{Titemize}
		\Titem \refElem{DAEJefeDeRegistro} 
		\Titem \refElem{DAEAdministradorDeRegistro}
\end{Titemize}}

	\UCitem{Propósito}{Actualizar de manera eficiente la información de los estudiantes de nuevo ingreso en el Calmécac.}

	\UCitem{Entradas}{%
		\begin{Titemize}
			\Titem El \refElem{Programa.nombre} del proceso de inscripción 
			\Titem La \refElem{Programa.frecuenciaDeEjecucion} del proceso de inscripción
			\Titem El tipo de ejecución que tendrá el proceso \footnote{Ver \refElem{ParteFecha.nombre}}
			\Titem La \refElem{Programa.fechaInicio} del proceso
			\Titem La \refElem{Programa.fechaTermino} del proceso
		\end{Titemize}
	}

	\UCitem{Origen}{%
		\begin{Titemize}
			\Titem \ioSeleccionar
			\Titem \ioEscribir
		\end{Titemize}
	}

	\UCitem{Salidas}{%
		\begin{Titemize}
			\Titem El \refElem{Programa.nombre} del proceso de inscripción 
			\Titem La \refElem{Programa.frecuenciaDeEjecucion} del proceso de inscripción
			\Titem El tipo de ejecución que tendrá el proceso \footnote{Ver \refElem{ParteFecha.nombre}}
			\Titem La \refElem{Programa.fechaInicio} del proceso
			\Titem La \refElem{Programa.fechaTermino} del proceso
			\Titem \refIdElem{MSG1}.
		\end{Titemize}
	}

	\UCitem{Destino}{Pantalla}

	\UCitem{Precondiciones}{\textbf{Manual:} Se hayan establecido los periodos de ejecución de importación de información de los aspirantes.}

	\UCitem{Postcondiciones}{Se creará el proceso registrado.}

	\UCitem{Reglas de Negocio}{%
		\begin{Titemize}
			\Titem \refIdElem{BR-S001}
			\Titem \refIdElem{BR-S002}
			\Titem \refIdElem{BR-IN-N020}
		\end{Titemize}
	}

	\UCitem{Errores}{%
		\begin{Titemize}
			
			\Titem \UCerr{Uno}{Cuando los campos marcados como obligatorios no fueron ingresados,}{se muestra el mensaje \refIdElem{MSG6} y regresa al paso \ref{IN-DAE-CU3.2:Ingresa} de la trayectoria principal.}
			
			\Titem \UCerr{Dos}{Cuando los campos ingresados no cumplen con el tipo de dato solicitado,}{se muestra el mensaje \refIdElem{MSG7} y regresa al paso \ref{IN-DAE-CU3.2:Ingresa} de la trayectoria principal.}
			
			\Titem \UCerr{Tres}{Cuando la fecha de inicio de ejecución es menor al día corriente,}{se muestra el mensaje \refIdElem{MSG18} y regresa al paso \ref{IN-DAE-CU3.2:Ingresa} de la trayectoria principal.}	
			
			\Titem \UCerr{Cuatro}{Cuando la fecha inicial es posterior a la fecha de término,}{se muestra el mensaje \refIdElem{MSG18} y regresa al paso \ref{IN-DAE-CU3.2:Ingresa} de la trayectoria principal.}
			
			\Titem \UCerr{Cinco}{Cuando la operación no se pudo llevar a cabo,}{se muestra el mensaje \refIdElem{MSG2} y regresa al paso \ref{IN-DAE-CU3.2:Aceptar} de la trayectoria principal.}
			
		\end{Titemize}
	}

	\UCitem{Viene de}{\refIdElem{IN-DAE-CU3}}

	\UCitem{Disparadores}{\begin{Titemize}
	
	\Titem Hay aspirantes aceptados
	\Titem Ha concluido algún proceso de ingreso
	\Titem Se han recibido alumnos de movilidad
	\Titem Se ha actualizado información por validación de documentos
	\Titem Hay alumnos a los que se les ha aceptado su cambio de carrera

	\end{Titemize}}

	\UCitem{Condiciones de Término}{Se despliega una tabla donde se visualiza el nombre de la nueva configuración, así como su frecuencia y periodicidad de ejecución.}

	\UCitem{Efectos Colaterales}{El proceso será ejecutado en el periodo y frecuencia registrada.}

	\UCitem{Referencia Documental}{C1-PF Proceso Fortalecido}

	\UCitem{Auditable}{Si, se debe registrar en la bitácora del sistema: el usuario y hora que realizó la operación, el caso de uso, y los valores modificados.}
	% V 0.1 Ok.
	\UCitem{Datos sensibles}{No se identifico}
\end{UseCase}

%Trayectoria Principal : Happy Path

\begin{UCtrayectoria}

	\UCpaso [\UCactor] Solicita modificar un proceso dando clic en el icono \IUEditar en la pantalla \refIdElem{IN-DAE-IU3}.
	
	\UCpaso Obtiene el nombre, frecuencia de ejecución, tipo de frecuencia de ejecución, fecha de inicio y fecha de término del proceso de inscripción seleccionado.

	\UCpaso Muestra la pantalla \refIdElem{IN-DAE-IU3.2} con la información obtenida.

	\UCpaso [\UCactor] \label{IN-DAE-CU3.2:Ingresa} Ingresa los datos solicitados.

	\UCpaso  [\UCactor] \label{IN-DAE-CU3.2:Aceptar} Solicita registrar la nueva configuración presionando el botón \IUbutton{\IUEditar Registrar}. \refTray{A} 

	\UCpaso Verifica que los datos marcados como obligatorios hayan sido ingresados con base en la regla de negocio \refIdElem{BR-S001}. \refErr{Uno}

	\UCpaso Verifica que los datos ingresados cumplan con el tipo de dato definido en el diccionario de datos con base en la regla de negocio \refIdElem{BR-S002}. \refErr{Dos} 

	\UCpaso Verifica que la fecha ingresada sea posterior al día corriente con base en la regla de negocio \refIdElem{BR-IN-N020}. \refErr{Tres} 

	\UCpaso Verifica que la fecha inicial registrada sea menor que la fecha final registrada con base en la regla de negocio \refIdElem{BR-IN-N020}. \refErr{Cuatro}

	\UCpaso Persiste la nueva configuración de proceso para las Unidades Académicas seleccionadas. \refErr{Cinco}

	\UCpaso Muestra el mensaje \refIdElem{MSG1} en la pantalla \refIdElem{IN-DAE-IU3} indicando que la nueva configuración de proceso fue registrada exitosamente.
\end{UCtrayectoria}

%----------------- Trayectoria A ----------------- 
\begin{UCtrayectoriaA}[Fin del caso de uso]{A}{El actor desea cancelar la operación.}

	\UCpaso [\UCactor] Solicita cancelar la operación presionando el botón \IUbutton{Cancelar}.

	\UCpaso Muestra la pantalla \refIdElem{IN-DAE-IU3}.

\end{UCtrayectoriaA}




