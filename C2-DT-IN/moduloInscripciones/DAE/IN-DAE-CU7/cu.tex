\begin{UseCase}{IN-DAE-CU7}{Consultar datos de estudiantes}{
	% V 0.1 Ok.	
	Permite al actor consultar el estado en el que se encuentra el estudiante como datos personales, asignación en la unidad de aprendizaje, estado en que se encuentra su inscripción. 	
}
    \UCccitem{Versión}{0.1}
    \UCccsection{Datos para el control Interno}	
    \UCccitem{Elaboró}{Bruno Suárez Cruz }
    \UCccitem{Supervisó}{Ulises Vélez Saldaña}
    \UCccitem{Operación}{Consultar}
    \UCccitem{Prioridad}{Media}
    \UCccitem{Complejidad}{Baja}
    \UCccitem{Volatilidad}{Alta}
    \UCccitem{Madurez}{Baja}
    \UCccitem{Estatus}{Revisado, por corregir.}
    \UCccitem{Dificultades}{}
    \UCccitem{Fecha del último estatus}{5 de Enero del 2018}
    \UCccsection{Revisión versión 0.1}
    \UCccitem{Fecha}{8 de Enero del 2018}
    \UCccitem{Evaluador}{Ulises Vélez Saldaña}
    \UCccitem{Resultado}{Aplicar correcciones marcadas en los TODO's.}
    \UCccitem{Observaciones}{}
    \UCsection{Atributos}
    % V 0.1 Ok.	
    \UCitem{Actor}{%
    	\begin{Titemize}
    		\Titem \refElem{DAEJefeDeRegistro}
    		\Titem \refElem{DAEAdministradorDeRegistro}
 	   \end{Titemize}
	}
	% V 0.1 Ok.	
    \UCitem{Propósito}{Visualizar los datos académicos y personales de un aspirante en especifico}
    % V 0.1 Ok.	
    \UCitem{Entradas}{Ninguna}
    % V 0.1 Ok.	
    \UCitem{Origen}{%
    %	\begin{Titemize}
    		%\Titem 
    		\ioObtener
    		%\Titem  Se selecciona con el mouse.
    	%\end{Titemize}
    }
	% V 0.1 DONE: Referenicar al modelo de información. Coordinate con Francisco. Basate en el primer ejemplo.
    \UCitem{Salidas}{%
    	\begin{Titemize}		
    		\Titem Se muestra en la pantalla los datos personales del estudiante: 			
                \begin{Citemize}
					 \item \refElem{Alumno.nombre}, \refElem{Alumno.primerAp} y \refElem{Alumno.segundoAp}.
					 \item \refElem{Alumno.CURP}
					 \item \refElem{tdSexo}
					 \item Edad
					 % Fecha de Nacimiento
					 \item \refElem{Alumno.fechaDeNacimiento} 
					 \item Lugar de Nacimiento
					 %Entidad Federativa
					 \item \refElem{tdEntidad.nombre}
					 %Calle
					 \item \refElem{DireccionDeAlumno.calle} 
					 %Número
					 \item \refElem{DireccionDeAlumno.numero} 
					 %Colonia
					 \item \refElem{DireccionDeAlumno.colonia} 
					 %Municipio
					 \item \refElem{DireccionDeAlumno.municipio} 
					 %CP
					 \item \refElem{DireccionDeAlumno.cp} 
                \end{Citemize}
    	 	\Titem Se muestra en la pantalla la escuela de procedencia.
    		\Titem Se muestra en la pantalla la asignación del estudiante (Unidad Académica, Programa Académico, Nivel, Plan de estudios, Modalidad, Especialidad ).
    		\Titem Se muestra en la pantalla Proceso, Folio, Preboleta, Boleta, Tipo.
    		\Titem Se muestra en la pantalla datos de la carga (Tipo de Carga, Estatus de carga).
    		\Titem Fotografía del estudiante.
    	\end{Titemize}			
    }
	% V 0.1 Ok.
    \UCitem{Destino}{Pantalla}
    % V 0.1 Ok.
    \UCitem{Precondición}{
    		\begin{Titemize}
    		\Titem \textbf{Sistematizada:} Que se haya seleccionado previamente un Ciclo Escolar y una Modalidad.
    		\Titem \textbf{Sistematizada:} Que haya al menos una unidad académica registrada.
    		\Titem \textbf{Sistematizada:} Que se encuentre un estudiante registrado.
    	\end{Titemize}	
    }
    % V 0.1 Ok.
    \UCitem{Postcondiciones}{ Ninguna }
    % V 0.1 TODO: Terminar
    \UCitem{Reglas de Negocio}{Ninguna
    %Falta escribir bien las reglas de negocio que determinan la forma en que se obtine la información de 
%    	\begin{Titemize}
%    		\Titem \refElem{BR-X} 
%		    \Titem  \refElem{BR-X}
%		 \end{Titemize} 
	}
	% V 0.1 Ok.
    \UCitem{Errores}{Ninguno
	   % \Titem \UCerr{Uno}{Cuando se ha establecido un periodo demasiado corto muestra el mensaje \refIdElem{MSG125} en la pantalla \refIdElem{IN-DAE-UI2.2} y continuamos en el paso \ref{IN-DAE-CU2.6:sel} de la trayectoria principal.}  
	}
	% V 0.1 DONE: Referenciar el Caso de uso 6.2.1
    \UCitem{Viene de}{\refIdElem{IN-DAE-CU6.2.1}}
    % V 0.1 Ok.
    \UCitem{Disparadores}{
    	El actor requiere visualizar los datos de un estudiante.
    }
    % V 0.1 Ok.
    \UCitem{Condiciones de Término}{
	    Se muestran todas la información de un estudiante. 
    }
    % V 0.1 Ok.
    \UCitem{Referencia Documental}{}
	% V 0.1 DONE: Si, se debe guardar en la bitácora quien tiene aceso a los datos personales, fecha y hora.
    \UCitem{Auditable}{Si, se debe guardar en la bitácora quien tiene aceso a los datos personales, fecha y hora.}
    % V 0.1 DONE: Si, listar todos los datos personales del alumno.
    \UCitem{Datos sensibles}{ 
    \begin{Titemize}
    \Titem Si,datos personales del alumno
	\Titem Nombre.
	\Titem CURP.
	\Titem Genero.
	\Titem Edad.
	\Titem Fecha de Nacimiento.
	\Titem Lugar de Nacimiento.
	\Titem Entidad Federativa.
	\Titem Calle
	\Titem Número
	\Titem Colonia
	\Titem Delegación o Municipio 
	\Titem C.P
	\Titem Fotografía
	  
\end{Titemize} 
    
 }    
\end{UseCase}

%Trayectoria Principal : Happy Path
\begin{UCtrayectoria}
	% V 0.1 Ok.
	\UCpaso [\UCactor] Da clic en el icono \IUVer de la pantalla \refIdElem{IN-DAE-IU6.2.1} o \refIdElem{IN-DAE-IU8} del estudiante que se requiere ver su información.
	% V 0.1 Ok.
	\UCpaso Obtiene los datos personales (Nombre, CURP, Genero, Edad, Fecha de Nacimiento, Lugar de Nacimiento, Entidad Federativa, Calle, Número, Colonia, Delegación o Municipio, C.P y fotografiá) asignados al estudiante del que se requiere visualizar su información seleccionado previamente en el caso de uso \refIdElem{IN-DAE-CU6.2.1}.
	% V 0.1 Ok.
	\UCpaso Obtiene la escuela de procedencia del aspirante asociada.
	% V 0.1 Ok.
	\UCpaso Obtiene la asignación del estudiante en la unidad académica (Unidad Académica, programa académico, nivel, plan de estudios, modalidad, especialidad). 
	
	% V 0.1 TODO: Proponer la regla.
	\UCpaso Obtiene el Proceso, Folio, Pre-Boleta, Boleta, Tipo y el Estatus que se encuentre asociado al aspirante dentro de la unidad académica a la que pertenece dicho estudiante.  
	% V 0.1 DONE: Quitar tipo de carga, manejar solo, fecha y hora de untima actualización de datos.
	\UCpaso Obtiene lo datos de la carga (Hora y Fecha) del momento en que la información del aspirantes de la ultima actualización. 
	% V 0.1 Ok.
	\UCpaso \label{IN-DAE-CU7:recarga} Muestra la pantalla \refIdElem{IN-DAE-IU7} con la información obtenida.
	% V 0.1 Ok.
	\UCpaso [\UCactor] El actor ya no requiere visualizar la información del estudiante y presiona el botón \IUbutton{Regresar}.
\end{UCtrayectoria}

\subsection{Puntos de extensión}

\UCExtensionPoint{Se requiere refrescar la información}{El actor requiere refrescar la información}{ Paso \ref{IN-DAE-CU7:recarga} de la Trayectoria Principal}{\refElem{IN-DAE-PR-01}}

