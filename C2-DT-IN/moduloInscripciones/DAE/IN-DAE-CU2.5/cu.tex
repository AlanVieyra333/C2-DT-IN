\begin{UseCase}{IN-DAE-CU2.5}{Visualizar calendario escolar personalizado}{
	Permite visualizar de la Unidades Académica seleccionada, las fechas establecidas para las actividades por omisión, días inhábiles, periodos vacacionales y el calendario donde se muestran los días marcados definidos con anterioridad en el \refElem{tCalendarioEscolar} mediante el cual las llevará acabo sus funciones en un ciclo escolar.
	}
    \UCccitem{Versión}{0.1}
    \UCccsection{Datos para el control Interno}	
    \UCccitem{Elaboró}{Bruno Suárez Cruz }
    \UCccitem{Supervisó}{Ulises Vélez Saldaña}
    \UCccitem{Operación}{Visualizar}
    \UCccitem{Prioridad}{Media}
    \UCccitem{Complejidad}{Baja}
    \UCccitem{Volatilidad}{Alta}
    \UCccitem{Madurez}{Baja}
    \UCccitem{Estatus}{Aprobado por Análisis.}
    \UCccitem{Dificultades}{}
    \UCccitem{Fecha del último estatus}{19 de Diciembre de 2017}
    %\UCccs{}
    \UCccitem{Fecha}{}
    \UCccitem{Evaluador}{}
    \UCccitem{Resultado}{}
    \UCccitem{Observaciones}{}
    %	\UCsection{Atributos}{}
    \UCitem{Actor}{%
	    \begin{Titemize}
	    	\Titem \refElem{DAEJefeDeRegistro}
    		\Titem \refElem{DAEAdministradorDeRegistro}
    	\end{Titemize}
	}
    \UCitem{Propósito}{Visualizar el \textbf{Calendario Escolar} con las Unidades Académicas y fechas previamente asociadas en el caso de uso \refIdElem{IN-DAE-CU2.1}.}
    \UCitem{Entradas}{Ninguna}
    \UCitem{Origen}{%
	    \begin{Titemize}
    		\Titem \ioObtener
    		\Titem  Se selecciona con el mouse.
	    \end{Titemize}
    }	
    \UCitem{Salidas}{%
    	\begin{Titemize}
    	    \Titem \refElem{UnidadAcademica.nombre}. 
        	\Titem Actividades definidas por omisión para cada Calendario Académico.
    	    \Titem Fecha de inicio y fin de cada actividad a definir del calendario (ver \refElem{ActividadCalendario.fechaInicio} y \refElem{ActividadCalendario.fechaFin}).
	    \end{Titemize}			
    }
    \UCitem{Destino}{Pantalla}
    \UCitem{Precondición}{%
        \begin{Titemize}
        	\Titem \textbf{Sistematizada:} Que se haya seleccionado previamente un Ciclo Escolar y una Modalidad.
        	\Titem \textbf{Sistematizada:} Que haya al menos un Calendario Escolar registrado.
	    \end{Titemize}	
    }
    \UCitem{Postcondiciones}{Ninguna}
    \UCitem{Reglas de Negocio}{Ninguna}
    \UCitem{Errores}{Ninguno}
    \UCitem{Viene de}{\refIdElem{IN-DAE-CU2}}
    \UCitem{Disparadores}{%
 	   El actor requiere visualizar el \textbf{Calendario Escolar} de una Unidad Académica, como consulta de las fechas establecidas para las actividades.
    }
    \UCitem{Condiciones de Término}{Ninguna}
    \UCitem{Referencia Documental}{}
    \UCitem{Auditable}{No}
    \UCitem{Datos sensibles}{Ninguno}    
\end{UseCase}


%Trayectoria Principal : Happy Path

\begin{UCtrayectoria}
	\UCpaso [\UCactor] Presiona el link \underline{Acrónimo de unidad académica} de la pantalla \refIdElem{IN-DAE-UI2} cuando se encuentre el icono \IUVer o \IUDone previo al acrónimo de la unidad académica.
	\UCpaso Obtiene el nombre y acrónimo de la Unidad Académica para la modalidad y ciclo escolar seleccionados en el caso de uso \refIdElem{IN-DAE-CU1}.
	\UCpaso \label{IN-DAE-CU2.5:mod} Obtiene las actividades definidas por omisión y por el actor así como las fechas correspondientes para el \textbf{Calendario Escolar} previamente definido en el caso de uso \refIdElem{IN-DAE-CU2.1} o \refIdElem{IN-DAE-CU2.6}, las cuales serán separadas por periodo.
	\UCpaso \label{IN-DAE-CU2.5:ventana} Muestra la pantalla \refIdElem{IN-DAE-UI2.5} con la información obtenida. \refTray{A} 
	\UCpaso [\UCactor] Presiona el botón \IUbutton{Regresar}.
	\UCpaso Muestra la pantalla \refIdElem{IN-DAE-UI2} 
\end{UCtrayectoria}


%-------------------------- Trayectoria Alternativa A --------------------------------- 

\begin{UCtrayectoriaA}{A}{Se requiere visualizar la justificación de la Unidad Académica}
	
	
	\UCpaso [\UCactor] Da clic en el icono \IUUTematicas.
	
	\UCpaso Obtiene la fechas establecidas para la actividad en el calendario padre.
	
	\UCpaso Obtiene la fechas establecidas en el calendario hijo por parte de la Unidad Académica.
	
	\UCpaso Obtiene la justificación de la Unidad Académica. 
	
	\UCpaso Muestra la pantalla  \refIdElem{IN-DAE-UI2.5a} con los datos obtenidos.
	
	\UCpaso [\UCactor] Cierra la venta emergente.
	
	\UCpaso Sigue en el paso \ref{IN-DAE-CU2.5:ventana} de la trayectoria principal.
	
\end{UCtrayectoriaA}



\subsection{Puntos de extensión}

\UCExtensionPoint{Modificar calendario escolar personalizado}{El actor requiere Editar el Calendario Escolar de una Unidad Académica}{ Paso \ref{IN-DAE-CU2.5:mod} de la Trayectoria Principal}{\refIdElem{IN-DAE-CU2.6}}
