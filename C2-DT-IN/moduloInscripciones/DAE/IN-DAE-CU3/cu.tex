% !TEX root = ../../../integrado.tex
\begin{UseCase}{IN-DAE-CU3}{Gestionar proceso de inscripción}{
	% DONE V 0.1
	Permite al \refElem{DAEJefeDeRegistro} ver la lista de los procesos programados para su ejecución y la configuración con la que cuenta cada uno de ellos. Cuando el \refElem{DAEJefeDeRegistro} requiera una configuración distinta a las listadas, tendrá la opción de registrar una nueva, así como las opciones de editar o eliminar una configuración existente.		
	} 
	\UCccitem{Versión}{0.2}
	\UCccsection{Datos para el control Interno}	
	\UCccitem{Elaboró}{Eduardo Espino Maldonado}
	\UCccitem{Supervisó}{Ulises Vélez Saldaña}
	\UCccitem{Operación}{Gestionar}
	\UCccitem{Prioridad}{Medio}
	\UCccitem{Complejidad}{Baja}
	\UCccitem{Volatilidad}{Baja}
	\UCccitem{Madurez}{Baja}
	\UCccitem{Estatus}{Por revisar con desarrollo}
	\UCccitem{Dificultades}{}
	\UCccitem{Fecha del último estatus}{01 de diciembre del 2017}
	
	\UCccsection{Revisión Versión 0.1}
	\UCccitem{Fecha}{1 de Diciembre del 2017}
	\UCccitem{Evaluador}{Ulises Vélez Saldaña}
	\UCccitem{Resultado}{Corregir}
	\UCccitem{Observaciones}{
		\begin{Titemize}
			\Titem Vamos a cambiar el nombre de ``Servicios WEB'' por ``procesos''.
%			\Titem TOCHK Agregar al actor JefeDeRegistroDeLaDAE.
			\Titem Realizar los cambios marcados en los comentarios con ``TODO''.
			\Titem Terminar.
		\end{Titemize}	
	}
	\UCccsection{Revisión Versión 0.2}
	\UCccitem{Fecha}{13 de Diciembre del 2017}
	\UCccitem{Evaluador}{Ulises Vélez Saldaña}
	\UCccitem{Resultado}{Correcto, pasa a revisión con desarrollo.}
	\UCccitem{Observaciones}{Ninguna.}
	\UCsection{Atributos}
	% V 0.2 Ok.
	\UCitem{Actor}{\begin{Titemize}
			\Titem \refElem{DAEJefeDeRegistro} 
			\Titem \refElem{DAEAdministradorDeRegistro}
	\end{Titemize}}
	% V 0.1 TOCHK: Cambiar por una explicación de ``para que sirve este CU'', por ejemplo: Proveer un mecanismo para que se carguen de manera automática y eficiente la información de los aspirantes aceptados, alumnos de cambio de carrera y de movilidad al CALMECEAC para su inscripción.
	\UCitem{Propósito}{Proveer un mecanismo para programar la importación de información de los aspirantes de nuevo ingreso, de cambio de carrera o alumno de movilidad.}
	% V 0.1 DONE.
	\UCitem{Entradas}{Ninguna.}
	% V 0.1 DONE.
	\UCitem{Origen}{No aplica.}
	% V 0.1 TOCHK: Hace falta actualizar la pantalla pero los datos de salida deben ser:
	%  - Ciclo escolar.
	%  - Nombre de la configuración.
	%  - Frecuencia de ejecución.
	%  - Periodo de ejecución.
	% V 0.2 TOCHK: Agregar referencias al modelo de información. puede ser: Configuración.nombre, configuracion.frecuencia, etc. o si corresponde la lista con todos los atributos de la configuración poner solamente datos de la \refElem{Configuración}
	\UCitem{Salidas}{
		\begin{Titemize}
%			\Titem \refElem{CicloEscolar.clave} del Ciclo Escolar
			\Titem El \refElem{Programa.nombre} del proceso de inscripción.
			\Titem La \refElem{Programa.frecuenciaDeEjecucion} del proceso de inscripción con base en la regla de negocio \refIdElem{BR-IN-N018}.
			\Titem La \refElem{Programa.fechaInicio} y \refElem{Programa.fechaTermino} del proceso de inscripción \refIdElem{BR-IN-N019}.
		\end{Titemize}	
	}
	% V 0.1 Ok.
	\UCitem{Destino}{Pantalla}
	% V 0.1 Ok.
	\UCitem{Precondiciones}{Ninguna}
	% V 0.1 Ok.
	\UCitem{Postcondiciones}{Ninguna}
	% V 0.1 Ok.
	\UCitem{Reglas de Negocio}{\begin{Titemize}
		\Titem \refIdElem{BR-IN-N018}
		\Titem \refIdElem{BR-IN-N019}
	\end{Titemize}
	}
	% V 0.1 Ok.
	\UCitem{Errores}{Ninguno}
	% V 0.1 Ok.
	\UCitem{Viene de}{Primario}
	% V 0.1 Ok.
	\UCitem{Disparadores}{
		\begin{Titemize}
			\Titem El actor desea consultar las distintas configuraciones de los procesos.
			\Titem El actor desea editar una configuración de procesos.
			\Titem El actor desea eliminar una configuración de procesos.
			\Titem El actor desea registrar una nueva configuración de procesos.
	\end{Titemize}} 
	% V 0.1 Ok.
	\UCitem{Condiciones de Término}{Se despliega una tabla donde se visualizan las configuraciones de los procesos registrados.}
	% V 0.1 Ok.
	\UCitem{Efectos Colaterales}{Ninguno.}
	% V 0.1 Ok.
	\UCitem{Referencia Documental}{C1-PF Proceso Fortalecido}
	% V 0.1 TOCHK: Cambiar por No. (No es lo mismo ``No'' que ``No aplica'')
	% V 0.2 Ok.
	\UCitem{Auditable}{No}
	% V 0.1 TOCHK: Cambiar por Ninguno. (No es lo mismo ``Ninguno'' que ``No aplica'')
	% V 0.2 Ok.
	\UCitem{Datos sensibles}{Ninguno}
\end{UseCase}

%Trayectoria Principal : Happy Path

\begin{UCtrayectoria}
	% V 0.1 TOCHK: Cambiar ``Consejos'' por ``Configuraciones de proceos'', cambiar ``clic'' por ``click''. quitar la frase ``que viene''
	% V 0.2 Ok.
	\UCpaso [\UCactor] Solicita gestionar la configuración de procesos presionando la opción de \textbf{Procesos de Incripción} del menú \refElem{IN-DAE-MN1}.
	% V 0.1 TOCHK: Terminar la trayectoria, considerar la trayectoria cuando no hay configuraciones registradas.
	\UCpaso \label{IN-DAE-CU3:Obtiene} Busca las configuraciones de procesos registradas en el Calmécac.
	
	\UCpaso Determina las fechas de ejecución de los procesos de inscripción obtenidos en el paso \ref{IN-DAE-CU3:Obtiene} con base en la regla de negocio \refIdElem{BR-IN-N018}.
	
	\UCpaso Determina los periodos de programación de los procesos de inscripción obtenidos en el paso \ref{IN-DAE-CU3:Obtiene} con base en la regla de negocio \refIdElem{BR-IN-N019}.
	% V 0.2 Ok.
	\UCpaso Ordena por fecha de creación las configuraciones de procesos de la más reciente a la más antigua. 
%	\UCpaso Habilita las opciones correspondientes a la gestión de configuraciones.  
	% V 0.2 Ok.
	\UCpaso Muestra la pantalla \refIdElem{IN-DAE-IU3} con la información obtenida.% en una tabla y las acciones que fueron habilitadas para su gestión.  
	% V 0.2 Ok.
	\UCpaso [\UCactor] \label{IN-DAE-CU3:Gestionar} Gestiona las configuraciones mediante el botón \IUbutton{Nueva configuración} y los iconos {\IUEditar}, {\IUEliminar}.
\end{UCtrayectoria}

\subsection{Puntos de extensión}

% V 0.2 Ok.
\UCExtensionPoint{Registrar nueva configuración de procesos}
{El \refElem{DAEJefeDeRegistro} requiere registrar una nueva configuración de procesos}
{En el paso \ref{IN-DAE-CU3:Gestionar} de la trayectoria principal}
{\refIdElem{IN-DAE-CU3.1}}

% V 0.2 Ok.
\UCExtensionPoint{Editar configuración de procesos}
{El \refElem{DAEJefeDeRegistro} requiere editar una configuración de procesos}
{En el paso \ref{IN-DAE-CU3:Gestionar} de la trayectoria principal}
{\refIdElem{IN-DAE-CU3.2}}

% V 0.2 Ok.
\UCExtensionPoint{Eliminar configuración de procesos}
{El \refElem{DAEJefeDeRegistro} requiere eliminar una configuración de procesos}
{En el paso \ref{IN-DAE-CU3:Gestionar} de la trayectoria principal}
{\refIdElem{IN-DAE-CU3.3}}
