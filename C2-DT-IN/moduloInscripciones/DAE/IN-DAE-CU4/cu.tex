% !TEX root = ../../../integrado.tex

\begin{UseCase}{IN-DAE-CU4}{Gestionar E.T.S.}{
		Permite realizar las acciones necesarias para manipular los periodos de E.T.S.'s (registro, aplicación y evaluación) correspondientes a una unidad académica durante un ciclo escolar y modalidad determinada.
	}
	\UCccitem{Versión}{0.1}
	\UCccsection{Datos para el control Interno}	
	\UCccitem{Elaboró}{Alan Fernando Rincón Vieyra}
	\UCccitem{Supervisó}{Eduardo Espino Maldonado}
	\UCccitem{Operación}{Gestionar}
	\UCccitem{Prioridad}{Alta}
	\UCccitem{Complejidad}{Media}
	\UCccitem{Volatilidad}{Media}
	\UCccitem{Madurez}{Baja}
	\UCccitem{Estatus}{Por revisar}
	\UCccitem{Dificultades}{
		\begin{Titemize}
			\Titem Modelo de información de los periodos de E.T.S.
			\Titem Especificar cómo se usará el tipo de E.T.S.
			\Titem ¿Quienes son los actores?
		\end{Titemize}
	}
	\UCccitem{Fecha del último estatus}{08 de Enero del 2018}
	\UCccsection{Revisión Versión 0.1}
	\UCccitem{Fecha}{}
	\UCccitem{Evaluador}{}
	\UCccitem{Resultado}{}
	\UCccitem{Observaciones}{}
	\UCsection{Atributos}
	\UCitem{Actores}{
		\begin{Titemize}
			\Titem \refElem{DAEJefeDeRegistro}
			\Titem \refElem{DAEAdministradorDeRegistro}
		\end{Titemize}
	}
	\UCitem{Propósito}{Programar los periodos de registro, aplicación y evaluación de E.T.S.}
	\UCitem{Entradas}{Ninguna}
	\UCitem{Origen}{No aplica}
	\UCitem{Salidas}{
		\begin{Titemize}
			\Titem La \refElem{ActividadCalendario.fechaInicio} y \refElem{ActividadCalendario.fechaFin} del periodo de E.T.S.
			\Titem La \refElem{ActividadCalendario.fechaInicio} y \refElem{ActividadCalendario.fechaFin} del \refElem{Actividad.registroDeETS}.
			\Titem La \refElem{ActividadCalendario.fechaInicio} y \refElem{ActividadCalendario.fechaFin} de la \refElem{Actividad.aplicacionDeETS}.
			\Titem La \refElem{ActividadCalendario.fechaInicio} y \refElem{ActividadCalendario.fechaFin} del \refElem{Actividad.registroDeEvaluacion}.
		\end{Titemize}
	}
	\UCitem{Destino}{Pantalla}
	\UCitem{Precondiciones}{
		\begin{Titemize}
			\Titem \textbf{Sistematizada:} Que se haya seleccionado previamente un \refElem{CicloEscolar}, una \refElem{tModalidad} y una \refElem{tUnidadAcademica}.
		\end{Titemize}
	}
	\UCitem{Postcondiciones}{Ninguna}
	\UCitem{Reglas de Negocio}{Ninguna}
	\UCitem{Errores}{
		\begin{Titemize}
			\Titem \UCerr{Uno}{Cuando no hay conexión con el sistema}{se muestra el mensaje \refIdElem{MSG2} y regresa al paso \ref{IN-DAE-CU4:solicitarGestionar} de la trayectoria principal.}
		\end{Titemize}
	}
	\UCitem{Viene de}{\refIdElem{IN-DAE-CU2.5}}
	\UCitem{Disparador}{
		\begin{Titemize}
			\Titem El actor requiere Consultar un periodo de E.T.S.
			\Titem El actor requiere Registrar un periodo de E.T.S.
			\Titem El actor requiere Modificar un periodo de E.T.S.
			\Titem El actor requiere Eliminar un periodo de E.T.S.
		\end{Titemize}
	}
	\UCitem{Condiciones de Término}{Se muestra la información correspondiente a cada periodo de E.T.S. registrado en el sistema.}
	\UCitem{Efectos Colaterales}{Ninguno}
	\UCitem{Referencia Documental}{C1-PF Proceso Fortalecido}
	\UCitem{Auditable}{No}
	\UCitem{Datos sensibles}{Ninguno}
	
\end{UseCase}


%Trayectoria Principal : Happy Path


\begin{UCtrayectoria}
	\UCpaso[\UCactor]  \label{IN-DAE-CU4:solicitarGestionar}Solicita gestionar un periodo de E.T.S. presionando el botón \IUbutton{Gestionar ETS's} en la pantalla \refIdElem{IN-DAE-UI2.5}.
	
	\UCpaso Obtiene los periodos de E.T.S. \refErr{Uno}
	\UCpaso Muestra la pantalla \refIdElem{IN-DAE-IU4} con la información obtenida.
	\UCpaso[\UCactor] \label{IN-DAE-CU4:gestionar}Gestiona un periodo de E.T.S. mediante los íconos {\IUAdd}, {\IUMenos} y {\IUCalendario}. \refTray{A}
\end{UCtrayectoria}

%Trayectorias Alternativas

\begin{UCtrayectoriaA}[Fin del caso de uso.]{A}{El actor desea cancelar la operación.}
	\UCpaso[\UCactor] Solicita cancelar la operación presionando el botón \IUbutton{Cancelar}.
	\UCpaso Muestra la pantalla \refIdElem{IN-DAE-UI2.5}.
\end{UCtrayectoriaA}

\subsection{Puntos de extensión}
\UCExtensionPoint{Registrar periodo de E.T.S.}
{El actor desea registrar un nuevo periodo de E.T.S.}
{En el paso \ref{IN-DAE-CU4:gestionar} de la trayectoria principal}
{\refIdElem{IN-DAE-CU4.1}}

\UCExtensionPoint{Editar periodo de E.T.S.}
{El actor desea editar un periodo de E.T.S.}
{En el paso \ref{IN-DAE-CU4:gestionar} de la trayectoria principal}
{\refIdElem{IN-DAE-CU4.2}}

\UCExtensionPoint{Eliminar periodo de E.T.S.}
{El actor desea eliminar un periodo de E.T.S.}
{En el paso \ref{IN-DAE-CU4:gestionar} de la trayectoria principal}
{\refIdElem{IN-DAE-CU4.3}}
