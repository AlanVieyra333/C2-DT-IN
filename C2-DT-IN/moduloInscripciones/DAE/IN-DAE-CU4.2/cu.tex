% !TEX root = ../../../integrado.tex

\begin{UseCase}{IN-DAE-CU4.2}{Modificar evaluación de E.T.S.}{
		Permite modificar el periodo de E.T.S. asignado a una unidad académica durante un ciclo escolar y modalidad determinada, redefiniendo la fecha de inicio y fin de:
		\begin{itemize}
			\item Periodo de E.T.S.
			\item Registro de E.T.S.
			\item Aplicación de E.T.S.
			\item Registro de evaluación.
		\end{itemize}
	}
	\UCccitem{Versión}{0.1}
	\UCccsection{Datos para el control Interno}	
	\UCccitem{Elaboró}{Alan Fernando Rincón Vieyra}
	\UCccitem{Supervisó}{Eduardo Espino Maldonado}
	\UCccitem{Operación}{Modificar}
	\UCccitem{Prioridad}{Alta}
	\UCccitem{Complejidad}{Media}
	\UCccitem{Volatilidad}{Media}
	\UCccitem{Madurez}{Baja}
	\UCccitem{Estatus}{Por revisar}
	\UCccitem{Dificultades}{
		\begin{Titemize}
			\Titem Regla de negocio de periodo válido.
			\Titem Modelo de información de los periodos de E.T.S.
			\Titem Especificar cómo se usará el tipo de E.T.S.
			\Titem ¿Quienes son los actores?
		\end{Titemize}
	}
	\UCccitem{Fecha del último estatus}{09 de Enero del 2018}
	\UCccsection{Revisión Versión 0.1}
	\UCccitem{Fecha}{}
	\UCccitem{Evaluador}{}
	\UCccitem{Resultado}{}
	\UCccitem{Observaciones}{}
	\UCsection{Atributos}
	\UCitem{Actores}{
		\begin{Titemize}
			\Titem \refElem{Actor}
		\end{Titemize}
	}
	\UCitem{Propósito}{Mantener actualizadas las fechas de los periodos de registro, aplicación y evaluación de E.T.S. que cada \refElem{tUnidadAcademica} realiza durante un \refElem{tCicloEscolar} y \refElem{tModalidad} determinada.}
	
	\UCitem{Entradas}{
		\begin{Titemize}
			\Titem \refElem{ActividadCalendario.fechaInicio} y \refElem{ActividadCalendario.fechaFin} del periodo de E.T.S.
			\Titem \refElem{ActividadCalendario.fechaInicio} y \refElem{ActividadCalendario.fechaFin} del \refElem{Actividad.registroDeETS}.
			\Titem \refElem{ActividadCalendario.fechaInicio} y \refElem{ActividadCalendario.fechaFin} de la \refElem{Actividad.aplicacionDeETS}.
			\Titem \refElem{ActividadCalendario.fechaInicio} y \refElem{ActividadCalendario.fechaFin} del \refElem{Actividad.registroDeEvaluacion}.
		\end{Titemize}
	}
	\UCitem{Origen}{
		\begin{Titemize}
			\Titem Se selecciona con el mouse.
		\end{Titemize}
	}
	\UCitem{Salidas}{
		\begin{Titemize}
			\Titem \refElem{ActividadCalendario.fechaInicio} y \refElem{ActividadCalendario.fechaFin} actualizadas para las actividades del Periodo de E.T.S. modificado.
		\end{Titemize}
	}
	\UCitem{Destino}{Pantalla}
	\UCitem{Precondiciones}{
		\begin{Titemize}
			\Titem \textbf{Sistematizada:} Que se haya seleccionado previamente un \refElem{CicloEscolar}, una \refElem{tModalidad} y una \refElem{tUnidadAcademica}.
			\Titem \textbf{Sistematizada:} Que exista al menos un Periodo de E.T.S. registrado en el sistema.
		\end{Titemize}
	}
	\UCitem{Postcondiciones}{
		\begin{Titemize}
			\Titem Se actualiza en el sistema la información del Periodo de E.T.S., el cual permite a los alumnos inscribirse a los mismos.
		\end{Titemize}
	}
	\UCitem{Reglas de Negocio}{
		\begin{Titemize}
			\Titem \refIdElem{BR-IN-S015}
			\Titem \refIdElem{BR-IN-S016}
			\Titem \refIdElem{BR-IN-S017}
		\end{Titemize}
	}
	\UCitem{Errores}{
		\begin{Titemize}
			\Titem \UCerr{Uno}{Cuando se ha establecido un Periodo de E.T.S. inválido}{se muestra el mensaje \refIdElem{MSG18} y regresa al paso \ref{IN-DAE-CU4.2:seleccionarPeriodo} de la trayectoria principal.}
			\Titem \UCerr{Dos}{Cuando se ha establecido un periodo de Registro de E.T.S. inválido}{se muestra el mensaje \refIdElem{MSG18} y regresa al paso \ref{IN-DAE-CU4.2:seleccionarRegistro} de la trayectoria principal.}
			\Titem \UCerr{Tres}{Cuando se ha establecido un periodo de Registro de E.T.S. fuera del
				Periodo de E.T.S.}{se muestra el mensaje \refIdElem{MSG130} y regresa al paso \ref{IN-DAE-CU4.2:seleccionarRegistro} de la trayectoria principal.}
			\Titem \UCerr{Cuatro}{Cuando se ha establecido un periodo de Aplicación de E.T.S. inválido}{se muestra el mensaje \refIdElem{MSG18} y regresa al paso \ref{IN-DAE-CU4.2:seleccionarAplicacion} de la trayectoria principal.}
			\Titem \UCerr{Cinco}{Cuando se ha establecido un periodo de Aplicación de E.T.S. fuera del
				Periodo de E.T.S.}{se muestra el mensaje \refIdElem{MSG130} y regresa al paso \ref{IN-DAE-CU4.2:seleccionarAplicacion} de la trayectoria principal.}
			\Titem \UCerr{Seis}{Cuando se ha establecido un periodo de Aplicación de E.T.S. antes o durante el periodo de Registro de E.T.S.}{se muestra el mensaje \refIdElem{MSG130} y regresa al paso \ref{IN-DAE-CU4.2:seleccionarAplicacion} de la trayectoria principal.}
			\Titem \UCerr{Siete}{Cuando se ha establecido un periodo de Registro de evaluación inválido}{se muestra el mensaje \refIdElem{MSG18} y regresa al paso \ref{IN-DAE-CU4.2:seleccionarEvaluacion} de la trayectoria principal.}
			\Titem \UCerr{Ocho}{Cuando se ha establecido un periodo de Registro de evaluación fuera del
				Periodo de E.T.S.}{se muestra el mensaje \refIdElem{MSG130} y regresa al paso \ref{IN-DAE-CU4.2:seleccionarEvaluacion} de la trayectoria principal.}
			\Titem \UCerr{Nueve}{Cuando se ha establecido un periodo de Registro de evaluación antes de la fecha de inicio del periodo de Aplicación de E.T.S.}{se muestra el mensaje \refIdElem{MSG130} y regresa al paso \ref{IN-DAE-CU4.2:seleccionarEvaluacion} de la trayectoria principal.}
		\end{Titemize}
	}
	
	\UCitem{Viene de}{\refIdElem{IN-DAE-CU4}}
	\UCitem{Disparador}{
		\begin{Titemize}
			\Titem El actor requiere Modificar un periodo de E.T.S.
		\end{Titemize}
	}
	
	\UCitem{Condiciones de Término}{
		\begin{Titemize}
			\Titem Se muestra la información actualizada del Periodo de E.T.S. seleccionado.
		\end{Titemize}
	}
	\UCitem{Efectos Colaterales}{Ninguno}
	\UCitem{Referencia Documental}{C1-PF Proceso Fortalecido}
	\UCitem{Auditable}{Si, se registra el usuario, la fecha y la acción que realizó.}
	\UCitem{Datos sensibles}{Ninguno}
\end{UseCase}


%Trayectoria Principal : Happy Path


\begin{UCtrayectoria}
	\UCpaso[\UCactor]  \label{IN-DAE-CU4.2:seleccionarPeriodo} Solicita modificar un Periodo de E.T.S. dando clic en el ícono {\IUCalendario} del campo \refElem{ActividadCalendario.fechaInicio} o \refElem{ActividadCalendario.fechaFin} en la pantalla \refIdElem{IN-DAE-IU4}. \refTray{A} \refTray{B} \refTray{C} \refTray{D}
	
	\UCpaso Verifica si el Periodo de E.T.S. seleccionado es válido con base en la regla de negocio \refIdElem{BR-N0X1}. \refErr{Uno}
	
	\UCpaso Modifica el Periodo de E.T.S.
	
	\UCpaso Muestra el mensaje \refIdElem{MSG1} indicando que la modificación del Periodo de E.T.S. se realizo de manera exitosa.
\end{UCtrayectoria}

%Trayectorias Alternativas

\begin{UCtrayectoriaA}[Fin del caso de uso.]{A}{El actor desea modificar un periodo de Registro de E.T.S.}
	\UCpaso[\UCactor]  \label{IN-DAE-CU4.2:seleccionarRegistro} Solicita modificar un periodo de Registro de E.T.S. dando clic en el ícono {\IUCalendario} del campo \refElem{ActividadCalendario.fechaInicio} o \refElem{ActividadCalendario.fechaFin} en la pantalla \refIdElem{IN-DAE-IU4}.
	\UCpaso Verifica si el periodo de Registro de E.T.S. seleccionado es válido con base en la regla de negocio \refIdElem{BR-N0X1}. \refErr{Dos}
	\UCpaso Verifica si el periodo de Registro de E.T.S. seleccionado se encuentra dentro del
	Periodo de E.T.S. definido previamente en el \refIdElem{IN-DAE-CU4.1} con base en la regla de negocio \refIdElem{BR-IN-S015}. \refErr{Tres}
	\UCpaso Modifica el periodo de Registro de E.T.S.
	\UCpaso Muestra el mensaje \refIdElem{MSG1} indicando que la modificación del periodo de Registro de E.T.S. se realizo de manera exitosa.
\end{UCtrayectoriaA}

\begin{UCtrayectoriaA}[Fin del caso de uso.]{B}{El actor desea modificar un periodo de Aplicación de E.T.S.}
	\UCpaso[\UCactor]  \label{IN-DAE-CU4.2:seleccionarAplicacion} Solicita modificar un periodo de Aplicación de E.T.S. dando clic en el ícono {\IUCalendario} del campo \refElem{ActividadCalendario.fechaInicio} o \refElem{ActividadCalendario.fechaFin} en la pantalla \refIdElem{IN-DAE-IU4}.
	\UCpaso Verifica si el periodo de Aplicación de E.T.S. seleccionado es válido con base en la regla de negocio \refIdElem{BR-N0X1}. \refErr{Cuatro}
	\UCpaso Verifica si el periodo de Aplicación de E.T.S. seleccionado se encuentra dentro del
	Periodo de E.T.S. definido en el \refIdElem{IN-DAE-CU4.1} con base en la regla de negocio \refIdElem{BR-IN-S015}. \refErr{Cinco}
	\UCpaso Verifica que el periodo de Aplicación de E.T.S. seleccionado sea posterior al periodo de Registro de E.T.S. definido en el \refIdElem{IN-DAE-CU4.1} con base en la regla de negocio \refIdElem{BR-IN-S016}. \refErr{Seis}
	\UCpaso Modifica el periodo de Aplicación de E.T.S.
	\UCpaso Muestra el mensaje \refIdElem{MSG1} indicando que la modificación del periodo de Aplicación de E.T.S. se realizo de manera exitosa.
\end{UCtrayectoriaA}

\begin{UCtrayectoriaA}[Fin del caso de uso.]{C}{El actor desea modificar un periodo de Registro de evaluación.}
	\UCpaso[\UCactor]  \label{IN-DAE-CU4.2:seleccionarEvaluacion} Solicita modificar un periodo de Registro de evaluación. dando clic en el ícono {\IUCalendario} del campo \refElem{ActividadCalendario.fechaInicio} o \refElem{ActividadCalendario.fechaFin} en la pantalla \refIdElem{IN-DAE-IU4}.
	\UCpaso Verifica si el periodo de Registro de evaluación seleccionado es válido con base en la regla de negocio \refIdElem{BR-N0X1}. \refErr{Siete}
	\UCpaso Verifica si el periodo de Registro de evaluación seleccionado se encuentra dentro del
	Periodo de E.T.S. definido en el \refIdElem{IN-DAE-CU4.1} con base en la regla de negocio \refIdElem{BR-IN-S015}. \refErr{Ocho}
	\UCpaso Verifica que el periodo de Registro de evaluación seleccionado sea posterior o igual a la fecha de inicio del periodo de Aplicación de E.T.S. definido en el \refIdElem{IN-DAE-CU4.1} con base en la regla de negocio \refIdElem{BR-IN-S017}. \refErr{Nueve}
	\UCpaso Modifica el periodo de Registro de evaluación.
	\UCpaso Muestra el mensaje \refIdElem{MSG1} indicando que la modificación del periodo de Registro de evaluación se realizo de manera exitosa.
\end{UCtrayectoriaA}

\begin{UCtrayectoriaA}[Fin del caso de uso.]{D}{El actor desea cancelar la operación.}
	\UCpaso[\UCactor] Solicita cancelar la operación presionando el botón \IUbutton{Regresar}.
	\UCpaso Muestra la pantalla \refIdElem{IN-DAE-UI2.5}.
\end{UCtrayectoriaA}