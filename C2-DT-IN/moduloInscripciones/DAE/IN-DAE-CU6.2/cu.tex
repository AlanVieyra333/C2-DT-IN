% !TEX root = ../../../integrado.tex
\begin{UseCase}{IN-DAE-CU6.2}{Gestionar estudiantes cargados por programa académico}{
	% V 0.1 Ok.
	Permite visualizar el avance de los estudiantes de una Unidad Académica, así como el avance de la carga de estudiantes para una Unidad Académica.
}
	\UCccitem{Versión}{0.1}
	\UCccsection{Datos para el control Interno}	
	\UCccitem{Elaboró}{Eduardo Espino Maldonado}
	\UCccitem{Supervisó}{Ulises Vélez Saldaña}
	\UCccitem{Operación}{Consulta}
	\UCccitem{Prioridad}{Media} 
	\UCccitem{Complejidad}{Media}
	\UCccitem{Volatilidad}{Baja} 
	\UCccitem{Madurez}{Media} 
	\UCccitem{Estatus}{Edición}
	\UCccitem{Dificultades}{%
		\begin{Titemize}
			\Titem \DONE En la pantalla se señala cuantos tienen boleta y cuantos tienen inscripción, pero en el modelo de información solo se tiene un campo donde esta el estado, ¿de donde vamos a obtener esa información? ¿es necesario agregar dos campos boolean a la BD que señale si tiene boleta o inscripción? 
			\Titem ¿Como se va a documentar el tipo de carga que tuvo si es compuesto? Ej. Automática desde el 25/12/2017 hasta el 25/12/2017 con validaciones y confirmación automática. 
			\Titem Falta crear términos así que van a aparecer rotos muchas ligas.
		\end{Titemize}
	}
	\UCccitem{Fecha del último estatus}{5 de enero de 2018}
	\UCccsection{Revisión Version 0.1}
	\UCccitem{Fecha}{5 de enero de 2018}
	\UCccitem{Evaluador}{Ulises Vélez Saldaña}
	\UCccitem{Resultado}{Por corregir}
	\UCccitem{Observaciones}{Aplicar los cambios marcados con TODO en los comentarios.}
	\UCsection{Atributos}
	% V 0.1 Ok.
	\UCitem{Actor}{%
		\begin{Titemize}
			\Titem \refElem{DAEJefeDeRegistro}	
			\Titem \refElem{DAEAdministradorDeRegistro}	
		\end{Titemize}
	} 
	% V 0.1 Ok.
	\UCitem{Propósito}{Brindar una herramienta que permita visualizar el avance de los estudiantes de nuevo ingreso por cada Unidad Académica.}
	% V 0.1 TOCHK: Agregar Unidad Académica (se introduce al seleccionarla de la pantalla de origen (CU6)
	\UCitem{Entradas}{
		\begin{Titemize}
			\Titem \refElem{PeriodoEscolar.clave}
			\Titem Unidad Académica seleccionada en el caso de uso \refIdElem{IN-DAE-CU6}
		\end{Titemize}}
	% V 0.1 Ok.
	\UCitem{Origen}{\ioSeleccionar}
	% V 0.1 TODO:
	\UCitem{Salidas}{%
		\begin{Titemize}
			\Titem El número de Programas Académicos (Ver \refElem{tProgramaAcademico}) ofertados en la Unidad Académica seleccionada.
			% V 0.1 TOCHK: Cuales? es una lista separada de las especialidades de todos los programas académicos ofertados por la unidad académica o es la suma y cuando val cero se coloca ``ninguna''? 
			% Se eliminará del caso de uso cuando es Superior o Media superior
			\Titem El número de especialidades ofertadas en la Unidad Académica.(Ver \refElem{tEspecialidad}) 
			% V 0.1 TOCHK: es una lista separada de las modalidades de los programas académicos ofertados por la unidad académica en el ciclo escolar?
			% Yo opino que ésta debería ser eliminada debido a que en el caso de uso IN-DAE-CU1 ya se selecciono la modalidad de la cual será elaborada la gestión
%			\Titem Las Modalidades (Ver \refElem{tModalidad}) ofertadas en la Unidad Académica seleccionada.
			% V 0.1 TOCHK: Cambiar tipo por solo ``carga'' y especificar en ese campo como un mensaje La frecuencia, el nombre y el periodo en el que se ejecuta. 
			% Se elimino de la pantalla
%			\Titem El tipo de carga que tuvo el proceso de inscripción ejecutado.
			% V 0.1 Ok.
			\Titem La \refElem{Corrida.fechaDeEjecucion} de la ultima carga que tuvo el proceso de inscripción.	
			% V 0.1 TODO: Especificar la forma en que se calculan esos datos. Referenciar reglas de negocio correspondientes.
			% Vienen directamente de la base de datos, entonces solo el total debe ser especificado con base en una regla de negocio
			\Titem Tabla que muestra el \refElem{PlanDeEstudio}, \refElem{Modalidad}, cantidad de estudiantes \refElem{esperados}, \refElem{cargados}, \refElem{conBoleta}, \refElem{inscritos} y \refElem{rechazados} de los \refElem{ProgramaAcademico} ofertados en el ciclo escolar seleccionado, de la Unidad Académica seleccionada.
			\Titem El total de alumnos
		\end{Titemize}
	}
	% V 0.1 Ok.
	\UCitem{Destino}{Pantalla}
	% V 0.1 Ok.
	\UCitem{Precondiciones}{%
		\begin{Titemize}
			\Titem \textbf{Sistematizada:} Que se haya seleccionado un \textbf{Ciclo Escolar} y \textbf{Modalidad}.	
		\end{Titemize}
	}
	% V 0.1 Ok.
	\UCitem{Postcondiciones}{Ninguna} 
	% V 0.1 TODO: actualizar.
	\UCitem{Reglas de Negocio}{Ninguna}
	% V 0.1 Ok.
	\UCitem{Errores}{Ninguno
%	\begin{Titemize}
%	
%		\Titem \UCerr{Uno}{Cuando no hay lugares definidos para las Unidades Académicas,}{se muestra el mensaje \refIdElem{MSGX} y termina el caso de uso.}
%		% La DAE no ha definido los lugares disponibles, favor de contactarla.
%	
%	\end{Titemize}
	}
	% V 0.1 Ok.
	\UCitem{Viene de}{\refIdElem{IN-DAE-CU6}}
	% V 0.1 Ok.
	\UCitem{Disparador}{El actor requiere visualizar el avance de los {\bf estudiantes de nuevo ingreso} en una Unidad Académica.}
	% V 0.1 Ok.
	\UCitem{Condiciones de Término}{Se muestra el avance de estudiantes importados al Calmécac.}
	% V 0.1 Ok.
	\UCitem{Efectos Colaterales}{Ninguno.}
	% V 0.1 Ok.
	\UCitem{Referencia Documental}{C1-PF Proceso Fortalecido}
	% V 0.1 Ok.
	\UCitem{Auditable}{No}
	% V 0.1 Ok.
	\UCitem{Datos sensibles}{No se encontró ninguno}
\end{UseCase}

%Trayectoria Principal : Happy Path

\begin{UCtrayectoria}
	% V 0.1 Ok.
	\UCpaso [\UCactor] Solicita visualizar el avance de aspirantes cargados a una Unidad Académica dando clic en el ícono de la pantalla \refIdElem{IN-DAE-UI6}.
	% V 0.1 TODO: Agregar un paso en el que se selecciona por defecto el periodo escolar seleccionado en el CU 6.
	% V 0.1 TODO: Pasar el siguiente paso al final después de calcular todos los datos.
	\UCpaso Muestra la pantalla \refIdElem{IN-DAE-IU6.2}.
	% V 0.1 TODO: Manejar como trayectoria alternativa después de mostrar la pantalla 6.2.
	\UCpaso [\UCactor] Selecciona el periodo escolar del cual será visualizada la información de carga.
	% V 0.1 TODO: Separar por cada dato y referenciar las reglas de negocio o mensajes si es que aplican.
	\UCpaso Obtiene el número de programas académicos, las especialidades y las modalidades de los programas académicos que serán ofertados en la Unidad Académica seleccionada.
	% V 0.1 TODO: Este dato de donde se obtiene?
	\UCpaso Obtiene el tipo y fecha de la ultima carga del proceso de inscripción.
	% V 0.1 TODO: Separar por cada dato y referenciar las reglas de negocio correspondientes.
	\UCpaso \label{IN-DES-CU6:Obtiene} Obtiene el plan, modalidad, el número de estudiantes esperados, cargados, con boleta, inscritos y rechazados de los programas académicos ofertados.
	% V 0.1 TODO: Especificar que los totales son la suma de las columnas, ver CU6.
	\UCpaso Calcula el total de alumnos esperados, cargados, con boleta, inscritos y rechazados obtenidos en el paso \ref{IN-DES-CU6:Obtiene}.
	% V 0.1 TODO: Indicar este paso en la trayectoria alternativa.
	\UCpaso \label{IN-DAE-CU6:Gestiona} Actualiza la pantalla \refIdElem{IN-DAE-IU6.2} con la información obtenida.
	% V 0.1 Ok.
	\UCpaso [\UCactor] Gestiona el avance de aspirantes cargados a una Unidad Académica con los botones \IUbutton{Correr la carga en este momento para esta escuela} y \IUVer. \refTray{A}
\end{UCtrayectoria}

\begin{UCtrayectoriaA}{A}{El actor desea regresar a la gestión por Unidad Académica.}
	% V 0.1 Ok.
	\UCpaso [\UCactor] Solicita regresar a la gestión por Unidad Académica dando clic en el botón \IUbutton{Regresar}.
	% V 0.1 Ok.
	\UCpaso Muestra la pantalla \refIdElem{IN-DAE-UI6}.
\end{UCtrayectoriaA}

\subsection{Puntos de extensión}

\UCExtensionPoint{Ejecutar proceso de inscripción}
{El actor requiere actualizar la información de los estudiantes de la Unidad Académica}
{Paso \ref{IN-DAE-CU6:Gestiona} de la Trayectoria Principal}
{\refIdElem{IN-DAE-PR1}}

\UCExtensionPoint{Gestionar estudiantes cargados en programa académico}
{El actor requiere Gestionar estudiantes cargados en un programa académico}
{Paso \ref{IN-DAE-CU6:Gestiona} de la Trayectoria Principal}
{\refIdElem{IN-DAE-CU6.2.1}}