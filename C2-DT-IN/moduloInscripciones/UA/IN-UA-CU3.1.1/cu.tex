\begin{UseCase}{IN-UA-CU3.1.1}{Inscribir alumnos por grupo}{
	% V 0.1 Ok.
	Permite inscribir a uno o mas alumnos a las \refElem{tUnidadDeAprendizaje} pertenecientes a un \refElem{Grupo} determinado, durante el \refElem{PeriodoEscolar} previamente seleccionado.
}
    \UCccitem{Versión}{0.3}
    \UCccsection{Datos para el control Interno}	
    \UCccitem{Elaboró}{Alan Fernando Rincón Vieyra}
    \UCccitem{Supervisó}{Ulises Vélez Saldaña}
    \UCccitem{Operación}{Registro}
    \UCccitem{Prioridad}{Alta}
    \UCccitem{Complejidad}{Media}
    \UCccitem{Volatilidad}{Baja}
    \UCccitem{Madurez}{Baja}
    \UCccitem{Estatus}{Revisada, por corregir}
    \UCccitem{Dificultades}{}
    \UCccitem{Fecha del último estatus}{07 de Febrero de 2018}
    \UCsection{Revisión de la Versión 0.1}
    \UCccitem{Fecha}{9 de febrero de 2018}
    \UCccitem{Evaluador}{Ulises eVélez Saldaña}
    \UCccitem{Resultado}{Corregir}
    \UCccitem{Observaciones}{Aplicar los cambios marcados con TODO's}
    \UCsection{Revisión de la Versión 0.2}
    \UCccitem{Fecha}{14 de febrero de 2018}
    \UCccitem{Evaluador}{Ulises eVélez Saldaña}
    \UCccitem{Resultado}{Corregir}
    \UCccitem{Observaciones}{Aplicar los cambios marcados con TODO's}
    \UCsection{Revision de la Versión 0.3}
    \UCccitem{Fecha}{14 de febrero de 2018}
    \UCccitem{Evaluador}{Ulises eVélez Saldaña}
    \UCccitem{Resultado}{Corregir}
    \UCccitem{Observaciones}{Aplicar los cambios marcados con TODO's}
    \UCsection{Atributos}
	% V 0.1 Ok.
    \UCitem{Actor}{
    	\begin{Titemize}
    		\Titem \refElem{UAJefeDeGestionEscolar}
    	\end{Titemize}
    }
	% V 0.1 Ok.
    \UCitem{Propósito}{
    	\begin{Titemize}
    		\Titem Inscribir varios alumnos a las Unidades de Aprendizaje de un grupo.
    	\end{Titemize}
    }
	% V 0.1 Ok.
    \UCitem{Entradas}{
        \begin{Titemize}
        	\Titem La \refElem{PlanDeEstudio.division} del \refElem{PlanDeEstudio.planDeEstudio} actual.
        	\Titem El \refElem{Grupo.nombre}.
        	\Titem La \refElem{UdeA.udeA}.
        	\Titem La \refElem{AlumnoAsignado.boleta} ó \refElem{AlumnoAsignado.preboleta} de los estudiantes.
        \end{Titemize}	
    }
	% V0.1 DONE: Indicar que datos se seleccionan y cuales se ingresan, puedes agrupar los que se seleccionan separdos por comas en un solo renglón
	% V0.2 Ok.
    \UCitem{Origen}{
        \begin{Titemize}
        	\Titem Se selecciona con el mouse la \refElem{PlanDeEstudio.division}, el \refElem{Grupo.nombre} y la \refElem{UdeA.udeA}.
        	\Titem Se ingresa desde el teclado o Se importan desde una hoja de cálculo la \refElem{AlumnoAsignado.boleta} o \refElem{AlumnoAsignado.preboleta}.
        \end{Titemize}
    }
	% V 0.1 TODO: Hace falta agregar todos los datos de salida, incluyendo los que llenan los combos de selección, creditos SATCA y TEPIC, lugares disponibles, etc.
	% V 0.2 Ok.
    \UCitem{Salidas}{
    	\begin{Titemize}
    		\Titem El \refElem{ProgramaAcademico.nombre} del programa académico seleccionado.
    		
    		\Titem El \refElem{PlanDeEstudio.nombre} del plan de estudio vigente del programa académico seleccionado.
    		
    		\Titem La \refElem{PeriodoEscolar.clave} del periodo escolar seleccionado.
    		
    		\Titem El \refElem{PlanDeEstudio.tipoDeCreditos} de créditos del plan de estudio vigente.

    		\Titem La \refElem{PlanDeEstudio.cargaMinima} del plan de estudio vigente.

    		\Titem La \refElem{PlanDeEstudio.cargaMedia} del plan de estudio vigente.
    		
    		\Titem La \refElem{PlanDeEstudio.cargaMaxima} del plan de estudio vigente.
    		
    		\Titem Numeración del 1 a ${n}$, donde ${n}$ es el número de niveles/semestres\footnote{Ver \refElem{PlanDeEstudio.division}} del plan de estudio vigente.
    		
    		\Titem El \refElem{Grupo.nombre} de los grupos registrados en la \refElem{EstructuraEducativa} correspondiente al plan de estudio y nivel/semestre seleccionados.
    		
    		\Titem El nombre de cada \refElem{UdeAEnOferta} en el grupo seleccionado.
    		
    		\Titem Los \refElem{UdeA.creditosSATCA} o \refElem{UdeA.creditosTEPIC} de cada unidad de aprendizaje ofertada en el grupo seleccionado dependiendo del tipo de créditos marcados en el Plan de estudios.
    		
    		\Titem La \refElem{GrupoConUdeAEnOferta.capacidad} de cada unidad de aprendizaje ofertada en el grupo seleccionado.
    		
    		\Titem La \refElem{GrupoConUdeAEnOferta.ocupacion} de cada unidad de aprendizaje ofertada en el grupo seleccionado.
    		
    		\Titem El número de sobrecupos que existen en una unidad de aprendizaje ofertada en un grupo, calculado con base en la regla de negocio \refIdElem{BR-IN-N027}.
    		
    		\Titem La \refElem{AlumnoAsignado.boleta} o \refElem{AlumnoAsignado.preboleta} de cada \refElem{tEstudiante} ingresado en el paso \ref{IN-UA-CU3.1.1:selBoletas}.
    		
    		\Titem El \refElem{Alumno.nombre} de cada \refElem{tEstudiante} ingresado en el paso \ref{IN-UA-CU3.1.1:selBoletas}.
    		
    		\Titem La \refElem{AlumnoAsignado.claveEnElProceso} de cada \refElem{tEstudiante} ingresado en el paso \ref{IN-UA-CU3.1.1:selBoletas}.
    		
    		\Titem El número total de inscripciones donde el \refElem{tEstudiante} puede inscribirse en cada \refElem{UdeA.udeA} seleccionadas en el paso \ref{IN-UA-CU3.1.1:selUA}.
    		
    		\Titem Los créditos SATCA de un \refElem{tEstudiante} son el resultado de la suma de los \refElem{UdeA.creditosSATCA} acumulados por cada \refElem{UdeA.udeA} en las que fue posible inscribirlo. 
    		
    		\Titem Los créditos TEPIC de un \refElem{tEstudiante} son el resultado de la suma de los \refElem{UdeA.creditosTEPIC} acumulados por cada \refElem{UdeA.udeA} en las que fue posible inscribirlo.
    	\end{Titemize}	
    }
    % V 0.1 Ok.
    \UCitem{Destino}{Pantalla.}
    % V 0.1 TODO: Que la Unidad Académcia tenga alumnos por inscribir, que la Unidad académica tenga programas académcios vigentes o en transición en el periodo seleccionado, que el programa academico tenga una estructura educativa aprobada en el periodo escolar seleccionado.
    % V 0.3 Ok.
    \UCitem{Precondiciones}{%
    	\begin{Titemize}
    		\Titem \textbf{Sistematizada:} Que la \refElem{tUnidadAcademica} tenga al menos un \refElem{tEstudiante} por inscribir.
    		
    		\Titem \textbf{Sistematizada:} Que la \refElem{tUnidadAcademica} tenga al menos un \refElem{tProgramaAcademico} cuyo plan de estudios se encuentre vigente y ofertado en el \refElem{tPeriodoEscolar} y de la modalidad seleccionados.
    		
    		\Titem \textbf{Sistematizada:} Que el \refElem{tProgramaAcademico} tenga una \refElem{tEstructuraEducativa} aprobada o en proceso de aprobación en el \refElem{tPeriodoEscolar} seleccionado y con al menos un Grupo con una Unidad de Aprendizaje Ofertada.
    	\end{Titemize}
    }
    % V 0.1 TODO: Se inscriben los alumnos seleccionados a las materias seleccionadas, se canbia el estado del alumno, se otorgan los sobrecupos cuando aplican.
    % V 0.3 
    \UCitem{Postcondiciones}{%
    	\begin{Titemize}
    		\Titem Se inscribe cada \refElem{tEstudiante} seleccionado a cada \refElem{tUnidadDeAprendizaje} seleccionada.
    		\Titem Se cambia el estado de cada \refElem{tEstudiante} que se inscribió a inscrito.
    		\Titem Se otorgan los sobrecupos cuando aplican.
			\Titem Se actualiza la disponibilidad de cada Unidad de aprendizaje ofertada en la que se hayan inscrito estudiantes.
    	\end{Titemize}
    }
    % V 0.1 TODO: Terminar la trayectoria primero para poder hacer la relaciónd e reglas.
    % V 0.2 Ok
    \UCitem{Reglas de Negocio}{%
    	\begin{Titemize}
    		\Titem \refIdElem{BR-IN-N024}
    		\Titem \refIdElem{BR-IN-N025}
    		\Titem \refIdElem{BR-IN-N026}
    		\Titem \refIdElem{BR-IN-N027}
    	\end{Titemize}
    }
    % V 0.1 TODO: Solicitar los mensajes que hagan falta y atregar los errores cuando no lah Programas Académicos, periodos, etc.
    % V 0.3 
    \UCitem{Errores}{%
    	\begin{Titemize}
    		\Titem \UCerr{Uno}{Cuando la \refElem{tUnidadAcademica} no tiene ningún \refElem{tEstudiante} sin inscribir a una \refElem{tUnidadDeAprendizaje},}{el sistema muestra el mensaje \refIdElem{MSG181} y termina el caso de uso.}
    		
    		\Titem \UCerr{Dos}{Cuando la \refElem{tUnidadAcademica} no tiene un \refElem{tProgramaAcademico} con plan de estudios vigente en el \refElem{tPeriodoEscolar},}{el sistema muestra el mensaje \refIdElem{MSG182} y termina el caso de uso.}
    		
    		\Titem \UCerr{Tres}{Cuando el Plan de estudios no tiene una \refElem{tEstructuraEducativa} aprobada o en proceso de aprobación,}{el sistema muestra el mensaje \refIdElem{MSG183} y termina el caso de uso.}
    		
	    	\Titem \UCerr{Cuatro}{Cuando no se encuentra \refElem{PlanDeEstudio.cargaMinima}, \refElem{PlanDeEstudio.cargaMedia} o \refElem{PlanDeEstudio.cargaMaxima} registrada,}{el sistema muestra el mensaje \refIdElem{MSG3} donde $<ELEMENTOS>$ = 'Datos de Carga' y termina el caso de uso.}
	    	
    		\Titem \UCerr{Cinco}{Cuando no se encuentra registrada la \refElem{PlanDeEstudio.division} del \refElem{tPlanEstudio} seleccionado,}{el sistema muestra el mensaje \refIdElem{MSG3} donde $<ELEMENTOS>$ = 'Semestres ó Niveles' y termina el caso de uso.}
    		
    		\Titem \UCerr{Seis}{Cuando no se encuentra ningún  \refElem{tGrupo} registrado,}{el sistema muestra el mensaje \refIdElem{MSG3} donde $<ELEMENTOS>$ = 'Grupos' y termina el caso de uso.}
    		
    		\Titem \UCerr{Siete}{Cuando no se encuentra ninguna  \refElem{tUnidadDeAprendizaje} registrada,}{el sistema muestra el mensaje \refIdElem{MSG3} donde $<ELEMENTOS>$ = 'Unidades de Aprendizaje' y termina el caso de uso.}
    		    		
    		\Titem \UCerr{Ocho}{Cuando el número de estudiantes ingresados sobrepasa el 150\% de la capacidad del grupo,}{el sistema muestra el mensaje \refIdElem{MSG188} y continúa en el paso \ref{IN-UA-CU3.1.1:selBoletas}.}
    		
    		\Titem \UCerr{Nueve}{Cuando el número de estudiantes inscritos que tiene una \refElem{tUnidadDeAprendizaje} es mayor o igual al de su capacidad,}{el sistema muestra el mensaje \refIdElem{MSG184} en la columna 'Mensaje' de la tabla mostrada en el paso \ref{IN-UA-CU3.1.1:tabla} y continúa con su trayectoria.}
    		
    		\Titem \UCerr{Diez}{Cuando la cantidad de créditos acumulados sea menor a la \refElem{PlanDeEstudio.cargaMinima},}{el sistema muestra el mensaje \refIdElem{MSG185} en la columna 'Mensaje' de la tabla mostrada en el paso \ref{IN-UA-CU3.1.1:tabla} y continúa con su trayectoria.}
    	\end{Titemize}				
    }
	% V 0.1 TODO: Revisa lo que tien espino para ver que validó ya él y no hagas trabajo doble.
	% V 0.3 Ok.
    \UCitem{Viene de}{\refIdElem{IN-UA-CU3.1}}
    % V 0.1 TODO: No está echo, corregir.
    % V 0.3 Ok.
    \UCitem{Disparadores}{
    	\begin{Titemize}
    		\Titem Requiere inscribir a los Estudiantes (Ver \refElem{tEstudiante}) a las Unidades de Aprendizaje (Ver \refElem{tUnidadDeAprendizaje}) en los Grupos(Ver \refElem{tGrupo}) deseados.
    	\end{Titemize}
    } 
	% V 0.1 Ok.
    \UCitem{Condiciones de Término}{
    	Los Estudiantes (Ver \refElem{tEstudiante}) seleccionados quedan inscritos en las Unidades de Aprendizaje Ofertadas de un \refElem{tGrupo}.}
	% V 0.1 TODO: Se puede quedar el grupo sin lugares, El alumno ya no puede volverse a inscribir mediante esta pantalla si cubre el mínimo de créditos.
	% V 0.3 Ok.
    \UCitem{Efectos Colaterales}{
    	\begin{Titemize}
    		\Titem Se puede quedar sin lugares un \refElem{tGrupo}.
    		\Titem El \refElem{tEstudiante} ya no se puede volver a inscribir mediante esta pantalla si cubre la \refElem{PlanDeEstudio.cargaMinima} de Créditos\footnote{Ver \refElem{tCredito}}.
    	\end{Titemize}
    }
	% V 0.1 Ok.
    \UCitem{Referencia Documental}{}
	% V 0.1 TODO: agregar operación, hora, usuario y descripcion del grupo, alumno y materias instritas.
	% V 0.3 Ok.
    \UCitem{Auditable}{Si, se registra la operación, fecha, hora, usuario que realizo la inscripción, el alumno, unidades de aprendizaje inscritas y los grupos inscritos}
	% V 0.1 TODO: Datos personales , basate en la redacción del UA-CU4
	% V 0.3 Ok.
    \UCitem{Datos sensibles}{La consulta de los datos personales del alumno: Nombre completo y CURP. Este caso de uso habilita a los actores el uso de la CURP y nombre del alumno para su búsqueda e identificación para poder realizar su trabajo que es inscribirlos a las Unidades de Aprendizaje (Ver \refElem{tUnidadDeAprendizaje} de un \refElem{tGrupo} determinado.).}
\end{UseCase}


%Trayectoria Principal : Happy Path
\begin{UCtrayectoria}	
	% V 0.1 Ok.
    \UCpaso[\UCactor] \label{IN-UA-CU3.1.1:solocitarInscribir}Presiona el botón \IUbutton{Inscribir por Grupo} de la pantalla \refIdElem{IN-UA-IU3.1}.
    % V 0.1 DONE: Indicar de donde se toman los creditos y los demas datos, validar todas las precondiciones. aplica para los cuatro pasos
    % Este dato aún no se tiene claro en el modelo de información.
    %\UCpaso  \label{IN-UA-CU3.1.1:obtenerCreditos}Obtiene los tipos de \refElem{tCredito} vigentes en el \refElem{ProgramaAcademico}.\refErr{UnoFail}
    % V 0.3 Ok.
    \UCpaso Valida que la \refElem{tUnidadAcademica} tenga al menos un \refElem{tEstudiante} no inscrito cargado en el sistema con base en la regla de negocio \refIdElem{BR-IN-N024}.\refErr{Uno}
    
    \UCpaso Valida que el Programa académico tenga un Plan de estudios  vigente en el \refElem{tPeriodoEscolar} seleccionado en \refElem{IN-UA-CU3.1} con base en la regla de negocio \refIdElem{BR-IN-N025}.\refErr{Dos}
    
    \UCpaso Valida que el \refElem{tProgramaAcademico} tenga una \refElem{tEstructuraEducativa} aprobada en el \refElem{tPeriodoEscolar} seleccionado en \refElem{IN-UA-CU3.1} con base en la regla de negocio \refIdElem{BR-IN-N026}.\refErr{Tres}
    
    \UCpaso Obtiene la \refElem{PlanDeEstudio.cargaMinima}, \refElem{PlanDeEstudio.cargaMedia} y \refElem{PlanDeEstudio.cargaMaxima} del \refElem{PlanDeEstudio.nombre} vigente.\refErr{Cuatro}

    \UCpaso Obtiene la \refElem{PlanDeEstudio.division} del \refElem{tPlanEstudio} vigente en el \refElem{PeriodoEscolar} seleccionado en \refIdElem{IN-UA-CU1} para determinar los niveles/semestres.\refErr{Cinco}
    
    \UCpaso Muestra la pantalla \refIdElem{IN-UA-IU3.1.1} con la información obtenida.
    % V 0.1 Ok.
    \UCpaso[\UCactor]  \label{IN-UA-CU3.1.1:selNivel}Selecciona un nivel/semestre.
    % V 0.1  Ok.
    \UCpaso Obtiene los grupos que pertenecen al nivel/semestre seleccionado en el paso \ref{IN-UA-CU3.1.1:selNivel}.\refErr{Seis}
    % V 0.1 Ok.
    \UCpaso[\UCactor]  \label{IN-UA-CU3.1.1:selGrupo}Selecciona un grupo.
    % V 0.1 Ok.
    \UCpaso \label{IN-UA-CU3.1.1:obtenerUA}Obtiene el nombre de la \refElem{UdeA.udeA}, los \refElem{UdeA.creditosSATCA}, los \refElem{UdeA.creditosTEPIC}, la \refElem{GrupoConUdeAEnOferta.capacidad} y la \refElem{GrupoConUdeAEnOferta.ocupacion} de cada unidad de aprendizaje que pertenece al grupo seleccionado en el paso \ref{IN-UA-CU3.1.1:selUA}.\refErr{Siete}
    
    % V 0.1 DONE: indicar que hay datos que se calculan, cuales se obtienen y cuales se calculan? para los cálculos se deben marcar las Reglas de Negocio que necesitas.
    % V 0.3 Ok.
    \UCpaso  \label{IN-UA-CU3.1.1:calcularSobrecupo}Calcula la cantidad de sobrecupo que tiene la \refElem{UdeA.udeA} con base en la regla de negocio \refIdElem{BR-IN-N027}.
    % V 0.3 Ok.
    \UCpaso Muestra una tabla con la información obtenida en el paso \ref{IN-UA-CU3.1.1:obtenerUA} y la información calculada en el paso \ref{IN-UA-CU3.1.1:calcularSobrecupo}, mostrándolos de izquierda a derecha:
	    \begin{itemize}
	    	\item unidad de aprendizaje
	    	\item Créditos SATCA
	    	\item Créditos TEPIC
	    	\item Cupo
	    	\item Inscritos
	    	\item Sobrecupo
	    \end{itemize}
    % V 0.1 DONE: Cambia una o muchas por ``las''
    % V 0.3 Ok.
    \UCpaso[\UCactor]  \label{IN-UA-CU3.1.1:selUA}Selecciona las Unidades de Aprendizaje deseadas.
    % V 0.1 DONE: Cambia una o muchas por ``las''.
    % V 0.3 Ok.
    \UCpaso[\UCactor]  \label{IN-UA-CU3.1.1:selBoletas}Ingresa las Boletas\footnote{Ver \refElem{AlumnoAsignado.boleta}}/Preboletas\footnote{Ver \refElem{AlumnoAsignado.preboleta}} deseadas.
	% V 0.1 DONE: solo se validará que sean cadenas separadas por espacios comas tabuladores o enter.
	% V 0.3 Ok.
    \UCpaso Valida que la información ingresada en el paso \ref{IN-UA-CU3.1.1:selBoletas} sean cadenas separadas por espacios, comas, tabuladores o enter.
    % V 0.3 Ok.
    \UCpaso Valida que el número de Boletas\footnote{Ver \refElem{AlumnoAsignado.boleta}}/Preboletas\footnote{Ver \refElem{AlumnoAsignado.preboleta}} ingresadas en el paso \ref{IN-UA-CU3.1.1:selBoletas} no exceda el 150\% de la \refElem{GrupoConUdeAEnOferta.capacidad} del grupo con base en la regla de negocio \refIdElem{BR-IN-N027}.\refErr{Ocho}
    % V 0.1 Ok.
    \UCpaso Habilita el botón \IUbutton{Agregar al grupo}.
    % V 0.1 Ok.    
    \UCpaso[\UCactor] \label{IN-UA-CU3.1.1:agregarAlGrupo}Presiona el botón \IUbutton{Agregar al grupo}.
    % V 0.1 DONE: No cimula, busca sus datos, lo agrega al grupo y lleva la cuenta de lugares ocupados para marcar cuando un alumno ya no cabe en el grupo
    \UCpaso \label{IN-UA-CU3.1.1:obtenerEstudiante}Para cada boleta o preboleta ingresados en el paso \ref{IN-UA-CU3.1.1:selBoletas} obtiene sus datos (\refElem{AlumnoAsignado.boleta} o \refElem{AlumnoAsignado.preboleta}, \refElem{Alumno.nombre}, \refElem{AlumnoAsignado.claveEnElProceso}).
    % V 0.3 Ok.
    \UCpaso \label{IN-UA-CU3.1.1:calcularOcupabilidad}Agrega cada \refElem{tEstudiante} a cada Unidad de Aprendizaje Ofertada ingresada en el paso \ref{IN-UA-CU3.1.1:selUA}, llevando la cuenta de la Unidades de Aprendizaje a las que es posible agregarlo, y llevando la cuenta de los lugares ocupados en la \refElem{tUnidadDeAprendizaje} para verificar que éstos sean menor que su \refElem{GrupoConUdeAEnOferta.capacidad} con base en la regla de negocio \refIdElem{BR-IN-N027}.\refErr{Nueve} \refErr{Diez}
    % V 0.3 Ok.
    \UCpaso \label{IN-UA-CU3.1.1:calcularCreditosSATCA}Calcula la cantidad de créditos SATCA o TEPIC (según sea el caso) de cada \refElem{tEstudiante} sumando los \refElem{UdeA.creditosSATCA} SATCA o TEPIC acumulados por cada \refElem{UdeA.udeA} en las que fue posible agregarlo.
    % V 0.3 Ok.
    \UCpaso  \label{IN-UA-CU3.1.1:calcularCreditosTEPIC}Calcula la cantidad de créditos SATCA o TEPIC de cada \refElem{tEstudiante} sumando los creditos acumulados por cada \refElem{UdeA.udeA} en las que fue posible agregarlo.
	% Marcar con trayectorias alternativas los mensajes que llegan a ocurrir.
    % TOCHK El estandar que se esta utilizando en los procesos es que si sólo se muestra un mensaje, se maneje como error.
    % V 0.3 Ok.
    \UCpaso  \label{IN-UA-CU3.1.1:tabla}Muestra una tabla con la información obtenida en los pasos \ref{IN-UA-CU3.1.1:obtenerEstudiante}, \ref{IN-UA-CU3.1.1:calcularOcupabilidad}, \ref{IN-UA-CU3.1.1:calcularCreditosSATCA} y \ref{IN-UA-CU3.1.1:calcularCreditosTEPIC}, mostrando la misma de izquierda a derecha:
	    \begin{itemize}
	    	\item Boleta / Preboleta
	    	\item Nombre
	    	\item Tipo
	    	\item No. de Materias
	    	\item Créditos SATCA
	    	\item Créditos TEPIC
	    	\item Mensaje
	    	\item Acciones
	    \end{itemize}
    % V 0.1 Ok.
    \UCpaso Actualiza el número de estudiantes inscritos de cada \refElem{tUnidadDeAprendizaje}, sumándole la cuenta de los lugares ocupados calculados en el paso \ref{IN-UA-CU3.1.1:calcularOcupabilidad}.
    % V 0.3 TODO: Agrega un paso para indicar que si no alcanza el mínimo de creditos se muestra el mensaje que indica que ``no cubre carga mínima''.

    % V 0.1 Ok.    
    \UCpaso[\UCactor]  \label{IN-UA-CU3.1.1:inscribir}Presiona el botón \IUbutton{Inscribir}.\refTray{A} \refTray{B} \refTray{C}
    % V 0.1 DONE: Calcula los datos del mensaje primero.
    % V 0.3 Ok.
    \UCpaso Calcula el número de \refElem{tEstudiante} que se inscribirán al \refElem{tGrupo} seleccionado en el paso \ref{IN-UA-CU3.1.1:selGrupo}.
    % V 0.3 Ok.
    \UCpaso Muestra el mensaje \refIdElem{MSG186} solicitando la confirmación de la inscripción.
    % V 0.1 Ok.    
    \UCpaso[\UCactor]  \label{IN-UA-CU3.1.1:confirmar}Presiona el botón \IUbutton{Si}.\refTray{E}
    % V 0.1 DONE: Actualiza el estado, actualiza los lugares disponibles, otorga sobre cupos si es necesario.
    % V 0.3 Ok.
    \UCpaso Registra a los estudiantes en las Unidades de Aprendizaje\footnote{Ver \refElem{tUnidadDeAprendizaje}} seleccionadas en el paso \ref{selUA}.
    % V 0.3 Ok.
    \UCpaso Actualiza el estado de los \refElem{tEstudiante} en las Unidades de Aprendizaje\footnote{Ver \refElem{tUnidadDeAprendizaje}} seleccionadas en el paso \ref{selUA}.
    % V 0.3 Ok.
    \UCpaso Actualiza la \refElem{GrupoConUdeAEnOferta.ocupacion} de cada \refElem{UdeA.udeA}, otorgando sobrecupos si es necesario en el \refElem{tGrupo} seleccionado en el paso \ref{IN-UA-CU3.1.1:selGrupo}.
\end{UCtrayectoria}

%Trayectoria Alternativas

%----------------- A
\begin{UCtrayectoriaA}{A}{El actor requiere cancelar la inscripción de un estudiante.}
	% V 0.3 Ok.
	\UCpaso[\UCactor]  \label{IN-UA-CU3.1.1:eliminarEstudiante}Da clic sobre el ícono \IUBorrar.
	% V 0.3 Ok.
	\UCpaso Actualiza el número de estudiantes inscritos de cada \refElem{tUnidadDeAprendizaje} en la que el \refElem{tEstudiante} iba a ser inscrito, restándole 1.
	% V 0.3 Ok.
	\UCpaso Elimina la información del \refElem{tEstudiante} seleccionado en el paso \ref{IN-UA-CU3.1.1:eliminarEstudiante} de la tabla generada en el paso \ref{IN-UA-CU3.1.1:tabla}.
	% V 0.3 Ok.
	\UCpaso Continúa en el paso \ref{IN-UA-CU3.1.1:inscribir}  de la trayectoria principal.
\end{UCtrayectoriaA}

%----------------- B
\begin{UCtrayectoriaA}{B}{El actor requiere dar sobrecupo a un estudiante.}
	\UCpaso[\UCactor]  \label{IN-UA-CU3.1.1:sobrecupo}Da clic sobre el ícono \IUContenido.
	% V 0.3 Ok.
	\UCpaso Muestra la pantalla \refIdElem{IN-UA-IU3.1} con las Unidades de Aprendizaje\footnote{Ver \refElem{tUnidadDeAprendizaje}} obtenidas en el paso \ref{IN-UA-CU3.1.1:obtenerUA}.
	% V 0.3 Ok.
	\UCpaso[\UCactor]  \label{IN-UA-CU3.1.1:sobrecupoSelUA}Selecciona las Unidades de Aprendizaje\footnote{Ver \refElem{tUnidadDeAprendizaje}} en las que desea dar sobrecupo.
	% V 0.3 Ok.
	\UCpaso[\UCactor] Presiona el botón \IUbutton{Aceptar}.\refTray{D}
	% V 0.3 Ok.
	\UCpaso Actualiza el número de estudiantes inscritos y el número de \refElem{GrupoConUdeAEnOferta.capacidad} de cada \refElem{tUnidadDeAprendizaje} seleccionadas en el paso \ref{IN-UA-CU3.1.1:sobrecupoSelUA}, sumándole 1.
	% V 0.3 Ok.
	\UCpaso Actualiza el estatus de la inscripción del \refElem{tEstudiante} en la tabla generada en el paso \ref{IN-UA-CU3.1.1:tabla}.
	% V 0.3 Ok.
	\UCpaso Continúa en el paso \ref{IN-UA-CU3.1.1:inscribir}  de la trayectoria principal.
\end{UCtrayectoriaA}

%----------------- C
\begin{UCtrayectoriaA}[Termina el caso de uso]{C}{El actor requiere cancelar la inscripción.}
	% V 0.3 Ok.    
	\UCpaso[\UCactor] Presiona el botón \IUbutton{Cancelar}.
	% V 0.3 Ok.    
	\UCpaso Muestra la pantalla \refIdElem{IN-UA-IU3.1}.
\end{UCtrayectoriaA}

%----------------- D
\begin{UCtrayectoriaA}{D}{El actor requiere cancelar la asignación de sobrecupo.}
	% V 0.1 Ok.    
	\UCpaso[\UCactor] Presiona el botón \IUbutton{Cancelar}.
	% V 0.1 Ok.    
	\UCpaso Continúa en el paso \ref{IN-UA-CU3.1.1:inscribir} de la trayectoria principal.
\end{UCtrayectoriaA}

%----------------- E
\begin{UCtrayectoriaA}{E}{El actor requiere cancelar la confirmación de la inscripción.}
	% V 0.3 Ok.    
	\UCpaso[\UCactor] Presiona el botón \IUbutton{No}.
	% V 0.1 Ok.    
	\UCpaso Continúa en el paso \ref{IN-UA-CU3.1.1:inscribir} de la trayectoria principal.
\end{UCtrayectoriaA}

