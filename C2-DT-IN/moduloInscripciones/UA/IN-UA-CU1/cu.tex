\begin{UseCase}{IN-UA-CU1}{Selección de ciclo escolar y modalidad}{
	% V 0.1 Ok.
	Permite seleccionar una \refElem{tModalidad} y \refElem{CicloEscolar} para poder trabajar en la configuración de los elementos del sistema de una manera ordenada: Consultar el histórico de Ciclos escolares anteriores, Monitorear y gestionar lo relativo al ciclo escolar actual o planear lo relativo a el ciclo escolar próximo.
}
    \UCccitem{Versión}{0.1}
    \UCccsection{Datos para el control Interno}	
    \UCccitem{Elaboró}{Bruno Suárez Cruz}
    \UCccitem{Supervisó}{Ulises Vélez Saldaña}
    \UCccitem{Operación}{Seleccionar}
    \UCccitem{Prioridad}{Baja}
    \UCccitem{Complejidad}{Baja}
    \UCccitem{Volatilidad}{Baja}
    \UCccitem{Madurez}{Alta}
    \UCccitem{Estatus}{Revisado}
    \UCccitem{Dificultades}{}
    \UCccitem{Fecha del último estatus}{05 de Diciembre de 2017}
    \UCccsection{Revisión de la versión 0.1}
    \UCccitem{Fecha}{05 de Diciembre de 2017}
    \UCccitem{Evaluador}{Ulises Vélez Saldaña}
    \UCccitem{Resultado}{Aprueba la revisión}
    \UCccitem{Observaciones}{Ninguna}
    \UCsection{Atributos}
    % V 0.1 Ok.
    \UCitem{Actor}{\refElem{UAJefeDeGestionEscolar}}
    % V 0.1 Ok.
    \UCitem{Propósito}{%
    	\begin{Titemize}
    		\Titem Consultar el histórico de Ciclos escolares anteriores.
    		\Titem Monitorear y gestionar lo relativo al ciclo escolar actual.
    		\Titem Planear lo relativo a el ciclo escolar próximo.
    	\end{Titemize}
    }
    % V 0.1 Ok.
    \UCitem{Entradas}{%
        \begin{Titemize}
        	\Titem \refElem{tModalidad}.
        	\Titem \refElem{CicloEscolar}.	
        \end{Titemize}	
    }
    % V 0.1 Ok.
    \UCitem{Origen}{%
        \begin{Titemize}
        	\Titem \ioObtener
        	\Titem Se selecciona con el mouse.
        \end{Titemize}
    }
    % V 0.1 Ok.
    \UCitem{Salidas}{%
    	\begin{Titemize}
    		\Titem El nombre de las Modalidades registradas en el sistema (vea \refElem{tModalidad}).
    		\Titem El nombre de los Ciclos escolares disponibles para selección (vea \refElem{CicloEscolar.clave}).
    	\end{Titemize}	
    }
    % V 0.1 Ok.
    \UCitem{Destino}{Pantalla.}
    % V 0.1 Ok.
    \UCitem{Precondiciones}{%
    	\begin{Titemize}
    		\Titem \textbf{Sistematizada:} Que exista por lo menos un ciclo escolar.
    		\Titem \textbf{Sistematizada:} Que exista por lo menos una Modalidad.
    	\end{Titemize}
    }
    % V 0.1 Ok.
    \UCitem{Postcondiciones}{%
    	\begin{Titemize}
    		\Titem La modalidad y Ciclo escolar seleccionados determinarán toda la información y operaciones para los demás casos de uso de este módulo.
    	\end{Titemize}
    }
    % V 0.1 Ok.
    \UCitem{Reglas de Negocio}{%
    	\begin{Titemize}
    		\Titem \refIdElem{BR-IN-N005}
    	\end{Titemize}
    }
    % V 0.1 Ok.
    \UCitem{Errores}{%
    	\begin{Titemize}
    		\Titem \UCerr{Uno}{Cuando no se encuentran  \textbf{Ciclos escolares} o {\bf Modalidades} registrados,}{ el sistema muestra el mensaje \refIdElem{MSG3} para Modalidades y Ciclos escolares en la pantalla \refIdElem{IN-UA-IU1} y termina el caso de uso.}
    	\end{Titemize}				
    }
    % V 0.1 Ok.
    \UCitem{Viene de}{\refIdElem{CU-Login}}
    % V 0.1 Ok.
    \UCitem{Disparadores}{%
    	\begin{Titemize}
    		\Titem Requiere consultar el histórico de Ciclos escolares anteriores.
    		\Titem Requiere monitorear y gestionar lo relativo al ciclo escolar actual.
    		\Titem Requiere planear lo relativo a el ciclo escolar próximo.
    	\end{Titemize}
    } 
    % V 0.1 Ok.
    \UCitem{Condiciones de Término}{%
    	Se muestra en las pantallas siguientes el ciclo escolar y la modalidad seleccionados.}
    % V 0.1 Ok.
    \UCitem{Efectos Colaterales}{Ninguno}
    % V 0.1 Ok.
    \UCitem{Referencia Documental}{}
    % V 0.1 Ok.
    \UCitem{Auditable}{No}
    % V 0.1 Ok.
    \UCitem{Datos sensibles}{Ninguna}
\end{UseCase}


%Trayectoria Principal : Happy Path
\begin{UCtrayectoria}	
	% V 0.1 Ok.
    \UCpaso[\UCactor] Da clic en la opción \textbf{Selección de ciclo escolar y modalidad} del menú \textbf{Ciclo escolar y modalidad}. (\refElem{IN-UA-MN1})
    % V 0.1 Ok.
    \UCpaso Obtiene las Modalidades registradas asociadas a la Unidad Académica.\refErr{Uno}
    % V 0.1 Ok.
    \UCpaso Obtiene los últimos 5 ciclos escolares anteriores al actual, el ciclo actual y 2 ciclos escolares posteriores con base a la regla de negocio \refIdElem{BR-IN-N005}.
    % V 0.1 Ok.    
    \UCpaso Muestra la pantalla \refIdElem{IN-UA-IU1} con la información obtenida.
    % V 0.1 Ok.
    \UCpaso [\UCactor] Selecciona un ciclo escolar.
    % V 0.1 Ok.
    \UCpaso [\UCactor] Selecciona una modalidad.
    % V 0.1 Ok.
    \UCpaso [\UCactor] \label{UA-IN-CU1:acp} Presiona el botón \IUbutton{Aceptar}. 
    \UCpaso Establece en sesión el {\bf Ciclo Escolar Seleccionado} y la {\bf Modalidad seleccionada} con los valores seleccionados.
    % V 0.1 Ok.
    \UCpaso Habilita el menú \refElem{IN-UA-MN1}.    
\end{UCtrayectoria}

%%Trayectoria Alternativas
%%----------------- A
%\begin{UCtrayectoriaA}[Termina el caso de uso]{A}{El actor requiere cancelar la operación.}
%\UCpaso Presiona el botón \IUbutton{Cancelar}.
%\UCpaso Muestra la pantalla \refIdElem{IN-UA-IU1}.
%\end{UCtrayectoriaA}
