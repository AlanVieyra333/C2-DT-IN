\begin{UseCase}{IN-UA-CU3.1.2}{Inscribir alumnos por alumno}{
	Permite inscribir a un alumno a una \refElem{UdeA.udeA} perteneciente a un \refElem{Grupo} determinado, durante el \refElem{PeriodoEscolar} previamente seleccionado.
}
    \UCccitem{Versión}{0.1}
    \UCccsection{Datos para el control Interno}	
    \UCccitem{Elaboró}{Alan Fernando Rincón Vieyra}
    \UCccitem{Supervisó}{Ulises Vélez Saldaña}
    \UCccitem{Operación}{Registro}
    \UCccitem{Prioridad}{Alta}
    \UCccitem{Complejidad}{Media}
    \UCccitem{Volatilidad}{Baja}
    \UCccitem{Madurez}{Baja}
    \UCccitem{Estatus}{Revisión}
    \UCccitem{Dificultades}{}
    \UCccitem{Fecha del último estatus}{14 de Febrero de 2017}
    \UCccitem{Fecha}{}
    \UCccitem{Evaluador}{}
    \UCccitem{Resultado}{}
    \UCccitem{Observaciones}{}
    \UCsection{Atributos}

    \UCitem{Actor}{
    	\begin{Titemize}
    		\Titem \refElem{UAJefeDeGestionEscolar}
    	\end{Titemize}
    }

    \UCitem{Propósito}{
    	\begin{Titemize}
    		\Titem Inscribir un alumno a las Unidades de Aprendizaje en un grupo.
    	\end{Titemize}
    }

    \UCitem{Entradas}{
    	\begin{Titemize}
	    	\Titem La \refElem{AlumnoAsignado.boleta} o \refElem{AlumnoAsignado.preboleta} del estudiante a inscribir.
    		\Titem La \refElem{PlanDeEstudio.division} del \refElem{PlanDeEstudio.planDeEstudio} actual.
    		\Titem El \refElem{Grupo.nombre}.
    		\Titem La \refElem{UdeA.udeA}.
    	\end{Titemize}	
    }
    
    \UCitem{Origen}{
    	\begin{Titemize}
    		\Titem Se selecciona con el mouse la \refElem{PlanDeEstudio.division}, el \refElem{Grupo.nombre} y la \refElem{UdeA.udeA}.
    		\Titem Se ingresa desde el teclado la \refElem{AlumnoAsignado.boleta} o \refElem{AlumnoAsignado.preboleta}.
    	\end{Titemize}
    }
    %TODO
    \UCitem{Salidas}{
    	\begin{Titemize}
    		\Titem El \refElem{ProgramaAcademico.nombre} del programa académico seleccionado.
    		
    		\Titem El \refElem{PlanDeEstudio.nombre} del plan de estudio vigente del programa académico seleccionado.
    		
    		\Titem La \refElem{PeriodoEscolar.clave} del periodo escolar seleccionado.
    		
    		\Titem El \refElem{PlanDeEstudio.tipoDeCreditos} de créditos del plan de estudio vigente.
    		
    		\Titem La \refElem{PlanDeEstudio.cargaMinima} del plan de estudio vigente.
    		
    		\Titem La \refElem{PlanDeEstudio.cargaMedia} del plan de estudio vigente.
    		
    		\Titem La \refElem{PlanDeEstudio.cargaMaxima} del plan de estudio vigente.
    		
    		\Titem Numeración del 1 a ${n}$, donde ${n}$ es el número de niveles/semestres\footnote{Ver \refElem{PlanDeEstudio.division}} del plan de estudio vigente.
    		
    		\Titem El \refElem{Grupo.nombre} de los grupos registrados en la \refElem{EstructuraEducativa} correspondiente al plan de estudio y nivel/semestre seleccionados.
    		
    		\Titem El nombre de cada \refElem{UdeAEnOferta} en el grupo seleccionado.
    		
    		\Titem Los \refElem{UdeA.creditosSATCA} o \refElem{UdeA.creditosTEPIC} de cada unidad de aprendizaje ofertada en el grupo seleccionado dependiendo del tipo de créditos marcados en el Plan de estudios.
    		
    		\Titem La \refElem{GrupoConUdeAEnOferta.capacidad} de cada unidad de aprendizaje ofertada en el grupo seleccionado.
    		
    		\Titem La \refElem{GrupoConUdeAEnOferta.ocupacion} de cada unidad de aprendizaje ofertada en el grupo seleccionado.
    		
    		\Titem El número de sobrecupos que existen en una unidad de aprendizaje ofertada en un grupo, calculado con base en la regla de negocio \refIdElem{BR-IN-N027}.
    		
    		\Titem La \refElem{AlumnoAsignado.boleta} o \refElem{AlumnoAsignado.preboleta} de cada \refElem{tEstudiante} ingresado en el paso \ref{IN-UA-CU3.1.2:selBoletas}.
    		
    		\Titem El \refElem{Alumno.nombre} de cada \refElem{tEstudiante} ingresado en el paso \ref{IN-UA-CU3.1.2:selBoletas}.
    		
    		\Titem La \refElem{AlumnoAsignado.claveEnElProceso} de cada \refElem{tEstudiante} ingresado en el paso \ref{IN-UA-CU3.1.2:selBoletas}.
    		
    		\Titem El número total de inscripciones donde el \refElem{tEstudiante} puede inscribirse en cada \refElem{UdeA.udeA} seleccionadas en el paso \ref{IN-UA-CU3.1.2:selUA}.
    		
    		\Titem Los créditos SATCA de un \refElem{tEstudiante} son el resultado de la suma de los \refElem{UdeA.creditosSATCA} acumulados por cada \refElem{UdeA.udeA} en las que fue posible inscribirlo. 
    		
    		\Titem Los créditos TEPIC de un \refElem{tEstudiante} son el resultado de la suma de los \refElem{UdeA.creditosTEPIC} acumulados por cada \refElem{UdeA.udeA} en las que fue posible inscribirlo.
    	\end{Titemize}	
    }

    \UCitem{Destino}{Pantalla.}

    \UCitem{Precondiciones}{
    	\begin{Titemize}
    		\Titem \textbf{Sistematizada:} Que el \refElem{tProgramaAcademico} tenga una \refElem{tEstructuraEducativa} aprobada o en proceso de aprobación en el \refElem{tPlanEstudio} y \refElem{tPeriodoEscolar} seleccionado.
    		
    		\Titem \textbf{Sistematizada:} Que la \refElem{tUnidadAcademica} tenga una \refElem{UdeAEnOferta} en el \refElem{Grupo} seleccionado.
    		
    		\Titem \textbf{Sistematizada:} Que la \refElem{tUnidadAcademica} tenga un \refElem{tEstudiante} por inscribir en el \refElem{tProgramaAcademico} seleccionado.
    	\end{Titemize}
    }
    %TODO
    \UCitem{Postcondiciones}{
    	\begin{Titemize}
    		\Titem Se inscribe cada \refElem{tEstudiante}. seleccionado a cada \refElem{tUnidadDeAprendizaje} seleccionada.
    		\Titem Se cambia el estado de cada \refElem{tEstudiante} que se inscribió.
    		\Titem Se otorgan los sobrecupos cuando aplican.
    	\end{Titemize}
    }
    %TODO
    \UCitem{Reglas de Negocio}{
    	\begin{Titemize}
    		\Titem \refIdElem{BR-IN-N024}
    		\Titem \refIdElem{BR-IN-N025}
    		\Titem \refIdElem{BR-IN-N026}
    		\Titem \refIdElem{BR-IN-N027}
    	\end{Titemize}
    }
    %TODO
    \UCitem{Errores}{
    	\begin{Titemize}
    		\Titem \UCerr{Uno}{Cuando la \refElem{tUnidadAcademica} no tiene ningún \refElem{tEstudiante} sin inscribir a una \refElem{tUnidadDeAprendizaje},}{el sistema muestra el mensaje \refIdElem{MSG181} y termina el caso de uso.}
    		
    		\Titem \UCerr{Dos}{Cuando la \refElem{tUnidadAcademica} no tiene un \refElem{tProgramaAcademico} vigente o en transición en el \refElem{tPeriodoEscolar},}{el sistema muestra el mensaje \refIdElem{MSG182} y termina el caso de uso.}
    		
    		\Titem \UCerr{Tres}{Cuando el \refElem{tProgramaAcademico} no tiene una \refElem{tEstructuraEducativa} aprobada o en proceso de aprobación,}{el sistema muestra el mensaje \refIdElem{MSG183} y termina el caso de uso.}
    		
    		\Titem \UCerr{Cuatro}{Cuando no se encuentra \refElem{PlanDeEstudio.cargaMinima}, \refElem{PlanDeEstudio.cargaMedia} o \refElem{PlanDeEstudio.cargaMaxima} registrada,}{el sistema muestra el mensaje \refIdElem{MSG3} donde $<ELEMENTOS>$ = 'Datos de Carga' y termina el caso de uso.}
    		
    		\Titem \UCerr{Cinco}{Cuando no se encuentra registrada la \refElem{PlanDeEstudio.division} del \refElem{tPlanEstudio} seleccionado,}{el sistema muestra el mensaje \refIdElem{MSG3} donde $<ELEMENTOS>$ = 'Semestres ó Niveles' y termina el caso de uso.}
    		
    		\Titem \UCerr{Seis}{Cuando no se encuentra ningún  \refElem{tGrupo} registrado,}{el sistema muestra el mensaje \refIdElem{MSG3} donde $<ELEMENTOS>$ = 'Grupos' y termina el caso de uso.}
    		
    		\Titem \UCerr{Siete}{Cuando no se encuentra ninguna  \refElem{tUnidadDeAprendizaje} registrada,}{el sistema muestra el mensaje \refIdElem{MSG3} donde $<ELEMENTOS>$ = 'Unidades de Aprendizaje' y termina el caso de uso.}
    		
    		\Titem \UCerr{Ocho}{Cuando el número de estudiantes ingresados sobrepasa el 150\% de la capacidad del grupo,}{el sistema muestra el mensaje \refIdElem{MSG188} y continúa en el paso \ref{IN-UA-CU3.1.2:selBoletas}.}
    		
    		\Titem \UCerr{Nueve}{Cuando el número de estudiantes inscritos que tiene una \refElem{tUnidadDeAprendizaje} es mayor o igual al de su capacidad,}{el sistema muestra el mensaje \refIdElem{MSG184} en la columna 'Mensaje' de la tabla mostrada en el paso \ref{IN-UA-CU3.1.2:tabla} y continúa con su trayectoria.}
    		
    		\Titem \UCerr{Diez}{Cuando la cantidad de créditos acumulados sea menor a la \refElem{PlanDeEstudio.cargaMinima},}{el sistema muestra el mensaje \refIdElem{MSG185} en la columna 'Mensaje' de la tabla mostrada en el paso \ref{IN-UA-CU3.1.2:tabla} y continúa con su trayectoria.}
    	\end{Titemize}				
    }

    \UCitem{Viene de}{\refIdElem{IN-UA-CU3.1}}

    \UCitem{Disparadores}{
    	\begin{Titemize}
    		\Titem Requiere inscribir un \refElem{tEstudiante} a las Unidades de Aprendizaje de los Grupos deseados.
    	\end{Titemize}
    } 
    %TODO
    \UCitem{Condiciones de Término}{
    	Los Estudiantes\footnote{Ver \refElem{tEstudiante}} seleccionados quedan inscritos en las Unidades de Aprendizaje\footnote{Ver \refElem{tUnidadDeAprendizaje}} de un \refElem{tGrupo}.}
    %TODO
    \UCitem{Efectos Colaterales}{
    	\begin{Titemize}
    		\Titem Se puede quedar sin lugares un \refElem{tGrupo}.
    		\Titem El \refElem{tEstudiante} ya no se puede volver a inscribir mediante esta pantalla si cubre la \refElem{PlanDeEstudio.cargaMinima} de Créditos\footnote{Ver \refElem{tCredito}}.
    	\end{Titemize}
    }

    \UCitem{Referencia Documental}{}

    \UCitem{Auditable}{Si, se registra la operación, fecha, hora, usuario que realizo la inscripción, el alumno, unidades de aprendizaje inscritas y los grupos inscritos}

    \UCitem{Datos sensibles}{La consulta de los datos personales del alumno: Nombre completo. Este caso de uso habilita a los actores el uso del nombre del alumno para su búsqueda e identificación para poder realizar su trabajo que es inscribirlos a las unidades de aprendizaje de un grupo determinado.}
\end{UseCase}

%Trayectoria Principal : Happy Path
%TODO
\begin{UCtrayectoria}	
	% V 0.1 Ok.
	\UCpaso[\UCactor]  \label{IN-UA-CU3.1.2:solocitarInscribir}Presiona el botón \IUbutton{Inscribir por Grupo} de la pantalla \refIdElem{IN-UA-IU3.1}.
	% V 0.1 TODO: Indicar de donde se toman los creditos y los demas datos, validar todas las precondiciones. aplica para los cuatro pasos
	% Este dato aún no se tiene claro en el modelo de información.
	%\UCpaso  \label{IN-UA-CU3.1.2:obtenerCreditos}Obtiene los tipos de \refElem{tCredito} vigentes en el \refElem{ProgramaAcademico}.\refErr{UnoFail}
	
	\UCpaso Valida que la \refElem{tUnidadAcademica} tenga al menos un \refElem{tEstudiante} no inscrito cargado en el sistema con base en la regla de negocio \refIdElem{BR-IN-N024}.\refErr{Uno}
	
	\UCpaso Valida que la \refElem{tUnidadAcademica} tenga un \refElem{tProgramaAcademico} vigente o en transición en el \refElem{tPeriodoEscolar} seleccionado en \refElem{IN-UA-CU3.1} con base en la regla de negocio \refIdElem{BR-IN-N025}.\refErr{Dos}
	
	\UCpaso Valida que el \refElem{tProgramaAcademico} tenga una \refElem{tEstructuraEducativa} aprobada en el \refElem{tPeriodoEscolar} seleccionado en \refElem{IN-UA-CU3.1} con base en la regla de negocio \refIdElem{BR-IN-N026}.\refErr{Tres}
	
	\UCpaso Obtiene la \refElem{PlanDeEstudio.cargaMinima}, \refElem{PlanDeEstudio.cargaMedia} y \refElem{PlanDeEstudio.cargaMaxima} del \refElem{PlanDeEstudio.nombre} vigente.\refErr{Cuatro}
	
	\UCpaso Obtiene la \refElem{PlanDeEstudio.division} del \refElem{tPlanEstudio} vigente en el \refElem{PeriodoEscolar} seleccionado en \refIdElem{IN-UA-CU1} para determinar los niveles/semestres.\refErr{Cinco}
	
	\UCpaso Muestra la pantalla \refIdElem{IN-UA-IU3.1.1} con la información obtenida.
	% V 0.1 Ok.
	\UCpaso[\UCactor]  \label{IN-UA-CU3.1.2:selNivel}Selecciona un nivel/semestre.
	% V 0.1  Ok.
	\UCpaso Obtiene los grupos que pertenecen al nivel/semestre seleccionado en el paso \ref{IN-UA-CU3.1.2:selNivel}.\refErr{Seis}
	% V 0.1 Ok.
	\UCpaso[\UCactor]  \label{IN-UA-CU3.1.2:selGrupo}Selecciona un grupo.
	% V 0.1 Ok.
	\UCpaso  \label{IN-UA-CU3.1.2:obtenerUA}Obtiene el nombre de la \refElem{UdeA.udeA}, los \refElem{UdeA.creditosSATCA}, los \refElem{UdeA.creditosTEPIC}, la \refElem{GrupoConUdeAEnOferta.capacidad} y la \refElem{GrupoConUdeAEnOferta.ocupacion} de cada unidad de aprendizaje que pertenece al grupo seleccionado en el paso \ref{IN-UA-CU3.1.2:selUA}.\refErr{Siete}
	
	% V 0.1 TODO: indicar que hay datos que se calculan, cuales se obtienen y cuales se calculan? para los cálculos se deben marcar las Reglas de Negocio que necesitas.
	\UCpaso  \label{IN-UA-CU3.1.2:calcularSobrecupo}Calcula la cantidad de sobrecupo que tiene la \refElem{UdeA.udeA} con base en la regla de negocio \refIdElem{BR-IN-N027}.
	
	\UCpaso Muestra una tabla con la información obtenida en el paso \ref{IN-UA-CU3.1.2:obtenerUA} y la información calculada en el paso \ref{IN-UA-CU3.1.2:calcularSobrecupo}, mostrándolos de izquierda a derecha:
	\begin{itemize}
		\item unidad de aprendizaje
		\item Créditos SATCA
		\item Créditos TEPIC
		\item Cupo
		\item Inscritos
		\item Sobrecupo
	\end{itemize}
	% V 0.1 TODO: Cambia una o muchas por ``las''
	\UCpaso[\UCactor]  \label{IN-UA-CU3.1.2:selUA}Selecciona las Unidades de Aprendizaje\footnote{Ver \refElem{tUnidadDeAprendizaje}} deseadas.
	% V 0.1 TODO: Cambia una o muchas por ``las''.
	\UCpaso[\UCactor]  \label{IN-UA-CU3.1.2:selBoletas}Ingresa las Boletas\footnote{Ver \refElem{AlumnoAsignado.boleta}}/Preboletas\footnote{Ver \refElem{AlumnoAsignado.preboleta}} deseadas.
	% V 0.1 TODO: solo se validará que sean cadenas separadas por espacios comas tabuladores o enter.
	\UCpaso Valida que la información ingresada en el paso \ref{IN-UA-CU3.1.2:selBoletas} sean cadenas separadas por espacios, comas, tabuladores o enter.
	
	\UCpaso Valida que el número de Boletas\footnote{Ver \refElem{AlumnoAsignado.boleta}}/Preboletas\footnote{Ver \refElem{AlumnoAsignado.preboleta}} ingresadas en el paso \ref{IN-UA-CU3.1.2:selBoletas} no exceda el 150\% de la \refElem{GrupoConUdeAEnOferta.capacidad} del grupo con base en la regla de negocio \refIdElem{BR-IN-N027}.\refErr{Ocho}
	% V 0.1 Ok.
	\UCpaso Habilita el botón \IUbutton{Agregar al grupo}.
	% V 0.1 Ok.    
	\UCpaso[\UCactor] \label{IN-UA-CU3.1.2:agregarAlGrupo}Presiona el botón \IUbutton{Agregar al grupo}.
	% V 0.1 TODO: No cimula, busca sus datos, lo agrega al grupo y lleva la cuenta de lugares ocupados para marcar cuando un alumno ya no cabe en el grupo
	\UCpaso  \label{IN-UA-CU3.1.2:obtenerEstudiante}Para cada \refElem{tEstudiante} ingresado en el paso \ref{IN-UA-CU3.1.2:selBoletas} obtiene sus datos (\refElem{AlumnoAsignado.boleta} ó \refElem{AlumnoAsignado.preboleta}, \refElem{Alumno.nombre}, \refElem{AlumnoAsignado.claveEnElProceso}).
	
	\UCpaso  \label{IN-UA-CU3.1.2:calcularOcupabilidad}Agrega a cada \refElem{tEstudiante} a cada \refElem{tUnidadDeAprendizaje} ingresada en el paso \ref{IN-UA-CU3.1.2:selUA}, llevando la cuenta de la Unidades de Aprendizaje a las que es posible agregarlo, y llevando la cuenta de los lugares ocupados en la \refElem{tUnidadDeAprendizaje} para verificar que éstos sean menor que su \refElem{GrupoConUdeAEnOferta.capacidad} con base en la regla de negocio \refIdElem{BR-IN-N027}.\refErr{Nueve} \refErr{Diez}
	
	\UCpaso  \label{IN-UA-CU3.1.2:calcularCreditosSATCA}Calcula la cantidad de créditos SATCA de cada \refElem{tEstudiante} sumando los \refElem{UdeA.creditosSATCA} acumulados por cada \refElem{UdeA.udeA} en las que fue posible agregarlo.
	
	\UCpaso  \label{IN-UA-CU3.1.2:calcularCreditosTEPIC}Calcula la cantidad de créditos TEPIC de cada \refElem{tEstudiante} sumando los \refElem{UdeA.creditosTEPIC} acumulados por cada \refElem{UdeA.udeA} en las que fue posible agregarlo.
	
	% Marcar con trayectorias alternativas los mensajes que llegan a ocurrir.
	% TOCHK El estandar que se esta utilizando en los procesos es que si sólo se muestra un mensaje, se maneje como error.
	\UCpaso  \label{IN-UA-CU3.1.2:tabla}Muestra una tabla con la información obtenida en los pasos \ref{IN-UA-CU3.1.2:obtenerEstudiante}, \ref{IN-UA-CU3.1.2:calcularOcupabilidad}, \ref{IN-UA-CU3.1.2:calcularCreditosSATCA} y \ref{IN-UA-CU3.1.2:calcularCreditosTEPIC}, mostrando la misma de izquierda a derecha:
	\begin{itemize}
		\item Boleta / Preboleta
		\item Nombre
		\item Tipo
		\item No. de Materias
		\item Créditos SATCA
		\item Créditos TEPIC
		\item Mensaje
		\item Acciones
	\end{itemize}
	% V 0.1 Ok.
	\UCpaso Actualiza el número de estudiantes inscritos de cada \refElem{tUnidadDeAprendizaje}, sumándole la cuenta de los lugares ocupados calculados en el paso \ref{IN-UA-CU3.1.2:calcularOcupabilidad}.
	% V 0.1 Ok.
	\UCpaso[\UCactor]  \label{IN-UA-CU3.1.2:inscribir}Presiona el botón \IUbutton{Inscribir}.\refTray{A} \refTray{B} \refTray{C}
	% V 0.1 TODO: Calcula los datos del mensaje primero.
	\UCpaso Calcula el número de \refElem{tEstudiante} que se inscribirán al \refElem{tGrupo} seleccionado en el paso \ref{IN-UA-CU3.1.2:selGrupo}.
	
	\UCpaso Muestra el mensaje \refIdElem{MSG186} solicitando la confirmación de la inscripción.
	% V 0.1 Ok.    
	\UCpaso[\UCactor]  \label{IN-UA-CU3.1.2:confirmar}Presiona el botón \IUbutton{Si}.\refTray{E}
	% V 0.1 TODO: Actualiza el estado, actualiza los lugares disponibles, otorga sobre cupos si es necesario.
	\UCpaso Registra a los estudiantes en las Unidades de Aprendizaje\footnote{Ver \refElem{tUnidadDeAprendizaje}} seleccionadas en el paso \ref{selUA}.
	
	\UCpaso Actualiza el estado de los \refElem{tEstudiante} en las Unidades de Aprendizaje\footnote{Ver \refElem{tUnidadDeAprendizaje}} seleccionadas en el paso \ref{selUA}.
	
	\UCpaso Actualiza la \refElem{GrupoConUdeAEnOferta.ocupacion} de cada \refElem{UdeA.udeA}, otorgando sobrecupos si es necesario en el \refElem{tGrupo} seleccionado en el paso \ref{IN-UA-CU3.1.2:selGrupo}.
\end{UCtrayectoria}

%Trayectoria Alternativas
%----------------- A
%TODO
\begin{UCtrayectoriaA}[Termina el caso de uso]{A}{El actor requiere cancelar la operación.}
\UCpaso Presiona el botón \IUbutton{Cancelar}.
\UCpaso Muestra la pantalla \refIdElem{IN-DAE-UI1}.
\end{UCtrayectoriaA}

