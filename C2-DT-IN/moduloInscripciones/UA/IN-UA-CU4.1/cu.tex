% !TEX root = ../../../integrado.tex

\begin{UseCase}{IN-UA-CU4.1}{Registrar entrega de documentos}{
		En este caso de uso el actor selecciona un alumno con boleta para hacerle entrega de sus documentos y registrar el resultado de dicha entrega. Completando así el proceso de inscripción.
	}
	\UCccitem{Versión}{0.1}
	\UCccsection{Datos para el control Interno}	
	\UCccitem{Elaboró}{Ulises Vélez Saldaña}
	\UCccitem{Supervisó}{}
	\UCccitem{Operación}{}
	\UCccitem{Prioridad}{Baja}
	\UCccitem{Complejidad}{Baja}
	\UCccitem{Volatilidad}{Baja}
	\UCccitem{Madurez}{Media}
	\UCccitem{Estatus}{Revisado}
	\UCccitem{Dificultades}{No.
%		\begin{Titemize}
%			\Titem \TODO
%		\end{Titemize}
	}
	\UCccitem{Fecha del último estatus}{06 de Febrero del 2018}
	\UCccsection{Revisión Versión 0.1}
	\UCccitem{Fecha}{}
	\UCccitem{Evaluador}{}
	\UCccitem{Resultado}{}
	\UCccitem{Observaciones}{}
	\UCsection{Atributos}
	\UCitem{Actores}{%
		\begin{Titemize}
			\Titem \refElem{UAJefeDeControlEscolar}
		\end{Titemize}
	}
	\UCitem{Propósito}{Llevar el control de la entrega de documentos originales a los alumnos de nuevo ingreso.}
	\UCitem{Entradas}{%
		\begin{Titemize}
			\Titem \refElem{EstadoDeDevolucion} de documentos de un Alumno.
			\Titem \refElem{DocumentoDeAsignacion} entregado al Alumno.
		\end{Titemize}
	}
	\UCitem{Origen}{Se introducen desde el teclado}
	\UCitem{Salidas}{
		\begin{Titemize}
			\Titem \refElem{Alumno.nombre} del alumno seleccionado
			\Titem \refElem{Alumno.primerApellido} del alumno seleccionado
			\Titem \refElem{Alumno.segundoApellido} del alumno seleccionado
			\Titem \refElem{Alumno.CURP} del alumno seleccionado
			\Titem \refElem{Alumno.fechaDeNacimiento} del alumno seleccionado
			\Titem \refElem{Alumno.lugarDeNacmiento} del alumno seleccionado
			\Titem \refElem{Alumno.fotografia} del alumno seleccionado
			\Titem \refElem{Domicilio} del alumno seleccionado
			\Titem Información del \refElem{AlumnoAsignado}
			\Titem Detalle de las Unidades de Aprendizaje Inscritas (ver \refElem{UnidadDeAprendizaje}) del alumno seleccionado
			\Titem \refElem{TipoDeDocumento} registrados en el sistema.
		\end{Titemize}
	}
	\UCitem{Destino}{Pantalla}
	\UCitem{Precondiciones}{%
		\begin{Titemize}
			\Titem Que exista al menos un alumno asociado a la Unidad Académica cuyo estado de entrega de documentos sea diferente de ``Satisfactoria''.
			\Titem Que exista al menos un Tipo de Documento registrado.
		\end{Titemize}
	}
	\UCitem{Postcondiciones}{%
		\begin{Titemize}
			\Titem Se cambia el estado de entrega de documentos.
			\Titem Se registran los documentos entregados.
			\Titem Se registra una justificación (en caso de que el estado final sea diferente a ``Satisfactoria''.
		\end{Titemize}
	}
	\UCitem{Reglas de Negocio}{%
		\begin{Titemize}
			\Titem 
		\end{Titemize}
	}
	\UCitem{Errores}{%
		\begin{Titemize}
			\Titem \UCerr{Uno}{Cuando no existe al menos un Alumno con Estado De Devolución distinto a satisfactorio}{se muestra el mensaje \refIdElem{MSGX} y regresa al caso de uso anterior tras confirmar el mensaje por el usuario}
		\end{Titemize}
	}
	\UCitem{Viene de}{\refIdElem{IN-UA-CU4}}
	\UCitem{Disparador}{%
		\begin{Titemize}
			\Titem Se han recibido documentos por parte de la DAE para entrega a los alumnos.
			\Titem Los alumnos se presentan en la ventanilla para recepción de sus documentos.
		\end{Titemize}
	}
	\UCitem{Condiciones de Término}{Se registra el estado de la entrega de documentos.}
	\UCitem{Efectos Colaterales}{Ninguno}
	\UCitem{Referencia Documental}{Proceso fortalecido}
	\UCitem{Auditable}{no}
	\UCitem{Datos sensibles}{Los datos personales de los alumnos, los cuales se deben utilizar solamente para la entrega de documentos.}
\end{UseCase}


%Trayectoria Principal : Happy Path
\begin{UCtrayectoria}
	\UCpaso[\UCactor] Solicita el registro de la entrega de documentos mediante el ícono \IUEstadoEdit{} del alumno en cuestión.
	\UCpaso Obtiene los datos del Alumno seleccionado.
	\UCpaso Obtiene los datos del Domicilio del Alumno seleccionado.
	\UCpaso Obtiene los datos de Asignación (ver \refElem{AlumnoAsignado}, \refElem{detalleAsignacion} y \refElem{DocumentoDeAsignacion}).
	\UCpaso Muestra la información obtenida mediante la pantalla \refElem{IN-DAE-UA4-1}.
	\UCpaso[\UCactor] Selecciona los documentos de los que está haciendo entrega.
	\UCpaso[\UCactor] Selecciona el resultado de la entrega.\refTray{A}
	\UCpaso[\UCactor] Introduce la \refElem{DetalleDeAsignacion.Justificacion}.
	\UCpaso[\UCactor] Presiona el botón ``Registrar entrega''.
	\UCpaso Verifica que los datos obligatorios hayan sido proporcionados correctamente.
	\UCpaso Verifica que haya una justificación de al menos tres palabras.
	\UCpaso \label{paso:CU4-1Registra}Registra el estado de la entrega de la documentación.
	\UCpaso Actualiza la lista de Alumnos de la pantalla de \refElem{IN-UA-CU4}
\end{UCtrayectoria}

%Trayectorias Alternativas

\begin{UCtrayectoriaA}[Continua en el paso \ref{paso:CU4-1Registra}]{A}{Se realiza una entrega con observaciones.}
	\UCpaso[\UCactor] Selecciona la opción de ``Entregados con observaciones''
	\UCpaso Activa la edición del campo titulado ``Observaciones''.
\end{UCtrayectoriaA}

%\subsection{Puntos de extensión}
%\UCExtensionPoint{Registrar periodo de E.T.S.}
%{El actor desea registrar un nuevo periodo de E.T.S.}
%{En el paso \ref{IN-DAE-CU4:gestionar} de la trayectoria principal}
%{\refIdElem{IN-DAE-CU4.1}}
%
%\UCExtensionPoint{Editar periodo de E.T.S.}
%{El actor desea editar un periodo de E.T.S.}
%{En el paso \ref{IN-DAE-CU4:gestionar} de la trayectoria principal}
%{\refIdElem{IN-DAE-CU4.2}}
%
%\UCExtensionPoint{Eliminar periodo de E.T.S.}
%{El actor desea eliminar un periodo de E.T.S.}
%{En el paso \ref{IN-DAE-CU4:gestionar} de la trayectoria principal}
%{\refIdElem{IN-DAE-CU4.3}}
