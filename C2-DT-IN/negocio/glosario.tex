% !TEX root = ../integrado.tex

	El presente glosario presenta los términos utilizados a lo largo del documento. Esta unificación de criterios está basada en la normatividad y cultura del IPN y tiene como finalidad establecer el lenguaje base que permita comprender la especificación del sistema y se utilizará en la construcción del mismo. 
	{\color{red}\bf Advertencia:} Este glosario no debe interpretarse como ``el término X es Y'' si no mas bien, su significado es ``el término X, en el CALMÉCAC y en el presente módulo, se refiere a Y''. La lista de términos se encuentra agrupada por áreas de conocimiento:

	\begin{Citemize}
		\item Términos técnicos: Agrupa los términos que tienen que ver con el sistema.
		\item Términos del negocio: Agrupa los términos que tienen significado dentro del IPN.
	\end{Citemize}

%\section{Glosario de términos}

%====================================================================
\section{Términos técnicos}
\label{gls:terminosTecnicos}

En esta sección se definen los términos técnicos que se utilizan para describir el comportamiento del sistema.

\begin{description}
	
	\bTerm{tAlfanumerico}{Alfanumérico} Es un \refElem{tTipoDato} definido por el conjunto de caracteres numéricos y alfabéticos.
	
	\bTerm{tArchivoDigital}{Archivo digital} Equivalente digital de los archivos escritos en libros, tarjetas, libretas, papel o microfichas del entorno de oficina tradicional.
	
	\bTerm{tAtributo}{Atributo} Son las características que definen o identifican a una entidad en un conjunto de entidades.
	
	\bTerm{tBooleano}{Booleano} Es un \refElem{tTipoDato} que puede tomar los siguientes valores: verdadero ó falso (1 ó 0).
	
	\bTerm{tCadena}{Cadena} Es el \refElem{tTipoDato} definido por cualquier valor que se compone de una secuencia de caracteres, con o sin acentos, espacios, dígitos y signos de puntuación. Existen tres tipos de cadenas: palabra, frase y párrafo.
	
	\bTerm{Calmecac}{CALMÉCAC} Se refiere al nuevo sistema de control escolar del Instituto Politécnico Nacional.
	\bTerm{tCatalogo}{Catálogo} Es una lista ordenada o clasificada de elementos relacionados.
	
	\bTerm{tDecimal}{Decimal} Es un \refElem{tTipoDato} \refElem{tNumerico}. Los números decimales son valores que denotan números racionales y la aproximación a números irracionales.
	
	\bTerm{tEntero}{Entero} Es el \refElem{tTipoDato} \refElem{tNumerico} definido por todos los valores numéricos enteros, tanto positivos como negativos.
	
	\bTerm{tEntidad}{Entidad} Término genérico que se utiliza para determinar un ente el cual puede ser concreto, abstracto o conceptual por ejemplo: Unidad administrativa, entregable, persona, etc. La entidades se caracterizan a través de atributos que personalizan a la entidad.	
		
	%Se usa para hacer referencia a un objeto con existencia física (entidad concreta) como: Una persona, un animal, una casa, etc.; o un objeto con existencia conceptual (entidad abstracta) como: Un puesto de trabajo, una asignatura de clases, un nombre, etc. Una \refElem{gls:entidad}{entidad} se representa por sus características o atributos, por ejemplo: La entidad persona tiene características como: Nombre, apellido, género, estatura, peso, fecha de nacimiento, etc.
	
	\bTerm{tFecha}{Fecha} Es un \refElem{tTipoDato} que indica un día único en referencia al calendario gregoriano. Los tipos de fecha utilizados son: \refElem{tFechaCorta} y \refElem{tFechaLarga}. %con formato DD/MM/YYYY, por ejemplo: 24/02/2013.
	
	\bTerm{tFechaCorta}{Fecha corta} Es la representación del \refElem{tTipoDato} \refElem{tFecha} en la forma DD/MM/YYYY, por ejemplo: 24/02/2013.
	
	\bTerm{tFechaLarga}{Fecha larga} Es la representación del \refElem{tTipoDato} \refElem{tFecha} en la forma DD de MM del YYYY, por ejemplo: 24 de febrero del 2013.
	
	\bTerm{tFrase}{Frase} Es un \refElem{tTipoDato} conformado por \refElem{tPalabra} y espacios.
	
	\bTerm{tNumerico}{Numérico} Es un \refElem{tTipoDato} que se compone de la combinación de los símbolos \textit{0,1,2,3,4,5,6,7,8,9,. y -.}  que expresan una cantidad en relación a su unidad.
	
	\bTerm{tOpcional}{Opcional} Es un elemento que el actor puede o no proporcionar en el formulario o la pantalla, su decisión no afectará la ejecución de la operación solicitada.
	
	\bTerm{tPalabra}{Palabra} Es un \refElem{tTipoDato} \refElem{tCadena} conformado por el alfabeto y símbolos especiales como son \textit{\#,-,\$,\%,\&,(,),etc} y se caracteriza por no tener espacios.
	
	\bTerm{tParrafo}{Párrafo} Es un \refElem{tTipoDato} conformado por \refElem{tFrase}{frases}.
	%\BRterm{gls:sn}{S/N} Abreviación del término ``Sin número'' utilizado para indicar cuando una Dirección Geográfica no tiene numeración.
	
	\bTerm{tRequerido}{Requerido} Es un \refElem{tTipoDato} que debe proporcionarse de manera obligatoria. La ejecución de la operación solicitada dependerá de que se proporcione este dato.
	
	%Es un atributo de una \refElem{gls:entidad}{entidad} que por definición no puede quedar indeterminado. Lo cual implica para el sistema, que, si se solicita mediante una pantalla, base de datos o servicio externo, el dato debe proporcionarse de manera obligatoria para el registro adecuado en el sistema.
	
	\bTerm{tTipoDato}{Tipo de dato} Es el dominio o conjunto de valores que puede tomar un atributo de una \refElem{tEntidad} en el modelo de información. Los tipos de datos utilizados son: \refElem{tPalabra}, \refElem{tFrase}, \refElem{tParrafo}, \refElem{tNumerico}, \refElem{tFecha} y \refElem{tBooleano}.
	
	%\BRterm{gls:na}{NA} Abreviación del término ``No Aplica'', se utiliza para indicar que algún elemento en la estructura del documento o en el sistema no aplica.
\end{description}

%====================================================================
\section{Términos del negocio}

En esta sección se definen los términos del negocio que se utilizan para comprender el comportamiento del sistema.

\begin{bGlosario}	
	
	%------------------------------------------------------------
	\bTerm{tAcademia}{Academia} Al órgano constituido por profesores que tiene la finalidad de proponer, analizar, opinar, estructurar y evaluar el proceso educativo.
	
	%------------------------------------------------------------
	\bTerm{tAlumno}{Alumno} A la persona inscrita en algún programa académico que se imparta en cualquier nivel educativo y modalidad educativa que ofrece el Instituto Politécnico Nacional. %% RGE
	%------------------------------------------------------------
	\bTerm{tAreaDeConocimiento}{Área de conocimiento} El Instituto ofrece las siguientes áreas de conocimiento:
	\begin{Titemize}
		\Titem Ingeniería y Ciencias Físico Matemáticas
		\Titem Ciencias Sociales y Administrativas
		\Titem Ciencias Médico Biológicas
	\end{Titemize}

	%------------------------------------------------------------
	\bTerm{tAspirantesNoInscritos}{Aspirantes no inscritos}	Aspirantes que serán asignados a un grupo dentro de la Unidad Académica en la que se encuentran asociados y se les asignará un número de boleta con el cual serán identificados como alumnos dentro del Instituto.

  	%-----------------------------------------------------------
 	\bTerm{tCalendarioAcademico}{Calendario Académico} Programación que define los tiempos en los cuales se realizan anualmente las actividades académicas y de gestión escolar, en las diversas modalidades educativas que se imparte en el Instituto.
 	%-----------------------------------------------------------
	 \bTerm{tCredito}{Crédito} A la unidad de reconocimiento académico que mide y cuantifica las actividades de aprendizaje contempladas en un plan de estudio; es universal, transferible entre programas académicos y equivalente al trabajo académico del alumno.
	 %------------------------------------------------------------	
	 \bTerm{tCicloEscolar}{Ciclo Escolar} El lapso anual que define el \refElem{tCalendarioAcademico}

	 %------------------------------------------------------------	
	 \bTerm{tCategoria}{Categoría} La categoría del personal académico  \refElem{tPromocion}
	 Clasificación que tiene el personal académico en el Instituto Politécnico Nacional.
	 
	 %-----------------------------------------------------------
	 \bTerm{tCategoriaPA}{Categoría del personal académico} El personal académico se divide en las siguientes categorías:\\
	 \begin{Titemize}
	 	\Titem [\refElem{tProfesor}]
	 	\Titem [\refElem{tTecnicoDocente}]
	 	%\Titem [PAE:] Preguntar por él 
	 \end{Titemize}
	 
	 	%------------------------------------------------------------	
	 \bTerm{tCategoriaProfC}{Categoría del profesor por carrera} El \refElem{tProfesor} por su categoría en medio superior y superior podrá ser:\\
	 \begin{Titemize}
	 	\Titem [Asistente]
	 	\Titem [Asociado]
	 	\Titem [Titular]
	 \end{Titemize}
	 
	 %------------------------------------------------------------	
	 \bTerm{tCategoriaTecnicoC}{Categoría del técnico docente por carrera} El \refElem{tTecnicoDocente} por su categoría en medio superior y superior podrá ser:\\
	 
	 \begin{Titemize}
	 	\Titem [Superior]:\\
	 	\begin{Titemize}
	 		\Titem Asistente
	 		\Titem Asociado
	 		\Titem Titular
	 	\end{Titemize}	
	 	
	 	\Titem [Medio superior:]\\	
	 	\begin{Titemize}
	 		\Titem Técnico auxiliar
	 		\Titem Técnico asociado
	 		\Titem Técnico titular
	 	\end{Titemize}
	 \end{Titemize}
	 
	 	%------------------------------------------------------------	
	 \bTerm{tCarga}{Carga} Los grupos asignados al \refElem{tPersonalAcademico} para impartir clases.
	 
	 %------------------------------------------------------------	
	 \bTerm{tCargaReglamentaria}{Carga reglamentaria} A la carga que tiene asignada el personal académico, el cual  dependerá de la categoría y nivel que tenga. La carga reglamentaria del personal académico se divide en: \\
	 	\begin{Titemize}
	 		\Titem [Carga máxima]
	 		\Titem [Carga mínima]
	 		\Titem [Carga media]
	 	\end{Titemize}
	 %------------------------------------------------------------	
	 \bTerm{tCicloEscolar}{Ciclo Escolar} El lapso anual que define el \refElem{tCalendarioAcademico}
	  %------------------------------------------------------------	
	 \bTerm{tDAE}{DAE} Dirección de Administración Escolar
	  %------------------------------------------------------------	
	 \bTerm{tDES}{DES} Dirección de Educación Superior
	 %------------------------------------------------------------	
	 \bTerm{tUA}{UA} Unidad Académica
	 %------------------------------------------------------------	
	 \bTerm{tEstructuraEducativa}{Estructura Educativa} Al servicio educativo de acuerdo con los planes y programas académicos vigentes aplicables y con base en los recursos autorizados.
 
	%------------------------------------------------------------
	\bTerm{tEdificio}{Edificio} Es una construcción fija destinada para la escuela y que permite la realización de distintas actividades.
	
	%------------------------------------------------------------	
	\bTerm{tEspacio}{Espacio} Lugar en el cual es impartida una \refElem{tUnidadDeAprendizaje}
	
	%------------------------------------------------------------	
	\bTerm{tEstudiante}{Estudiante} Persona que ha obtenido un resultado aprobatorio en el exámen de admisión que será inscrita y/o asociada a la \ref{tUnidadAcademica} a la que fue asignada.

	%------------------------------------------------------------	
 	\bTerm{tDescargaAcadémica}{Descarga académica} Descarga académica o actividades complementarias, comprenden la revisión, actualización y elaboración de planes y programas de estudios, apuntes, notas o textos, asesorías; revisión de tesis, revisión de prácticas profesionales; coordinación de actividades de servicio social; asistencia a reuniones de academia y de departamentos, a exámenes, impartición de cursos, seminarios, conferencias y foros académicos; supervisión a la enseñanza y otros similares; así como actividades de apoyo al personal académico y de investigación en la operación y manejo de equipos y materiales didácticos y en general a todas aquellas que contribuyen al mejoramiento
 	de la enseñanza.
 	
 	%------------------------------------------------------------	
 	\bTerm{tDictamenCategoria}{Dictamen de categoría} Es la \refElem{tCategoria} y \refElem{tTipoNivel} del profesor que obtiene al presentar una \refElem{tPromocion}.
 	
 	%------------------------------------------------------------	
 	\bTerm{tFormato}{Formato} Relación de profesores que agrupa características, estados y/o condiciones que se emplean en la elaboración de la estructura educativa.

	%------------------------------------------------------------
	\bTerm{tGrupo}{Grupo} Conjunto de alumnos que están juntos en un \refElem{tTipoEspacio} en el cual se imparte una \refElem{tUnidadDeAprendizaje}.
	
	%------------------------------------------------------------
	\bTerm{tHorario}{Horario} Distribución de las horas en que se realiza una actividad o trabajo.
	
	%------------------------------------------------------------	
	\bTerm{tHorasInterinato}{Horas de interinato} Horas asignada al personal acadeémico interino\\
		\begin{Titemize}
			\Titem [Incidencia]
			\Titem [Frente a grupo]
		\end{Titemize}
	
	%------------------------------------------------------------	
	\bTerm{tHorasBasificadas}{Horas basificadas} Horas en propiedad que tiene el personal académico de base
	
		%------------------------------------------------------------	
	\bTerm{tHorasCompactadas}{Horas compactadas} Horas asignadas a la plaza de un personal académico.
	
		%------------------------------------------------------------	
	\bTerm{tHorasNoCompactadas}{Horas no compactadas} Horas asignadas a un personal académico en distintas plazas.
		
	%------------------------------------------------------------
	\bTerm{tModalidad}{Modalidad Educativa} Forma en que se organizan, distribuyen y desarrollan los planes y programas de estudio para su impartición. Existen 3 tipos de modalidades:\\
	
	\begin{Titemize}
		\Titem [Escolarizada:] La que se desarrolla en aulas, talleres, laboratorios y otros ambientes de aprendizaje, en horarios y periodos determinados.
		\Titem [No Escolarizada:] Es la que se desarrolla fuera de aulas,
		talleres, laboratorios y no necesariemente comprende horarios determinados.
		\Titem [Mixta:] Es la combinación de modalidades educativas de
		acuerdo con el diseño un programa académico en particular.
	\end{Titemize}

	%------------------------------------------------------------	
	\bTerm{tNivelEdificio}{Nivel de edificio} Se considera a cada una de las alturas en que se divide el edificio ya sea por encima o por debajo del nivel del suelo, teniendo que estar cubiertas por un techo.
	
		%------------------------------------------------------------	
	\bTerm{tNivelEducativo}{Nivel educativo} Cada una de las etapas en las que se estructuran los estudios que ofrece el Instituto:
	\begin{itemize}
	\item Medio Superior (Bachillerato Bivalente)
	\item Superior (Licenciatura)
	\item Posgrado (Especialidad, Materia y Doctorado)
	\end{itemize}
	%------------------------------------------------------------	
	\bTerm{tOcupabilidad}{Ocupabilidad}	A la disponibilidad de lugares de una unidad de aprendizaje en un \refElem{tGrupo}

	%------------------------------------------------------------
	\bTerm{tOfertaEducativa}{Oferta Educativa} Unidades de apredizaje que se impartirán de un programa académico el \refElem{tPESigInmediato}.

 	%------------------------------------------------------------
	\bTerm{tPeriodo}{Periodo} Se refiere a un lapso de tiempo el cual está marcado por una fecha de inicio y una fecha final. 	
	
	%------------------------------------------------------------
	\bTerm{tPeriodoDeTrabajo}{Periodo de Trabajo} Lapso de tiempo en que una unidad de aprendizaje debe impartirse puede ser:
	\begin{itemize}
		\item Mensual.
		\item Bimestral.
		\item Trimestral.
		\item Cuatrimestral.
		\item Semestral.
		\item Anual.
	\end{itemize}

 	%------------------------------------------------------------
 	\bTerm{tPeriodoEscolar}{Periodo Escolar} Se refiere al \refElem{tPeriodo} que rige la ejecución de todas las actividades relevantes en la gestión escolar. En la modalidad presencial tiene una duración aproximada de seis meses y en modalidad mixta seis semanas.
 	
	%------------------------------------------------------------
 	\bTerm{tPEAntInmediato}{Periodo escolar siguiente inmediato} Se refiere al \refElem{tPeriodoEscolar} que va a comenzar concluyendo el periodo escolar actual
 	
	%------------------------------------------------------------
 	\bTerm{tPESigInmediato}{Periodo escolar anterior inmediato} Se refiere al \refElem{tPeriodoEscolar} que conluyó antes del periodo escolar actual.

	%------------------------------------------------------------
	\bTerm{tPEsiguiente}{Periodo escolar siguiente} Se refiere a los periodos escolares\footnote{Ver \refElem{tPeriodoEscolar}} que van a seguir concluyendo el periodo escolar actual
	%------------------------------------------------------------
	\bTerm{tPEanterior}{Periodo escolar anterior} Se refiere a los periodos escolares\footnote{Ver \refElem{tPeriodoEscolar}} que concluyeron antes del periodo escolar actual.
	
	%------------------------------------------------------------	
	\bTerm{tPlaza}{Plaza} Se conforma de las \refElem{tHorasBasificadas} que tiene el personal académico (\refElem{tProfesor} y \refElem{tTecnicoDocente}) pueden ser de tiempo completo, tres y cuarto de tiempo o medio tiempo, y de uno o varios dictámines de categoría \footnote{ \refElem{tDictamenCategoria}}.
	%------------------------------------------------------------
	\bTerm{tPlanEstudio}{Plan de Estudio} Estructura curricular que se deriva de un programa académico y que permite cumplir con los propósitos de formación general, la adquisición de conocimientos y el desarrollo de capacidades correspondientes a un nivel y modalidad educativa.%Reglamento General de Estudios 	
	
	%------------------------------------------------------------
	\bTerm{tProgramaAcademico}{Programa Académico} Conjunto organizado de elementos necesarios para generar, adquirir y aplicar	el conocimiento en un campo específico; así como para desarrollar habilidades, actitudes y valores en el alumno, en diferentes áreas del conocimiento. %Reglamento General de Estudios
	
 	%-----------------------------------------------------------
	\bTerm{tProfesor}{Profesor} Responsable de las funciones específicas de docencia, de investigación científica y tecnológica, desarrollo tecnológico e investigación educativa y demás asociadas, complementarias a las anteriores.
	
	%------------------------------------------------------------	
	\bTerm{tProfeInterino}{Profesor Interino} Horas asignada al personal acadeémico interino\\
	\begin{Titemize}
		\Titem [Frente a grupo:] Es aquel que cubre una vacante definitiva o una plaza de nueva
		creación y adquiere el carácter de inamovible.
		\Titem [Incidencia:] Es aquel que cubre licencia temporal del personal académico de
		base, hasta por seis meses.
	\end{Titemize}
	
	%Persona que conduce el proceso de enseñanza-aprendizaje enfocado a la transmisión de conocimientos y a la formación integral del alumno. Será la autoridad académica del grupo a su cargo y desempeñará sus actividades conforme al principio de libertad de cátedra e investigación, atendiendo los programas aprobados por las autoridades académico-administrativas del IPN y del centro de trabajo correspondiente.
	
	
	%------------------------------------------------------------	
	\bTerm{tPersonalAcademico}{Personal Académico} A la persona que presta sus servicios al Instituto Politécnico Nacional, desempeñando trabajos académicos en los términos del presente reglamento.
	
	%------------------------------------------------------------	
	\bTerm{tPersonalAcademicoCarrera}{Personal Académico por Carrera} Es quien tiene la responsabilidad de todas las actividades que lleven a la realización y superación
	académica e investigación, dentro del I.P.N., y podrá ser:\\
		\begin{Titemize}
			\Titem [De tiempo completo:] Con 40 horas de trabajo por semana.
			\Titem [De tres cuartos de tiempo:] Con 30 horas de trabajo por semana.
			\Titem [De medio tiempo] Con 20 horas de trabajo por semana.
		\end{Titemize}
	
		%------------------------------------------------------------	
	\bTerm{tPersonalAcademicoAsignatura}{Personal Académico por asignatura} Es aquel cuyo nombramiento podrá ser hasta de 19 horas y deberá desempeñar su actividad dentro de su centro de trabajo en preparación e impartición de cátedra.
	%------------------------------------------------------------	
	\bTerm{tPromocion}{Promoción} Promoción del personal académico o proceso de categorización, es el cambio de una categoría a la inmediata superior dentro del tabulador vigente, al cumplir el personal con los requisitos académicos y profesionales que señale el presente reglamento.
	
	%------------------------------------------------------------	
	\bTerm{tProcBasificacion}{Proceso de basificación} Es la asignación de horas en propiedad que tiene el \refElem{tProfesor}, al cumplir con los requisitos académicos y profesionales que pidan en la convocatoria.
	
	%------------------------------------------------------------	
	\bTerm{tTecnicoDocente}{Técnico docente} Responsable de las funciones técnicas y profesionales propias de su especialidad, que se requieren como complemento de las funciones de docencia e investigación, y de impartición de cátedra o de apoyo a la docencia e investigación en actividades de talleres y laboratorios.
		
 	%--------------------- --------------------------------------
%	\bTerm{tTipoPersonalAcademico}{Tipos de personal académico}	El personal académico  por su tipo podrá ser:\\
%	
%	\begin{Titemize}
%		\Titem [Personal académico definitivo:] Es aquel que cubre una vacante definitiva o una plaza de nueva creación y adquiere el carácter de inamovible.
%		\Titem [Personal acadpemico provisional:] Es aquel que ocupa una vacante temporal, mayor de seis meses, originada por licencia sin goce de sueldo.
%		\Titem [Personal académico interino:] Es aquel que cubre licencia temporal del personal académico de base, hasta por seis meses.
%		\Titem [Personal académico de tiempo fijo] Es aquel que presta sus servicios con una fecha cierta para su terminación.
%		\Titem [Personal académico por obra determinada:] Es aquel que presta sus servicios para realizar una obra concreta y perfectamente definida.
%	\end{Titemize}
%

	
	%------------------------------------------------------------
	\bTerm{tTipoEspacio}{Tipo de Espacio} Indica en que lugar la \refElem{tUnidadDeAprendizaje} se imparte. Puede ser:
	\begin{itemize}
		\item Aula.
		\item Taller.
		\item Laboratorio.
		\item Otros ambientes de aprendizaje.
	\end{itemize}

	%------------------------------------------------------------	
	\bTerm{tTipoPermanecia}{Tipo de permanecia} El personal académico por cuanto a su permanencia y condiciones podrá ser:\\
	
	\begin{Titemize}
		\Titem [\refElem{tPersonalcoAsignatura}]
		\Titem [\refElem{tPersonalAcademicoCarrera}]
	\end{Titemize}

		
	%------------------------------------------------------------	
	\bTerm{tTipoNivel}{Tipo de nivel} El personal académico por su nivel podrá ser:\\
	\begin{Titemize}
		\Titem [A]
		\Titem [B]
		\Titem [C] 
	\end{Titemize}


	
	%------------------------------------------------------------
	\bTerm{tTurno}{Turno} Espacio de tiempo que permite a la unidad académica organizar los Grupos\footnote{\refElem{tGrupo}} por lo general el turno se divide en tres turno:\\
	\begin{Titemize}
		\Titem [Matutino]
		\Titem [Vespertino]
		\Titem [Mixto]
	\end{Titemize}

	%------------------------------------------------------------
	\bTerm{tUnidadAcademica}{Unidad Académica} Espacio geográfico donde se ubican las escuelas, centros y unidades en los que se realizan actividades de docencia, investigación y difusión de la cultura, en los niveles superior y de posgrado.

	%------------------------------------------------------------
	\bTerm{tUnidadDeAprendizaje}{Unidad de Aprendizaje} Estructura didáctica que integra los contenidos formativos de un curso, materia, módulo, asignatura o sus equivalentes. En general, las unidades de aprendizaje deberán cursarse y acreditarse conforme lo establezca el plan de estudio, y podrán seleccionarse de entre la oferta disponible en el periodo escolar y sujeta a grupo. %Reglamento General de Estudios
	
	%------------------------------------------------------------
	\bTerm{tUnidadDeAprendizajeModular}{Unidad de Aprendizaje Modular} Es aquella unidad de aprendizaje que se compone de las unidades temáticas de una \refElem{tUnidadDeAprendizaje}.

	%------------------------------------------------------------
	\bTerm{tSalon}{Salón} Un aula es un compartimento o salón de un edificio que se destina a actividades de enseñanza, y es la unidad básica de todo recinto destinado a la educación.	
		
	%------------------------------------------------------------	
	\bTerm{tSoporteDocumental}{Soporte Documental} Documentación presentada por la unidad académica para justificar los casos de docentes que no cumplan su carga máxima reglamentaria
\end{bGlosario}
