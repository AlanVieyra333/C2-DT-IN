\begin{TipoDeDato}{tdEntidad}{Entidad}{Es un catálogo que contiene a cada uno de los 32 estados miembros del Estado federal(México) con el propósito de especificar:
	\begin{Citemize}
		\item La entidad federativa en donde un alumno nació.
		\item La entidad federativa en la que reside un alumno.
		\item La entidad federativa en la que una Unidad Académica reside.
	\end{Citemize}}
	
	\begin{tdAtributos}
		\tdAttr{nombre}{Nombre}{tdfrase}{Es la palabra o conjunto de palabras que tienen como propósito identificar a una entidad federativa que se encuentra dentro de los límites de la República Mexicana.}
	\end{tdAtributos}
	
	\subsection{Valores Iniciales}
	
	La población de este catálogo es con base en el \href{}{Catálogo de Códigos Postales de S.E.P.O.M.E.X.}.
\end{TipoDeDato}

\begin{TipoDeDato}{tdMunicipio}{Municipio}{Es un catálogo que contiene las distintas divisiones políticas y territoriales que se encuentran dentro de una entidad federativa con el propósito de especificar los domicilios de alumnos.}
	\begin{tdAtributos}
		\tdAttr{nombre}{Nombre}{tdfrase}{Es la palabra o conjunto de palabras que tienen como propósito identificar a un municipio o delegación(en el caso de la Ciudad de México).}
		\tdAttr{entidad}{Entidad}{tdEntidad}{Indica la entidad federativa a la que pertenece un Municipio.}
	\end{tdAtributos}
	
	\subsection{Valores Iniciales}
	
	La población de este catálogo es con base en el \href{}{Catálogo de Códigos Postales de S.E.P.O.M.E.X.}.
\end{TipoDeDato}
