\begin{TipoDeDato}{tdEstadoDeEstructuraEducativa}{Estado de Estructura Educativa}{Es un catálogo en el que se almacenan cada una de las posibles condiciones en las que la Estructura Educativa de una Unidad Acad\'emica, o sus elementos conocidos como segmentos se encuentran, definiendo así el comportamiento y funcionalidades posibles.}
	
	\begin{tdAtributos}
		\tdAttr{nombre}{Nombre}{tdfrase}{Es la palabra o el conjunto de palabras que identifica a una condición de la Estructura Educativa o de alguno de sus segmentos.}
	\end{tdAtributos}
	
	\subsection{Valores Iniciales}
	
	 A continuación se presenta una tabla que contiene los nombres de los posibles estados en los que una una Estructura Educativa se puede encontrar en el sistema, para conocer la descripción correspondiente ver la máquina de estados \refElem{sec:SM-EE}.\cdtEmpty
	\begin{longtable}{| p{0.5\textwidth}|}
	 			\rowcolor{colorPrincipal}
	 			\bf \color{white} Nombre del Estado\\
	 			\hline
	 				Creada \\
	 				\hline
	 				Revisión\\
	 				\hline
	 				Edición \\
	 				\hline
	 				Aprobada\\
	 				\hline
	 				Cerrada\\
	 			\hline
	 \end{longtable}
	
\end{TipoDeDato}

\begin{TipoDeDato}{tdActividadComplementaria}{Actividad Complementaria}{Es un catálogo en el que se almacenan todas aquellas actividades que un Profesor del Instituto puede realizar con el fin de cubrir ciertas horas de acuerdo a su categoría.}
	
	\begin{tdAtributos}
			\tdAttr{actividad}{Actividad}{tdfrase}{Es el conjunto de palabras que identifican a una de las actividades que un profesor puede realizar durante un período de acuerdo a su categoría y si es que tiene horas frente a grupo que no se encuentren asignadas.}
			
			\tdAttr{clave}{Clave}{tdentero}{Es un número que tiene como propósito representar a una actividad complementaria.}
	\end{tdAtributos}
	
	\subsection{Valores Iniciales}
	
	En la siguiente tabla se muestran aquellos que fueron obtenidos del \textbf{Artículo 14avo.} del \textbf{Reglamento de las condiciones interiores de trabajo del Personal Académico del Instituto Politécnico Nacional} y de acuerdo a la minuta \textbf{20171012-13 M EE Evaluacion de propuesta DES-DEMS}. 
	
	\begin{longtable}{|p{0.45\textwidth}|p{0.3\textwidth}|}
	\rowcolor{colorPrincipal}
	\multicolumn{2}{|c|}{\bf \color{white} Valores Iniciales}\\
	\hline
	\rowcolor{colorSecundario}
	\bf \color{white} Actividad & \bf\color{white} Clave \\
	\hline
	Revisión, Actualización y Elaboración de Planes y Programas de Estudio & \TODO Se requiere a DGyCE\\
	\hline
	Elaboración de Apuntes, Notas o Textos & \\
	\hline
	Asesorías & \\
	\hline
	Revisión de Tesis & \\
	\hline
	Revisión de Prácticas Profesionales & \\
	\hline
	Coordinación de Actividades de Servicio Social & \\
	\hline
	Asistencia a Reuniones de Academia y de Departamentos & \\
	\hline
	Impartición de Cursos,  Seminarios, Conferencias y Foros Académicos & \\
	\hline
	Supervisión a la Enseñanza & \\
	\hline
	\end{longtable}
\end{TipoDeDato}

\begin{TipoDeDato}{tdTipoDeAsignacion}{Tipo de Asignación}{Es un catálogo en el que se almacenan la clasificación de los distintos grados de responsabilidad que tiene un profesor con una Unidad de Aprendizaje en un  grupo.}
	
	\begin{tdAtributos}
		\tdAttr{nombre}{Nombre}{tdfrase}{Es la palabra o conjunto de palabras que identifican el grado de responsabilidad de un profesor con una Unidad de Aprendizaje en un grupo.}
		
		\tdAttr{descripcion}{Descripción}{tdfrase}{Es el conjunto de palabras que tienen como propósito exponer las características que definen el grado de responsabilidad de un profesor con una Unidad de Aprendizaje en un grupo.}
	
	\end{tdAtributos}

	\subsection{Valores Iniciales}
	
	En la siguiente tabla se muestran aquellos valores con los cuales el catálogo será poblado inicialmente de acuerdo a la minuta \textbf{20171012-13 M EE Evaluacion de propuesta DES-DEMS}.\cdtEmpty
	
		\begin{longtable}{| p{0.15\textwidth} | p{0.4\textwidth} |}
	 			\rowcolor{colorPrincipal}
	 			\multicolumn{2}{|c|}{\bf \color{white} Valores Iniciales}\\
	 			\hline
	 			\rowcolor{colorSecundario}
	 			\bf \color{white} Nombre & \bf \color{white}Descripción \\
	 			\hline
	 			Titular & Es el grado de responsabilidad más alto de un profesor con un grupo. El profesor define las actividades que se llevarán a cabo y cómo es que se realizarán. \\
	 			\hline
	 			Adjunto &  Es el grado que establece que un profesor apoyará en lo que se requiera al Titular en la exposición de su clase. Este papel se encuentra sólo establecido para el nivel medio superior.\\
	 			\hline
	 		\end{longtable}

\end{TipoDeDato}

\begin{TipoDeDato}{tdTipoDeExposicion}{Tipo de Exposición}{Es un catálogo en el que se almacenan la clasificación de las distintas formas en las que se imparte una clase de acuerdo a la definición del contenido de una Unidad de Aprendizaje. Se utiliza para hacer la separación correspondiente entre horas teóricas y horas prácticas que deben cubrirse.}
	
	\begin{tdAtributos}
		\tdAttr{nombre}{Nombre}{tdfrase}{Es la palabra o el conjunto de palabras que identifican a una de las formas en que los contenidos de una Unidad de Aprendizaje se expondrán.}
		\tdAttr{descripcion}{Descripción}{tdfrase}{Es el conjunto de palabras que tienen como propósito exponer las características que hacen que los contenidos de una Unidad de Aprendizaje se expongan de cierta forma.}
	\end{tdAtributos}
	
	\subsection{Valores Iniciales}

	 En la siguiente tabla se muestran aquellos valores con los cuales el catálogo será poblado inicialmente de acuerdo a la minuta \textbf{20171012-13 M EE Evaluacion de propuesta DES-DEMS}.\cdtEmpty
		
		\begin{longtable}{| p{0.25\textwidth} | p{0.4\textwidth} |}
	 			\rowcolor{colorPrincipal}
	 			\multicolumn{2}{|c|}{\bf \color{white} Valores Iniciales}\\
	 			\hline
	 			\rowcolor{colorSecundario}
	 			\bf \color{white} Nombre & \bf \color{white}Descripción \\
	 			\hline
	 			Teórica & Indica que el contenido de la unidad de aprendizaje será expuesto en un aula en el que la práctica no se pueda realizar. \\
	 			\hline
	 			Práctica &  Indica que el contenido de la unidad de aprendizaje será expuesto por medio de que el alumno realice actividades que fortalezcan su entendimiento.\\
	 			\hline
	 			Otros Ambientes &  Indica que el contenido de la unidad de aprendizaje requiere ser adquirido por otro medio. Este tipo de exposición sólo se se encuentra disponible para Unidades de Aprendizaje de nivel Medio Superior. \\
	 			\hline
	 		\end{longtable}
	\end{TipoDeDato}
