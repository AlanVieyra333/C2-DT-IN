%%======================================================================
\begin{BusinessRule}{BR-S001}{Campos obligatorios}
	{\bcIntegridad}    % Clase: \bcCondition,   \bcIntegridad, \bcAutorization, \bcDerivation.
	{\btEnabler}     % Tipo:  \btEnabler,     \btTimer,      \btExecutive.
	{\blControlling}    % Nivel: \blControlling, \blInfluencing.
	\BRItem[Versión] 1.1.
	\BRItem[Estado] Propuesta.
	\BRItem[Propuesta por] Ángeles
	\BRItem[Revisada por] Pendiente.
	\BRItem[Aprobada por] Pendiente.
	\BRItem[Descripción] Los campos proporcionados al sistema marcados como obligatorios no se deben omitir.
	\BRItem[Sentencia] Sea $campo$ un atributo de $Entidad$, tal que $campo.obligatorio = true$ entonces
	$ \forall campo \Rightarrow campo.valor \neq \emptyset $
	
	\BRItem[Motivación] Evitar la falta de información relevante en el sistema causada por la omisión del actor al introducir datos.
	\BRItem[Ejemplo positivo] Cumplen la regla:
	\begin{itemize}
		\item Para la entidad unidad aprendizaje, el actor introduce todos los atributos solicitados.
		\item Para el Perfil Docente, el actor introduce todos los datos con excepción de habilidades y actitudes.
		\item Para la entidad bibliografía, el actor introduce todos los atributos solicitados con excepción del ISSN.
	\end{itemize}
	\BRItem[Ejemplo negativo] No cumplen con la regla:
	\begin{itemize}
		\item El actor no proporciona el nombre para la entidad
		\item El actor no proporciona la fecha de validación para la entidad Plan de Estudio
		\item El actor no proporciona el lugar de realización para la entidad Practica
	\end{itemize}
\end{BusinessRule}


%%======================================================================
\begin{BusinessRule}{BR-S002}{Información correcta}
	{\bcIntegridad}    % Clase: \bcCondition,   \bcIntegridad, \bcAutorization, \bcDerivation.
	{\btEnabler}     % Tipo:  \btEnabler,     \btTimer,      \btExecutive.
	{\blControlling}    % Nivel: \blControlling, \blInfluencing.
	\BRItem[Versión] 1.0.
	\BRItem[Estado] Propuesta.
	\BRItem[Propuesta por] Ángeles
	\BRItem[Revisada por] Pendiente.
	\BRItem[Aprobada por] Pendiente.
	\BRItem[Descripción] Todos los datos proporcionados al sistema deben respetar el formato establecido en el diccionario de datos.
	\BRItem[Sentencia] Sea $formato$ la expresión regular que determina el formato de un campo definido en el diccionario de datos, $L$ el lenguaje que genera $formato$ y $campo$ un campo introducido por el actor, entonces
	$ \forall campo \Rightarrow campo \in L $
	\BRItem[Motivación] Mantener los datos del sistema dentro del formato definido en el diccionario de datos.
	\BRItem[Ejemplo positivo] Cumplen la regla:
	\begin{itemize}
		\item El actor introduce el nombre de una Unidad Académica que solamente contiene caracteres alfabéticos.
		\item El actor introduce un número de teléfono que contiene sólo números.
		\item El actor introduce un correo electrónico que contiene un símbolo '@' y cuya terminación es un dominio de correo electrónico.
	\end{itemize}
	\BRItem[Ejemplo negativo] No cumplen con la regla:
	\begin{itemize}
		\item El actor introduce un nombre de Unidad Académica que contiene el símbolo '@'.
		\item El actor introduce un número de teléfono que contiene símbolos alfabéticos.
		\item El actor introduce un correo electrónico que no contiene el carácter '@'.
	\end{itemize}
\end{BusinessRule}

%======================================================================
\begin{BusinessRule}{BR-S003}{Eliminación lógica de elementos}
	{\bcCondition}    % Clase: \bcCondition,   \bcIntegridad, \bcAutorization, \bcDerivation.
	{\btEnabler}     % Tipo:  \btEnabler,     \btTimer,      \btExecutive.
	{\blControlling}    % Nivel: \blControlling, \blInfluencing.
	\BRItem[Versión] 1.0.
	\BRItem[Estado] Propuesta.
	\BRItem[Propuesta por] Ángeles Cerritos
	\BRItem[Revisada por] Pendiente.
	\BRItem[Aprobada por] Pendiente.
	\BRItem[Descripción] Un elemento sólo se puede eliminar si no tiene asociaciones con otros elementos. Tomando en cuenta la definición de la sentencia, se tienen los siguientes valores para $e_1$ y $e_2$, respectivamente:
	
	\begin{itemize}
		\item Espacio y Edificio.
		\item Plan de Estudio y Programa Académico.
		\item Espacio y Nivel.
	\end{itemize}
	
	\BRItem[Sentencia] Sean $ e_1 \in Entidad1 , e_2 \in Entidad2, R(x,y) = x $ está asociado con $ y $ entonces
	
	$ e_2 $ se puede eliminar si y sólo si $ \nexists e_1 $ tal que $ R(e_1, e_2) $ se cumpla.
	
	
	\BRItem[Motivación] Evitar que existan elementos en el sistema asociados a elementos que ya no existen dentro de él.
	\BRItem[Ejemplo positivo] Cumplen la regla:
	\begin{itemize}
		\item Eliminar un edificio que no tiene espacios asociados.
		\item Eliminar un Programa Académico que no tiene planes de estudio asociados.
		\item Eliminar los espacios asociados a un edificio y después eliminar el edificio.
	\end{itemize}
	\BRItem[Ejemplo negativo] No cumplen con la regla:
	\begin{itemize}
		\item Eliminar un edificio con un espacio asociado.
		\item Eliminar un edificio con tres espacios asociados.
		\item Eliminar un Programa Académico con un Plan de Estudio asociado.
	\end{itemize}
\end{BusinessRule}

%%======================================================================
\begin{BusinessRule}{BR-S004}{Unicidad de elementos}
	{\bcIntegridad}    % Clase: \bcCondition,   \bcIntegridad, \bcAutorization, \bcDerivation.
	{\btEnabler}     % Tipo:  \btEnabler,     \btTimer,      \btExecutive.
	{\blControlling}    % Nivel: \blControlling, \blInfluencing.
	\BRItem[Versión] 1.0.
	\BRItem[Estado] Propuesta.
	\BRItem[Propuesta por] Ángeles.
	\BRItem[Revisada por] Pendiente.
	\BRItem[Aprobada por] Pendiente.
	\BRItem[Descripción] Un elemento no se puede duplicar en el ámbito donde es utilizado ni registrarse en más de una ocasión. Dada la sentencia se consideran dentro de la regla las siguientes entidades y atributos:
	
	\begin{itemize}
		\item Unidad Académica, \{nombre, acrónimo\}
		\item Programa Académico, \{nombre\}
		\item Unidad de Aprendizaje, \{nombre, Plan de Estudio, Programa Académico, Unidad Académica\}
		\item Bibliografía, \{ISBN\}
		\item Plan de Estudio, \{nombre, Programa Académico, Unidad Académica\}
		\item Edificio,    \{Nombre por Unidad Académica\}
		\item Espacio,     \{Nombre por Edificio por Unidad Académica\}
		\item Consejero,   \{Número de boleta, Número de empleado\}
		\item Analista,      \{CURP\}
	\end{itemize}
	\BRItem[Sentencia] Sean $ e_1, e_2 \in Entidad , atributos = {atributo_1, atributo_2,...,atributo_n}$ atributos de la entidad tal que:\\
	Si $ \forall a \in atributos$ se cumple que $ e_1.a = e_2.a$ entonces $ e_1 = e_2 $.
	
	\BRItem[Motivación] Evitar la duplicidad de elementos dentro del sistema.
	\BRItem[Ejemplo positivo] Cumplen la regla:
	\begin{itemize}
		\item Dos Unidades de Aprendizaje con diferentes nombres.
		\item Dos prácticas con el mismo lugar de realización pero diferentes nombres.
		\item Dos Unidades Académicas con la misma dirección pero diferente nombre.
	\end{itemize}
	\BRItem[Ejemplo negativo] No cumplen con la regla:
	\begin{itemize}
		\item Dos Unidades de Aprendizaje con el mismo nombre.
		\item Dos Bibliografías con el mismo ISBN.
		\item Dos Planes de Estudio con el mismo nombre.
	\end{itemize}
\end{BusinessRule}

