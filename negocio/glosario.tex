El presente glosario presenta los términos utilizados a lo largo del documento. Esta unificación de criterios está basada en la normatividad y cultura del IPN y tiene como finalidad establecer el lenguaje base que permita comprender la especificación del sistema y se utilizará en la construcción del mismo.

\section{Glosario de términos}
\begin{bGlosario}
	%------------------------------------------------------------
	\bTerm{tAcademia}{Academia} Al órgano constituido por profesores que tiene la finalidad de proponer, analizar, opinar, estructurar y evaluar el proceso educativo.
	%------------------------------------------------------------
	%\bTerm{tActividadDentroDelPeriodoEscolar}{Actividades dentro del Periodo Escolar} Se refiere a Periodos\footnote{\refElem{tPeriodo}.} y fechas marcadas en el \refElem{tCalendarioEscolar} del IPN a realizarse dentro de los Periodos Escolares\footnote{\refElem{tPeriodoEscolar}}. Las cuales son:
  %  \begin{itemize}
	%	\item \refElem{tRegistroDeEvaluacionOrdinaria}.
	%	\item \refElem{tRegistroDeEvaluacionExtraordinaria}.
	%	\item \refElem{tSistemaDeInduccion}.
	%	\item \refElem{tSuspensionDeLabores}.
	%	\item \refElem{tRegistroDeEvaluacionPorSaberesPreviamenteAdquiridos}.
  %  \end{itemize}
	%------------------------------------------------------------
	%\bTerm{tActividadFueraDelPeriodoEscolar}{Actividades fuera del Periodo Escolar} Se refiere a Periodos\footnote{\refElem{tPeriodo}.} y fechas marcadas en el \refElem{tCalendarioEscolar} del IPN a realizarse fuera de los Periodos Escolares\footnote{\refElem{tPeriodoEscolar}}. Las cuales son:
	%\begin{itemize}
	%	\item \refElem{tAplicacionETS}.
	%	\item \refElem{tActividadIntersemestral}.
	%	\item \refElem{tVacaciones}.
	%\end{itemize}
	%------------------------------------------------------------
	%\bTerm{tActividadIntersemestral}{Actividad Intersemestral} Se refiere a las actividades realizadas por profesores\footnote{\refElem{tProfesor}}
	%una vez que el período escolar ha terminado. %% Propuesta.
	%------------------------------------------------------------
	\bTerm{tAlumno}{Alumno} A la persona inscrita en algún programa académico que se imparta en cualquier nivel educativo y	modalidad educativa
	que ofrece el Instituto Politécnico	Nacional. %% RGE
 	%------------------------------------------------------------
 	\bTerm{tAntecedenteAcademico}{Antecedente Académico} Conjunto de unidades de aprendizaje necesarias que sirven como fundamento o de apoyo
	para cursar otra de un semestre o nivel más alto %Derivado verificar redacción
 	%------------------------------------------------------------
 	%\bTerm{tAplicacionETS}{Aplicación de Examen a Título de Suficiencia} A la aplicación de los Exámenes a Título de Suficiencia\footnote{\refElem{tETS}}. %Derivada

 	%-----------------------------------------------------------
 	\bTerm{tAreaFormacion}{Área de Formación} Conjunto de conocimientos que por su afinidad conceptual, teórica y metodológica, conforman
	una porción claramente identificable de los contenidos de un plan de estudios en una carrera tecnológica, técnica superior o de grado.
	Pueden identificarse con áreas de conocimientos disciplinares, áreas temáticas, experiencias de formación, etc. En el Instituto se definen tres tipos de áreas de formación:
	\begin{itemize}
		\item Institucional.
		\item Científica, Humanística y Tecnológica Básica.
		\item Profesional.
	\end{itemize}

 	%-----------------------------------------------------------
 	%\bTerm{tCalendarioEscolar}{Calendario Escolar} Conjunto de calendarios de un año (también llamado \refElem{tCicloEscolar}) que inicia en agosto
	%de un año y termina en agosto del siguiente. En donde se marcan los Periodos Escolares\footnote{\refElem{tPeriodoEscolar}} de todas las
	%Modalidades\footnote{\refElem{tModalidad}} y todas las fechas relevantes (ya sea \refElem{tActividadDentroDelPeriodoEscolar} o
	%\refElem{tActividadFueraDelPeriodoEscolar}).
 	%----------------------------------------------------------
 	\bTerm{tCargaMaxima}{Carga máxima en Créditos} Resultado de dividir el número total de créditos del programa académico entre el número de
	periodos escolares de la duración mínima del plan de estudio.
 	%-----------------------------------------------------------
 	\bTerm{tCargaMedia}{Carga media en Créditos} Resultado de dividir el número total de créditos del programa académico entre el número
	de periodos escolares de la duración establecida en el plan de estudio.
 	%----------------------------------------------------------
 	\bTerm{tCargaMinima}{Carga mínima en Créditos} Resultado de dividir el número total de créditos del programa académico entre
	el número de periodos escolares de la duración máxima del plan de estudio.
 	%------------------------------------------------------------
 	\bTerm{tCarrera}{Carrera} ver \refElem{tProgramaAcademico}.
 	%-----------------------------------------------------------
	%\bTerm{tCicloEscolar}{Ciclo Escolar} El lapso anual que define el calendario académico
 	%----------------------------------------------------------
	\bTerm{tCompetencia}{Competencia}  Es la posibilidad de movilizar e integrar diversos saberes y recursos cognitivos cuando se
	enfrenta una situación-problema inédita, para lo cual la persona requiere mostrar la capacidad de resolver problemas complejos y abiertos,
	en distintos escenarios y momentos. permite identificar, seleccionar, coordinar y movilizar de manera articulada e interrelacionada un
	conjunto de saberes diversos en el marco de una situación educativa en un contexto específico.
	%-----------------------------------------------------------
	\bTerm{tCompetenciaGeneral}{Competencia General} Es la \refElem{tCompetencia} que se busca adquirir al finalizar una \refElem{tUnidadDeAprendizaje}.
	%-----------------------------------------------------------
 	\bTerm{tCredito}{Crédito} A la unidad de reconocimiento académico que mide y cuantifica las actividades de
	aprendizaje contempladas en un plan de estudio; es universal, transferible entre programas académicos y equivalente al
	trabajo académico del alumno.
 	%----------------------------------------------------------
 	\bTerm{tCreditosSATCA}{Créditos SATCA}  Es  un  conjunto  de  criterios  simples  y  unívocos  para
 	asignar  valor  numérico  a  todas  las actividades  de  aprendizaje  del  estudiante  contempladas  en  un  plan  de  estudios, con la finalidad
	de acumular y transferir créditos académicos.
 	%----------------------------------------------------------
 	\bTerm{tCreditosTEPIC}{Créditos TEPIC} Es la unidad de valor o puntuación de una asignatura, que se calcula de la siguiente forma:
 	\begin{enumerate}
 		\item  En actividades que requieren estudio o trabajo adicional del alumno, como en las clases teóricas y en los seminarios, una hora de
		clase-semana-semestre corresponde a dos créditos.
 		\item En actividades que no requieren estudio o trabajo adicional del alumno, como las prácticas, los laboratorios y los talleres,
		una hora-semana-semestre corresponde a un crédito.
 		\item  El valor en créditos de actividades clínicas y de las prácticas para el aprendizaje de la música, las artes plásticas y las
		asignaturas de preparación para el trabajo, se computarán globalmente según su importancia en el plan de estudios y a criterio de los cuerpos
		académicos correspondientes.
 	\end{enumerate}
	%----------------------------------------------------------
 	%\bTerm{tETS}{Examen a Título De Suficiencia} Es la evaluación que comprende el total de los contenidos del programa de estudios y que el alumno podrá presentar cuando no haya acreditado de manera ordinaria o extraordinaria alguna unidad de aprendizaje.
 	%----------------------------------------------------------
 	%\bTerm{tEvaluacionExtraordinaria}{Evaluación Extraordinaria}A la que comprende el total de los contenidos del programa de estudios y que el alumno podrá presentar voluntariamente, dentro del mismo periodo escolar, una vez que cursó la unidad de aprendizaje y no haya obtenido un resultado aprobatorio, o bien, si habiéndola acreditado, desea mejorar su calificación.
 	%------------------------------------------------------------
 	%\bTerm{tEvaluacionOrdinaria}{Evaluación Ordinaria}A la que se presenta con fines de acreditación durante el periodo escolar y
	%considera las evidencias de aprendizaje señaladas en el programa de estudios.
 	%-----------------------------------------------------------
	\bTerm{tEvidenciaIntegradora}{Evidencia Integradora} Elemento didáctico utilizado como evidencia de que el \refElem{tAlumno} adquirio el conocimiento
	o alcanzó una \refElem{tCompetencia}.%Verificar redaccción.
	%-----------------------------------------------------------
 	\bTerm{tFundamentacion}{Fundamentación} Redacción que le sirve de apoyo al \refElem{tProgramaDeEstudios} de una \refElem{tUnidadDeAprendizaje}
	para justificar su incorporación a un \refElem{tProgramaAcademico}.
 	%----------------------------------------------------------
	\bTerm{tIntencionEducativa}{Intención Educativa} Objetivos  específicos  definidos  en términos  de  actividades
	pedagógicas  en  el  nivel  de  la  secuencia de  aprendizaje;  permiten  la realización  concreta  de  los  saberes  bajo  la  forma
	de una  serie  de  adquisiciones  escolares (saber, saber ser, saber hacer, actitudes y valores).
	%----------------------------------------------------------
 	\bTerm{tISBN}{ISBN} Un ISBN es un código normalizado internacional para libros (International Standard Book Number). Los ISBN
	tuvieron 10 dígitos hasta diciembre de 2006 pero, desde enero de 2007, tienen siempre 13 dígitos. Los ISBN se calculan utilizando una
	fórmula matemática específica e incluyen un dígito de control que valida el código.
 	%------------------------------------------------------------
 	\bTerm{tISSN}{ISSN} Código de 8 dígitos usado para identificar periódicos, journals, revistas y publicaciones periódicas de todo tipo y
	en media-print y electrónico.
 	%------------------------------------------------------------
 	\bTerm{tMapaCurricular}{Mapa Curricular} A la representación gráfica de las unidades de aprendizaje que conforman un plan de estudio.
 	%------------------------------------------------------------
 	\bTerm{tModalidad}{Modalidad} Determina la forma en que debe impartirse una \refElem{tUnidadDeAprendizaje}, \refElem{tProgramaAcademico} o
	en que un \refElem{tAlumno} está llevando su \refElem{tCarrera}. Existen tres modalidades en el Instituto:
 	\begin{Citemize}
 		\item \refElem{tModalidadPresencial}.
 		\item \refElem{tModalidadMixta}.
 		\item \refElem{tModalidadADistancia}.
 	\end{Citemize}
 	%------------------------------------------------------------
	\bTerm{tModalidadPresencial}{Modalidad Presencial}  La que se desarrolla en aulas, talleres, laboratorios y otros ambientes de aprendizaje, en
	horarios y periodos determinados. A esta modalidad generalmente también se le conoce como Modalidad Escolarizada.
 	%------------------------------------------------------------
	\bTerm{tModalidadMixta}{Modalidad Mixta} Es la combinación de modalidades educativas de acuerdo con el diseño un programa académico en particular.
 	%------------------------------------------------------------
	\bTerm{tModalidadADistancia}{Modalidad a Distancia}	Es la que se desarrolla fuera de las aulas, talleres, laboratorios y no
	necesariamente comprende horarios determinados. A esta modalidad generalmente también se le conoce como Modalidad No Escolarizada.
 	%------------------------------------------------------------
	\bTerm{tOrientacionDidactica}{Orientación Didáctica} Instrumentos que sirven para clasificar y especificar la forma en que una
	\refElem{tUnidadDeAprendizaje} debe ser impartida.
	%------------------------------------------------------------
 	\bTerm{tPaginaElectronica}{Página Electrónica} Documento en internet que contiene la información de las unidades de
	aprendizaje\footnote{ver \refElem{tUnidadDeAprendizaje}}.
 	%------------------------------------------------------------
 	\bTerm{tPlanEstudio}{Plan de Estudio} Estructura curricular que se deriva de un programa académico y que
	permite cumplir con los propósitos de formación general, la adquisición de conocimientos y el desarrollo de capacidades
	correspondientes a un nivel y modalidad educativa.%Reglamento General de Estudios
 	%------------------------------------------------------------
 	\bTerm{tPerfilDocente}{Perfil Docente} Requisitos que sirve den apoyo al \refElem{tProgramaDeEstudios} de una \refElem{tUnidadDeAprendizaje}
	para definir el perfil con el que un \refElem{tProfesor} debe(o sugiere) tener para impartir una \refElem{tUnidadDeAprendizaje}.
% 	%------------------------------------------------------------
% 	\bTerm{tPeriodo}{Periodo} Se refiere a un lapso de tiempo el cual está marcado por una fecha de inicio y una fecha final.
% 	%------------------------------------------------------------
% 	\bTerm{tPeriodoEscolar}{Periodo Escolar} Se refiere al \refElem{tPeriodo} que rige la ejecución de todas las actividades relevantes en la gestión escolar. En la modalidad presencial tiene una duración aproximada de seis meses y en modalidad mixta seis semanas.
 	%------------------------------------------------------------
 	\bTerm{tPeriodoDeTrabajo}{Periodo de Trabajo} Lapso de tiempo en que una unidad de aprendizaje debe impartirse puede ser:
 		\begin{itemize}
 			\item Mensual.
 			\item Bimestral.
 			\item Trimestral.
 			\item Cuatrimestral.
 			\item Semestral.
 			\item Anual.
 		\end{itemize}
 	
 	%------------------------------------------------------------
 	\bTerm{tPractica}{Práctica} Elemento didáctico para ayudar y dar soporte al conocimiento adquirido de una \refElem{tUnidadDeAprendizaje}.
 	%-----------------------------------------------------------
 	\bTerm{tProfesor}{Profesor} Persona que conduce el proceso de enseñanza-aprendizaje enfocado a la transmisión de conocimientos y a
	la formación integral del alumno. Será la autoridad académica del grupo a su cargo y desempeñará sus actividades conforme al
	principio de libertad de cátedra e investigación, atendiendo los programas aprobados por las autoridades académico-administrativas del
	IPN y del centro de trabajo correspondiente.
 	%------------------------------------------------------------
 	\bTerm{tPolivirtual}{Polivirtual} Es el sistema del Instituto Politécnico Nacional mediante el cual ofrece estudios
	de bachillerato, licenciatura, posgrado y servicios educativos complementarios en modalidades alternativas, innovadoras y
	flexibles con apoyo de las tecnologías de la información y las comunicaciones.
 	%------------------------------------------------------------
	\bTerm{tProgramaAcademico}{Programa Académico} Conjunto organizado de elementos necesarios para generar, adquirir y aplicar
	el conocimiento en un campo específico; así como para desarrollar habilidades, actitudes y valores en el alumno, en diferentes
	áreas del conocimiento. %Reglamento General de Estudios
 	%------------------------------------------------------------
 	\bTerm{tProgramaDeEstudios}{Programa de Estudios} Contenidos formativos de una unidad de aprendizaje contemplada en un plan de estudio;
	especifica los objetivos a lograr por los alumnos en un periodo escolar; establece la carga horaria,
	número de créditos, tipos de espacios, ambientes y actividades de aprendizaje, prácticas escolares, bibliografía, plan de evaluación y
	programa sintético.
 	%-----------------------------------------------------------
 	\bTerm{tProgramaEnRed}{Programa Académico en Red} Programa que desarrollan e imparten conjuntamente varias unidades académicas del Instituto
 	o con otras instituciones con las que se tenga convenio.
 	%------------------------------------------------------------
 	\bTerm{ramaConocimiento}{Rama del Conocimiento} Es la clasificación designada para un \refElem{tProgramaAcademico} de nivel
	superior o medio superior. Dentro del Instituto se encuentran :
	\begin{itemize}
	\item Ciencias Físico Matemáticas.
	\item Ciencias Médico Biológicas.
	\item Ciencias Sociales y Administrativas.
	\end{itemize}
 	%----------------------------------------------------------
 	\bTerm{tRAP}{Resultado de Aprendizaje Propuesto} También conocidas como RAP's, se abordan a través de actividades sustantivas que tienen como
	propósito indicar una generalidad para desarrollar las secuencias didácticas que atenderán cada RAP.
	Las evidencias con las que se evaluará formativamente cada RAP, se definen mediante un desempeño integrado, en el que
	los estudiantes mostrarán su saber hacer de manera reflexiva, utilizando el conocimiento que va adquiriendo durante el proceso
	didáctico para transferir el aprendizaje a situaciones similares y diferentes.
 	%------------------------------------------------------------
 	\bTerm{tReferenciaDocumental}{Referencia Documental} Bibliografía sugerida para ocuparse durante el curso de una \refElem{tUnidadDeAprendizaje} de nivel
	medio superior.

 	%---------------------------------------------------------------
 	%\bTerm{tRegistroDeEvaluacionOrdinaria}{Registro de Evaluación Ordinaria} Periodos\footnote{\refElem{tActividadDentroDelPeriodoEscolar}} marcados como \refElem{tPeriodo}. en los que se permite o requiere a los profesores subir calificaciones\footnote{\refElem{tEvaluacionOrdinaria}.} de sus alumnos. Tradicionalmente existen tres periodos en modalidad \refElem{tModalidadPresencial} y uno en modalidad \refElem{tPolivirtual}.
 	%------------------------------------------------------------
 	%\bTerm{tRegistroDeEvaluacionExtraordinaria}{Registro de Evaluación Extraordinaria} \refElem{tActividadDentroDelPeriodoEscolar} marcada como \refElem{tPeriodo} en el que se requiere a los profesores registrar las evaluaciones extraordinarias\footnote{\refElem{tEvaluacionExtraordinaria}} de sus alumnos.
 	%------------------------------------------------------------
 	%\bTerm{tRegistroDeEvaluacionPorSaberesPreviamenteAdquiridos}{Registro de Evaluación por Saberes Previamente Adquiridos} \refElem{tActividadDentroDelPeriodoEscolar} marcada como fecha la cual determina el día máximo en el que se puede registrar calificaciones derivadas de \refElem{tSaberesPreviamenteAdquiridos}.
	%------------------------------------------------------------
 	%\bTerm{tSaberesPreviamenteAdquiridos}{Saberes Previamente Adquiridos}A la que permite acreditar unidades de aprendizaje sin haberlas cursado. Su aplicación se sujetará a lo descrito en el plan y programa de estudios, y a los lineamientos aplicables.
 	%------------------------------------------------------------
 	%\bTerm{tSistemaDeInduccion}{Semana de Inducción} \refElem{tActividadDentroDelPeriodoEscolar} marcada como \refElem{tPeriodo} en el que las Unidades Académicas\footnote{\refElem{tUnidadAcademica}} deben dar información de orientación e inducción a los aspirantes asignados.
 	%------------------------------------------------------------
 	%\bTerm{tSuspensionDeLabores}{Suspensión de labores} \refElem{tActividadDentroDelPeriodoEscolar} marcadas como fechas específicas en las que se suspenden las labores académicas. En principio, su definición está basada en los días de descanso marcados por la secretaría del trabajo y por el calendario escolar de la Secretaría de Educación Pública.
 	%------------------------------------------------------------
	\bTerm{tTipoEspacio}{Tipo de Espacio} Para nivel medio superior, indica en que lugar la \refElem{tUnidadDeAprendizaje} se imparte. Puede ser:
		\begin{itemize}
			\item Aula.
			\item Taller.
			\item Laboratorio.
			\item Otros ambientes de aprendizaje.
		\end{itemize}
	%------------------------------------------------------------
	\bTerm{tUnidadAcademica}{Unidad Académica} Espacio geográfico donde se ubican las escuelas, centros y unidades en los que se realizan actividades de docencia, investigación y difusión de la cultura, en los niveles superior y de posgrado.
 	%------------------------------------------------------------
 	\bTerm{tUnidadDeAprendizaje}{Unidad de Aprendizaje} Estructura didáctica que integra los contenidos formativos de un curso, materia, módulo, asignatura o sus equivalentes. En general, las unidades de aprendizaje deberán cursarse y acreditarse conforme lo establezca el plan de estudio, y podrán seleccionarse de entre la oferta disponible en el periodo escolar y sujeta a grupo. %Reglamento General de Estudios
 	%------------------------------------------------------------
 	\bTerm{tUnidadDidactica}{Unidad Didáctica} Elementos que definen y componen a una \refElem{tUnidadDeAprendizaje} de nivel medio superior.
 	%------------------------------------------------------------
 	\bTerm{tUnidadTematica}{Unidad Temática} Elementos que definen y componen a una \refElem{tUnidadDeAprendizaje} de nivel superior.
 	%------------------------------------------------------------
 	%\bTerm{tVacaciones}{Vacaciones} Descanso temporal de una actividad habitual, principalmente del trabajo remunerado o de los estudios. %RAE


 	%------------------------------------------------------------
%	\bTerm{tCarrera}{}
% 	%------------------------------------------------------------
%	\bTerm{tUnidadDeAprendizaje}{}
% 	%------------------------------------------------------------
%	\bTerm{tProgramaAcademico}{}
% 	%------------------------------------------------------------
%	\bTerm{tAlumno}{}


%	\bTerm{tAlta}{Alta} Documento que acredita y reconoce la inscripción. %MP DAE
%
%	\bTerm{tAlumnoRegistradoEnSIRCEI}{Alumno Registrado en SIRCEI} Alumno del nivel medio superior o superior, reconocido en el Sistema de Registro y Control Escolar Institucional, para su control y seguimiento a su trayectoria académica, y consulta de calificaciones, vía internet.x %MP DAE
%
%	\bTerm{tAntecedentesDeIngreso}{Antecedentes de Ingreso} Reseña de la Admisión del alumno en la Institución. % MP DAE
%
%	\bTerm{tAntecedentesDeTrayectoria}{Antecedentes de Trayectoria} Situación del alumno que ha llevado una dirección dentro del Instituto a lo largo del tiempo. % MP DAE
%
%	\bTerm{AspiranteAceptado}{Aspirante Aceptado} Aspirante que al haber aprobado el examen de admisión y que cumple con los requisitos establecidos en la convocatoria correspondiente es asignado a una unidad académica. %Derivado
%
%	\bTerm{tAspiranteAExaminar}{Aspirante a Examinar}Persona que se registra en el proceso de admisión para ingresar a alguna de las escuelas, centros y unidades del Instituto, en los niveles que constituyen su oferta educativa. %MP DAE
%
%	\bTerm{tBoleta}{Boleta}Cédula de identidad que se otorga a los aspirantes, para que se inscriban como alumnos del Instituto Politécnico Nacional. %MP DAE
%
%	\bTerm{Cadena}{Cadena} Es el Tipo de Dato definido por cualquier valor que se componga de una secuencia de caracteres, con o sin acentos, espacios, dígitos y signos de puntuación. Existen tres tipos de Cadenas: Palabra, Frase y Párrafo.
%
%	\bTerm{tCalendarioAcademico}{Calendario Académico} Programación que define los tiempos en los cuales se realizan anualmente las actividades académicas y de gestión escolar, en las diversas modalidades educativas que se imparten en el Instituto
%
%	\bTerm{tCicloEscolar}{Ciclo Escolar} El lapso anual que define el calendario académico
%
%	\bTerm{tConvocatoria}{Convocatoria} Documento de carácter oficial emitido por una o varias Instituciones Educativas, en el cual se plantea tanto el procedimiento, los tiempos y fechas, los
%	lineamientos legales en los que se basa la misma, y los requisitos que hay que cumplir para que un aspirante sea admitido al nivel de estudios superior inmediato concluido por él mismo, en la o las Instituciones que ofrecen sus distintas carreras y niveles de estudios impartidos por las mismas. % MP DAE
%
%	\bTerm{tCredencialDeAlumno}{Credencial de Alumno} Identificación oficial expedida por la Dirección de Administración Escolar del IPN, para todos aquellos alumnos que cursan alguna carrera en uno de los planteles educativos del Instituto, ya sea a Nivel Medio Superior o Superior. %MP DAE
%
%	\bTerm{Entero}{Entero} Es el Tipo de Dato definido por todos los valores numéricos enteros, tanto positivos como negativos.
%
%	\bTerm{tEstructuraAcademica}{Estructura Académica} Al conjunto de grupos, horarios y unidades de aprendizaje organizadas para el período siguiente.
%
%	\bTerm{tETS}{Examen a Título De Suficiencia} Es la evaluación que comprende el total de los contenidos del programa de estudios y que el alumno podrá presentar cuando no haya acreditado de manera ordinaria o extraordinaria alguna unidad de aprendizaje.
%
%	\bTerm{tExpedienteAcademico}{Expediente Académico} Al documento que contiene la información y el historial académico del alumno.
%
%	\bTerm{tExpedienteDelAspirante}{Expediente del Aspirante} Se refiere a la documentación que el aspirante entrega durante el proceso de admisión para su verificación. % Derivado
%
%	\bTerm{Fecha}{Fecha} Es el Tipo de Dato definido por todas las fecha pasadas y futuras. Se representa de dos formas: Fecha Corta y Fecha Larga.
%
%	\bTerm{FechaCorta}{Fecha corta} Representación de una fecha de la forma DD/MM/YYYY, ejemplo: 24/02/2012.
%
%	\bTerm{FechaLarga}{Fecha Larga} Representación de una fecha de la forma DD de MMMM, del YYYY, ejemplo: 24 de Febrero, del 2012.
%
%	\bTerm{Frase}{Frase} Cadena formada por mas de una palabra y que puede ocupar hasta un par de renglones.
%
%	\bTerm{tHistorialAcademico}{Historial Académico} Se entiende por historial académico al conjunto de calificaciones obtenidas por un alumno en las unidades de aprendizaje,durante su vida académica dentro del Instituto.
%
%	\bTerm{tHojaDeResultadoDelExamenDeAdmision}{Hoja de Resultado del Examen de Admisión} Documento que el interesado imprime a través de internet, para conocer la opción educativa donde quedó aceptado. Con la Hoja de Resultados los aspirantes seleccionados obtendrán una cita donde podrán continuar con sus trámites de inscripción al Instituto. %MP DAE
%
%	\bTerm{tHorario}{Horario} Documento que se le otorga a un aspirante o alumno en el que se confirma su inscripción al período en curso.
%
%	\bTerm{tIncidencia}{Incidencia}Acontecimiento que sobreviene en el curso de un asunto o negocio y tiene con él alguna conexión.
%
%
%	\bTerm{tKardex}{Kardex} Documento donde se resgistran los datos personales del alumno, el cual contiene la trayectoria escolar mediante las calificaciones obtenidas en las asignaturas y/o unidades de aprendizaje cursadas, de acuerdo a un plan de estudios, formas de evaluación y periodos escolares. % CIRCULAR NO5 Criterios de Expedición de Boletas de calificaciones
%
%	\bTerm{tModalidad}{Modalidad Educativa} Forma en que se organizan, distribuyen y desarrollan los planes y programas de estudio para su impartición. Existen 3 tipos de modalidades:
%	\begin{itemize}
%		\item [Escolarizada:] La que se desarrolla en aulas, talleres, laboratorios y otros ambientes de aprendizaje, en horarios y periodos determinados.
%		\item [No Escolarizada:] Es la que se desarrolla fuera de aulas,
%		talleres, laboratorios y no necesariemente comprende horarios determinados.
%		\item [Mixta:] Es la combinación de modalidades educativas de
%		acuerdo con el diseño un programa académico en particular.
%	\end{itemize}
%
%	\bTerm{tNombramiento}{Nombramiento} Proceso por el cual un aspirante a dar cátedra en el instituto es elegido otorgándole las horas que deberá cumplir. % RCITPAIPN
%
%
%	\bTerm{tOficioDeAlumnoAceptado}{Oficio de Alumno Aceptado} Documento generado por la Dirección de Administración Escolar del IPN, por medio del cual el aspirante es notificado que al concluir los trámites de inscripción adquiere el estatus de alumno del Instituto. % MP DAE
%
%
%	\bTerm{tPreboleta}{Preboleta} Es un numéro de identificación que se le asgina al aspirante como fin de control interno durante el proceso de admisión. %Derivado
%
%	\bTerm{tProcesoDeAdmision}{Proceso de Admisión} Conjunto de etapas que deben ser realizadas tanto por la Institución educativa como por el aspirante. Comprende desde la publicación de la convocatoria, el regist ro de aspirantes, la aplicación del examen de admisión, publicación de resultados e inscripciones de aspirantes seleccionados. %MP DAE
%
%
%
%	\bTerm{RCITPAIPN}{Reglamento de  las  Condiciones Interiores de Trabajo del Personal Académico del IPN} Reglamento que fija las condiciones de trabajo del personal académico del
%	Instituto Politécnico Nacional, que conjuntamente con sus tres anexos:
%	\begin{itemize}
%		\item[I.] Prestaciones Sociales y Económicas.
%		\item[II.] Seguridad e Higiene.
%		\item[III.] Promoción Docente.
%	\end{itemize}
%	Son de observancia obligatoria para el personal académico, el titular y demás funcionarios del Instituto
%	Politécnico Nacional y de sus Organos de Apoyo.
%
%	\bTerm{tRUAA}{Registro Único de Actividades Académicas } Es  el  documento  que  avala  las  funciones  y  actividades  que  desarrolla  el  docente
%
%	\bTerm{tTrayectoriaEscolar}{Trayectoria Escolar} Al proceso a través del cual el alumno construye su formación con base en un plan de estudio.
%
%
%
%	\bTerm{tValidacionDeInscripcion}{Validación de Inscripción} Proceso a través del cual se dictamina la autenticidad y legitimidad de los documentos aportados por el aspirante para su inscripción, si la documentación es correcta se le comunica por escrito. %MP DAE
%


	% TODO:Agregar fuente


\end{bGlosario}
