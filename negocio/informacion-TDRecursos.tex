\begin{TipoDeDato}{tdTipoDeRecurso}{Tipo de Recurso}{Es un catálogo que tiene como propósito almacenar información acerca  de la clasificación que reciben los distintos conjuntos de personas que laboran en el Instituto y ser un mecanismo que apoye a la identificación del personal para labores administrativas o académicas. Esta clasificación se le es otorgada al Recurso  por medio de un contrato o un convenio aplicable. }
		
		\begin{tdAtributos}
			\tdAttr{nombre}{Nombre}{tdfrase}{Es la palabra o el conjunto de palabras que identifican a una clasificación del personal que labora en el Instituto en actividades administrativas o académicas.}
			
			\tdAttr{descripcion}{Descripción}{tdtexto}{Es el conjunto de palabras que tienen como propósito exponer las características que definen a la agrupación a la que un Recurso Humano pertenece.}
		\end{tdAtributos}
		
		\subsection{Valores Iniciales}
		
		 En la siguiente tabla se muestran aquellos valores con los cuales el catálogo será poblado inicialmente y cuyas descripciones fueron obtenidas del \textbf{Reglamento de las condiciones interiores de trabajo del Personal Académico del Instituto Politécnico Nacional} y del \textbf{Reglamento de las condiciones generales de trabajo del Personal No Docente del Instituto Politécnico Nacional}. \cdtEmpty
		
		\begin{longtable}{| p{0.3\textwidth}| p{0.5\textwidth}|}
	 			\rowcolor{colorPrincipal}
	 			\multicolumn{2}{|c|}{\bf \color{white} Valores Iniciales}\\
	 			\hline
	 			\rowcolor{colorSecundario}
	 			\bf \color{white} Tipo de Recurso & \bf \color{white}  Descripción \\
	 			\hline
	 			\endhead
	 			\hline
	 			Personal Académico Docente & Responsable institucional de las funciones y labores académicas, que debe desarrollar de manera permanente. Realiza actividades de docencia, investigación científica, desarrollo tecnológico y además actividades complementarias.\\
	 			\hline
	 			Personal Académico Visitante & Responsable de funciones académicas específicas, por tiempo determinado, cuyas actividades, obligaciones y derechos quedan sujetos a los contratos celebrados con las autoridades del Instituto, o bien, al convenio correspondiente. \\
	 			\hline
	 			Personal Académico Visitante por Honorarios & Responsable de funciones académicas específicas, por tiempo determinado, cuyas actividades, obligaciones y derechos quedan sujetos a los contratos por honorarios celebrados con las autoridades del Instituto, o bien, al convenio correspondiente.\\
	 			\hline
	 			Personal No Docente & Persona que presta sus servicios al Instituto Politécnico Nacional, desempeñando trabajos administrativos, técnicos y manuales.\\
	 			\hline
	 			Personal No Docente por Honorarios & Persona que presta sus servicios al Instituto Politécnico Nacional, desempeñando trabajos administrativos, técnicos y manuales y que es contratado por honorarios.\\
	 			\hline
	 		\end{longtable}
	\end{TipoDeDato}
	
	
	\begin{TipoDeDato}{tdTipoDeUnidadAdscripcion}{Tipo de Unidad de Adscripción}{Es un catálogo que tiene como propósito almacenar las distintas clasificaciones que se generan a partir del conjunto de funciones que se realizan dentro de un área, departamento o división del Instituto.}
	
	\begin{tdAtributos}
		\tdAttr{nombre}{Nombre}{tdfrase}{Es la palabra o el conjunto de palabras que identifican a un conjunto de funciones dentro de un área, departamento o división del Instituto.}
		
		\tdAttr{descripcion}{Descripción}{tdfrase}{Es el conjunto de palabras que tienen como propósito explicar las características que definen a la agrupación a la que una Unidad de Adscripción pertenece.}
	
	\end{tdAtributos}
	
		\subsection{Valores Iniciales}
		
		 En la siguiente tabla se muestran aquellos valores con los cuales el catálogo será poblado inicialmente y que fueron obtenidos de acuerdo al \textbf{Artículo 2do} del \textbf{Reglamento Órganico} del Instituto Politécnico Nacional.\cdtEmpty
		
		\begin{longtable}{| p{0.3\textwidth}| p{0.5\textwidth}|}
	 			\rowcolor{colorPrincipal}
	 			\multicolumn{2}{|c|}{\bf \color{white} Valores Iniciales}\\
	 			\hline
	 			\rowcolor{colorSecundario}
	 			\bf \color{white} Nombre & \bf \color{white} Descripción \\
	 			\hline
	 			Unidad Académica & Establecimiento académico en los que se realizan actividades de docencia, investigación y difusión de la cultura de nivel medio superior y superior.\\
	 			\hline
	 			Unidad Administrativa & Establecimiento encargado para el apoyo del Director General, para la regulación o evaluación o para la integración, seguimiento y control(tales como la Dirección de Educación Superior, Dirección de Educación Media Superior o la Dirección de Administración Escolar). \\
				\hline
				Centro de Investigación & Establecimiento académico en el que se realiza investigación científica y tecnológica y docencia de nivel posgrado.\\
				\hline
				Órgano Consultivo & Órgano Colegiado compuesto por varias partes de la comunidad politécnica que se dedica a la resolución de situaciones o conflictos en una o en varias Unidad Académicas. \\
				\hline
	 		\end{longtable}
	\end{TipoDeDato}
	
	
	\begin{TipoDeDato}{tdCategoria}{Categoría}{Es un catálogo donde se define el nivel de plaza con que cuenta un profesor. Esta categoría va aumentando de acuerdo a varios factores(antigüedad, grado de estudios, promociones) e indica el número de horas que debe cubrir el profesor frente a grupo como carga académica.}
	
		\begin{tdAtributos}
		
			\tdAttr{clave}{Clave}{tdpalabra}{Es un conjunto de dígitos alfanuméricos que tienen como propósito identificar a una categoría. Se puede validar a partir de la siguiente expresión regular :
			\begin{Citemize}
				\item $[A-Z]((2|3)[0-9]\{3\})$
			\end{Citemize}}
			
			\tdAttr{nombre}{Nombre}{tdfrase}{Es el conjunto de palabras que identifican a una categoría describiendo un tipo de profesor y la cantidad de horas que debe cubrir frente a grupo.}
			
			%\tdAttr{abreviatura}{Abreviatura}{tdfrase}{Es el símbolo o conjunto de símbolos que forman una clave con la cual se representa el nombre de una categoría.}
		\end{tdAtributos}
	
		\subsection{Valores Iniciales}
				
		Para conocer estos valores es requerido ver el \textbf{Tabulador del Personal Académico}.
	\end{TipoDeDato}
	
	\begin{TipoDeDato}{tdTipoDeNombramiento}{Tipo de Nombramiento}{Es un catálogo en el que se describe la formalización de manera oficial de la categoría y el número de horas que el profesor debe cubrir frente a grupo por cada plaza con la que se cuente.}
	
		\begin{tdAtributos}
			\tdAttr{nombre}{Nombre}{tdpalabra}{Es el conjunto de caracteres que identifica a una categoría y el número de horas que un profesor debe cubrir frente a grupo.}
			
			\tdAttr{descripcion}{Descripción}{tdfrase}{Es el conjunto de palabras que tienen como propósito explicar las características que definen a la agrupación a la que un Profesor pertenece.}
		\end{tdAtributos}
		
		\subsection{Valores Iniciales}	
	En la siguiente tabla se muestran aquellos valores con los cuales el catálogo será poblado inicialmente. \cdtEmpty
		\begin{longtable}{| p{0.3\textwidth}| p{0.5\textwidth}|}
	 			\rowcolor{colorPrincipal}
	 			\multicolumn{2}{|c|}{\bf \color{white} Valores Iniciales}\\
	 			\hline
	 			\rowcolor{colorSecundario}
	 			\bf \color{white} Nombre & \bf \color{white}  Descripción \\
	 			\hline
	 			\endhead
	 			Basificado & Indica que las horas de una plaza son propiedad del profesor. \\
	 			\hline
	 			Candidato & Indica que las horas de una plaza cubren una incidencia o una necesidad para la creación de una estructura educativa. Estas horas no son propiedad del profesor. \\
	 			\hline
	 		\end{longtable}
	\end{TipoDeDato}
	
	\begin{TipoDeDato}{tdRolFuncional}{Rol Funcional}{Es un catálogo en el que se almacenan la agrupaciones de funcionalidades del Sistema a la que un Recurso Humano tiene acceso de acuerdo a la labor que desempeña en el Instituto.}
		
		\begin{tdAtributos}
		
			\tdAttr{nombre}{Nombre}{tdfrase}{Es la palabra o el conjunto de palabras identifican a un rol asignado a un Recurso Humano en el Sistema.}
			
			\tdAttr{descripcion}{Descripción}{tdfrase}{Es un conjunto de palabras que describen a el rol asignado y sus actividades dentro del Sistema.}
		\end{tdAtributos}
		
		\subsection{Valores Iniciales}
		 En la siguiente tabla se muestran aquellos valores con los cuales el catálogo será poblado  inicialmente. \cdtEmpty
		\begin{longtable}{| p{0.3\textwidth}| p{0.6\textwidth}|}
			\rowcolor{colorPrincipal}
	 		\multicolumn{2}{|c|}{\bf \color{white} Valores Iniciales}\\
	 		\hline
	 		\rowcolor{colorSecundario}
	 		\bf \color{white} Nombre & \bf \color{white}  Descripción \\
	 		\hline
	 		\endhead
			Encargado de la COSIE de Medio Superior &	Es una persona que tiene como labor realizar la definición en el sistema requerida de los Consejos, sus Sesiones y los Consejeros que participarán en una Unidad Académica de nivel Medio Superior.\\
			\hline
			Encargado de la COSIE de Superior & Es una persona que tiene como labor realizar la definición en el sistema requerida de los Consejos, sus Sesiones y los Consejeros que participarán en una Unidad Académica de nivel Superior.\\	
			\hline
			Analista de la COSIE de Medio Superior	& Es un Recurso Humano que tiene como labor elaborar la predictaminación de Solicitudes de Dictamen en una Unidad Académica de nivel Medio Superior.\\
			\hline
			Analista de la COSIE de Superior & Es un Recurso Humano que tiene como labor elaborar la predictaminación de Solicitudes de Dictamen en una Unidad Académica de nivel Superior.\\
			\hline
			Responsable de la Estructura Educativa de Medio Superior & Es un Recurso Humano que tiene como labor realizar la planeación de la Estructura Educativa de una Unidad Académica de nivel Medio Superior.	\\
			\hline
			Responsable de la Estructura Educativa de Superior & Es un Recurso Humano que tiene como labor realizar la planeación de la Estructura Educativa de una Unidad Académica de nivel Superior.\\
			\hline	
			Auxiliar Estructura Eductiva de Medio Superior - Soporte Documental & Es un Recurso Humano que tiene como labor realizar la documentación necesaria para el Soporte Documental  de la Estructura Educativa de una Unidad Académica de nivel Medio Superior.\\
			\hline
			Auxiliar Estructura Eductiva de Superior - Soporte Documental & Es un Recurso Humano que tiene como labor realizar la documentación necesaria para el Soporte Documental  de la Estructura Educativa de una Unidad Académica de nivel Superior.\\
			\hline
			Auxiliar Estructura Educativa de Medio Superior - Horarios & Es un Recurso Humano que tiene como labor realizar la documentación necesaria para la generación de Horarios de la Estructura Educativa de una Unidad Académica de nivel Medio Superior.\\
			\hline
			Auxiliar Estructura Educativa de Superior - Horarios & Es un Recurso Humano que tiene como labor realizar la documentación necesaria para la generación de Horarios de la Estructura Educativa de una Unidad Académica de nivel Superior.\\
			\hline
			Auxiliar de Estructura Educativa de Medio Superior - Ambos & Es un Recurso Humano que tiene como labor realizar la documentación necesaria para el Soporte Documental y para la generación de Horarios de la Estructura Educativa de una Unidad Académica de nivel Medio Superior.\\
			\hline
			Auxiliar de Estructura Educativa de Superior - Ambos & Es un Recurso Humano que tiene como labor realizar la documentación necesaria para el Soporte Documental y para la generación de Horarios de la Estructura Educativa de una Unidad Académica de nivel Superior.\\
			\hline
	 		\end{longtable}
\end{TipoDeDato}