\begin{TipoDeDato}{tdTipoDeParentesco}{Tipo de Parentesco}{Es un catálogo en el que se encuentra la clasificación que recibe una la relación entre dos personas  que se encuentra definida por la consanguinidad, afinidad, adopción, matrimonio u otra clase. Tiene como utilidad brindar el mecanismo que permita identificar la afinidad de una persona con un alumno y así comunicar su situación escolar o sobre una emergencia.}
	
	\begin{tdAtributos}
	
		\tdAttr{nombre}{Nombre}{tdfrase}{Es la palabra o el conjunto de palabras que identifica la afinidad de una persona con un Alumno.}
		
		\tdAttr{abreviatura}{Abreviatura}{tdfrase}{Es el símbolo o conjunto de símbolos que representan la afinidad de una persona con un Alumno.}
		
		\tdAttr{descripcion}{Descripción}{tdfrase}{Es el conjunto de palabras que tienen como propósito explicar las características que definen la afinidad de una persona con un Alumno determinado así un grado de responsabilidad.}
	\end{tdAtributos}
	
	
	\subsection{Valores Iniciales}
	En la siguiente tabla se muestran aquellos valores con los cuales el catálogo será poblado inicialmente y que fueron obtenidos de acuerdo al catálogo del I.NE.G.I. el cual puede consultarse en la siguiente liga: \hyperlink{http://www3.inegi.org.mx/sistemas/clasificaciones/parentesco/parentesco.aspx}{Catálogo de Parentescos}.\cdtEmpty

		\begin{longtable}{| p{0.15\textwidth}| p{0.15\textwidth}|p{0.45\textwidth}|}
	 			\rowcolor{colorPrincipal}
	 			\multicolumn{3}{|c|}{\bf \color{white} Valores Iniciales}\\
	 			\hline
	 			\rowcolor{colorSecundario}
	 			\bf \color{white} Nombre & \bf \color{white} Abreviatura & \bf\color{white} Descripción\\
	 			\hline
	 			Jefa o jefe & J & Es el vínculo que se coloca en la cúspide de una relación jerárquica de mando-obediencia; en el ámbito familiar, refiere al parentesco reconocido como tal por los integrantes del hogar o los residentes de la vivienda. Este reconocimiento de jefa o jefe puede darse ya sea por razones de dependencia económica, por vínculo emocional, por edad, autoridad o respeto. Se clasifica en este grupo a las personas que viven solas.\\
	 			\hline
	 			Esposa o esposo & E & Este grupo comprende a los parentescos que se dan por el vínculo que existe por una relación conyugal, la cual se establece por matrimonio o unión consensual.\\
	 			\hline
	 			Hija o hijo & H & 	Este grupo comprende las descripciones de parentesco que se da por descendencia consanguínea directa de los padres. Considera el vínculo que se da por adopción.\\
	 			\hline
	 			Otro parentesco & OP & Este grupo contiene a los parentescos distintos a los miembros del núcleo familiar, cónyuge e hijos, considera parentescos por consanguinidad, afinidad o por costumbre, tanto ascendientes, descendientes, colaterales o transversales. \\
	 			\hline
	 			No tiene parentesco & NTP & En este grupo se clasifican las descripciones que hacen alusión a relaciones que no se consideran como parentesco, por ejemplo, amiga o amigo. Considera la tutela, ya que ésta de acuerdo con el Código Civil Federal, no se deriva de la afinidad y no crea parentesco.  \\
	 			\hline
	 			Trabajador doméstico & TD & Este grupo comprende las descripciones de aquella relación que se da con personas que prestan un servicio doméstico en la vivienda a cambio de una remuneración económica o en especie. Comprende también a los parientes de los trabajadores domésticos.\\
	 			\hline
	 			Huésped & H & Este grupo comprende las descripciones de las relaciones que se dan cuando una de ellas paga una remuneración por el alojamiento y en algunos casos por la alimentación y otros servicios.\\
	 			\hline
	 			Parentesco no especificado & PNE & Este grupo incluye aquellas descripciones ambiguas, vagas, ajenas al tema o por no saber si existe alguna relación.\\
	 			\hline
	 		\end{longtable}
	\end{TipoDeDato}
	
	
	\begin{TipoDeDato}{tdTipoDeDocumento}{Tipo de Documento}{Es un catálogo en el que se denota una característica particular, con la que se puede diferenciar un documento(soporte con que se prueba o acredita lo que el nombre del registro indique.) de otro.}
	\begin{tdAtributos}	
		\tdAttr{nombre}{Nombre}{tdfrase}{Es la palabra o el conjunto de palabras que identifica la clasificación a la que pertenece un documento de acuerdo a su contenido.}	
	\end{tdAtributos}
	
	\subsection{Valores Iniciales}
	En la siguiente tabla se muestran aquellos valores con los cuales el catálogo será poblado inicialmente.\cdtEmpty

		\begin{longtable}{| p{0.3\textwidth}|}
	 			\rowcolor{colorPrincipal}
	 			\bf \color{white} Valores Iniciales\\
	 			\hline
	 			\rowcolor{colorSecundario}
	 			\bf \color{white} Nombre\\
	 			\hline
	 			I.N.E. \\
	 			\hline
	 			C.U.R.P.\\
	 			\hline
	 			Acta de Nacimiento \\
	 			\hline
	 			Licencia para Conducir \\
	 			\hline
	 			Cartilla Militar\\
	 			\hline
	 			Pasaporte\\
	 			\hline
	 		\end{longtable}
\end{TipoDeDato}

\begin{TipoDeDato}{tdTipoDeSangre}{Tipo de Sangre}{Es un catálogo en el que se almacena la clasificación que reciben los distintos grupos sanguíneos existentes y que tienen como propósito brindar información que apoyará en emergencias de salud en las que la integridad de un alumno pueda verse afectada.}
	
	\begin{tdAtributos}
		\tdAttr{nombre}{Nombre}{tdfrase}{Es la palabra o el conjunto de palabras que identifican a un grupo sanguíneo.}
		
		\tdAttr{abreviatura}{Abreviatura}{tdfrase}{Es el símbolo o conjunto de símbolos que representan a un grupo sanguíneo.}
		
		\tdAttr{descripcion}{Descripción}{tdfrase}{Es el conjunto de palabras que tienen como propósito detallar la compatibilidad entre los distintos grupos sanguíneos.}
	
	\end{tdAtributos}
	
	\subsection{Valores Iniciales}
  En la siguiente tabla se muestran aquellos valores con los cuales el catálogo será poblado inicialmente. \cdtEmpty

		\begin{longtable}{| p{0.16\textwidth}| p{0.2\textwidth}|p{0.5\textwidth}|}
	 			\rowcolor{colorPrincipal}
	 			\multicolumn{3}{|c|}{\bf \color{white} Valores Iniciales}\\
	 			\hline
	 			\rowcolor{colorSecundario}
	 			\bf \color{white} Nombre & \bf \color{white} Abreviatura &  \bf \color{white} Descripción\\
	 			\hline
	 			 A positivo & A+ & A+ puede donar a personas de tipos A+ y AB+, y pueden recibir de cualquier tipo A u O.\\\hline
	 			 A negativo & A- & A- le puede donarle a otras personas de A- pero también a A+, AB+ y AB-, pero sólo puede recibir de A- y O-\\\hline
	 			 B positivo & B+ & B+ pueden donarle a  personas de tipo B+ y AB+, y pueden recibir de personas de cualquier tipo de sangre B u O.\\\hline
	 			 B negativo & B- & Este tipo de sangre menos común puede donarle a personas de sangre tipo B+, B-, AB+ y AB-, pero sólo puede recibir de B- y O-.\\\hline
	 			 O positivo & O+  & Las personas con O+ le pueden donar sangre a todos los tipos de sangre positivos, pero sólo pueden recibir de O+ u O-.\\\hline
	 			 O negativo & O- & Las personas con el tipo de sangre O- son consideradas donantes universales y pueden donar sangre a todos los tipos de sangre, pero sólo pueden recibir de sus donantes tipo O-. \\\hline
	 			 AB Positivo & AB+ & AB+ sólo le puede donar a otros receptores de AB+, pero como el  receptor universal, puede recibir de todos los otros tipos de sangre.\\\hline
	 			 AB Negativo & AB- & El tipo de sangre AB- le puede donar a AB- y a AB+, y puede recibir de todos los tipos de sangre negativos. \\
	 			\hline
	 		\end{longtable}
\end{TipoDeDato}


\begin{TipoDeDato}{tdTipoDeDiscapacidad}{Tipo de Discapacidad}{Es un catálogo en el que se se almacena la clasificación que recibe un conjunto de faltas o limitaciones de alguna facultad física o mental que imposibilita o dificulta el desarrollo normal de la actividad de una persona. Tiene como propósito indicar qué clase de discapacidad tiene un Alumno.}

	\begin{tdAtributos}
		
		\tdAttr{nombre}{Nombre}{tdfrase}{Es la palabra o el conjunto de palabras que identifica a una agrupación de faltas o limitaciones de alguna facultad física o mental que imposibilita o dificulta el desarrollo normal de una persona.}
		
		\tdAttr{abreviatura}{Abreviatura}{tdfrase}{Es la símbolo o conjunto te símbolos que forman una clave la cual representa propósito un nombre alternativo a la agrupación de faltas o limitaciones de alguna facultad física o mental que imposibilita o dificulta el desarrollo normal de una persona.}
		
		\tdAttr{descripcion}{Descripción}{tdfrase}{Es el conjunto de palabras que tienen como propósito detallar a un conjunto de faltas o limitaciones de alguna facultad física o mental que imposibilita o dificulta el desarrollo normal de una persona.}
	\end{tdAtributos}

	\subsection{Valores Iniciales}
	En la siguiente tabla se muestran aquellos valores con los cuales el catálogo será poblado inicialmente cuyos registros fueron obtenidos de acuerdo al catálogo del I.N.E.G.I. el cual puede consultarse en la siguiente liga: \hyperlink{http://www.inegi.org.mx/est/contenidos/proyectos/aspectosmetodologicos/clasificadoresycatalogos/doc/clasificacion_de_tipo_de_discapacidad.pdf}{Catálogo de Tipo de Discapacidades}\cdtEmpty
		\begin{longtable}{| p{0.3\textwidth}| p{0.15\textwidth}|p{0.5\textwidth}|}
	 			\rowcolor{colorPrincipal}
	 			\multicolumn{3}{|c|}{\bf \color{white} Valores Iniciales}\\
	 			\hline
	 			\rowcolor{colorSecundario}
	 			\bf \color{white} Nombre & \bf \color{white} Abreviatura &  \bf \color{white} Descripción\\
	 			\hline
				Discapacidades para ver & Subgrupo 110 & Incluye las descripciones que se refieren a la pérdida total de la visión, a la debilidad visual (personas que sólo ven sombras o bultos), y a otras limitaciones que no pueden ser superadas con el uso de lentes, como desprendimiento de retina, acorea, facoma y otras. Se considera que hay discapacidad cuando está afectado un sólo ojo o los dos.\\
				\hline
				Discapacidades para oír & Subgrupo 120& Comprende las descripciones que se relacionan con la pérdida total de la audición en uno o en ambos oídos, o con la pérdida parcial pero intensa, grave o severa en uno o en ambos oídos. \\
				\hline
				Discapacidades para hablar & Subgrupo 130 &  Se refiere exclusivamente a la pérdida total del habla.\\
				\hline
				Discapacidades de la comunicación y comprensión del lenguaje & Subgrupo 131 & Incluye las discapacidades que se refieren a la incapacidad para generar, emitir y comprender mensajes del habla. Comprende las limitaciones importantes, graves o severas del lenguaje, que impiden la producción de mensajes claros y comprensibles.\\
				\hline
				Discapacidades de las extremidades inferiores, tronco, cuello y cabeza & Subgrupo 210 & Comprende a las personas que tienen limitaciones para moverse o caminar debido a la falta total o parcial de sus piernas. Comprende también a aquellas que aún teniendo sus piernas no tienen movimiento en éstas, o sus movimientos tienen restricciones que provocan que no puedan desplazarse por sí mismas, de tal forma que necesitan la ayuda de otra persona o de algún instrumento como silla de ruedas, andadera o una pierna artificial (prótesis). \\
				\hline
				Discapacidades de las extremidades superiores & Subgrupo 220 & Comprende a las personas que tienen limitaciones para utilizar sus brazos y manos por la pérdida total o parcial de ellos, y aquellas personas que aun teniendo sus miembros superiores (brazos y manos) han perdido el movimiento, por lo que no pueden realizar actividades propias de la vida cotidiana tales como agarrar objetos, abrir y cerrar puertas y ventanas, empujar, tirar o jalar con sus brazos y manos etcétera.\\
				\hline
				Discapacidades intelectuales & Subgrupo 310 & Este subgrupo comprende las discapacidades intelectuales que se manifiestan como retraso o deficiencia mental y pérdida de la memoria. Comprende a las personas que presentan una capacidad intelectual inferior al promedio de las que tienen su edad, su grado de estudios y su nivel sociocultural. A ellas se les dificulta realizar una o varias de las actividades de la vida cotidiana, como asearse, realizar labores del hogar, aprender y rendir en la escuela o desplazarse en sitios públicos.\\
				\hline
				Discapacidades conductuales y otras mentales & Subgrupo 320 & En este subgrupo están comprendidas las discapacidades de moderadas a severas que se manifiestan en el comportamiento o manera de conducirse de las personas, tanto en las actividades de la vida diaria como en su relación con otros. En este tipo de discapacidades, la persona puede tener una interpretación y respuesta inadecuada a acontecimientos externos.\\
				\hline
				Discapacidades múltiples & Subgrupo 401-422 D& Se incluye en este subgrupo a las personas que tienen limitaciones o carencia de movimiento en las extremidades inferiores y superiores, como por ejemplo, parálisis cerebral, embolia o accidente cerebrovascular. Incluye descripciones relativas a dos o más discapacidades.\\
				\hline
	 		\end{longtable}
\end{TipoDeDato}


\begin{TipoDeDato}{tdDeporte}{Deporte}{Es un catálogo en el que se almacena la clasificación a la que pertenecen un conjunto de actividades físicas que son reglamentadas y de carácter competitivo las cuales son llevadas a cabo por un Alumno dentro o fuera del Instituto.}
	
	\begin{tdAtributos}
		\tdAttr{nombre}{Nombre}{tdfrase}{Es la palabra o el conjunto de palabras que identifican a un conjunto de actividades físicas reglamentadas y de carácter competitivo.}
		
		\tdAttr{abreviatura}{Abreviatura}{tdpalabra}{Es el símbolo o conjunto de símbolos que forman una clave la cual representa a una agrupación de actividades físicas, reglamentadas y de carácter competitivo.}
		
		\tdAttr{descripcion}{Descripción}{tdfrase}{Es el conjunto de palabras que define a una agrupación de actividades y detalla las características que definen que una actividad pertenezca a una agrupación.}

	\end{tdAtributos}
	
	\subsection{Valores Iniciales}
	
 En la siguiente tabla se muestran aquellos valores con los cuales el catálogo será poblado inicialmente. \cdtEmpty

		\begin{longtable}{| p{0.15\textwidth}| p{0.15\textwidth}|p{0.15\textwidth}|p{0.15\textwidth}| p{0.15\textwidth}|}
	 			\rowcolor{colorPrincipal}
	 			\multicolumn{5}{|c|}{\bf \color{white} Valores Iniciales}\\
	 			\hline
	 			\rowcolor{colorSecundario}
	 			\bf \color{white} Nombre & \bf\color{white} Nombre & \bf\color{white} Nombre & \bf\color{white}Nombre & \bf\color{white} Nombre \\
				\hline
				Ajedrez &Atletismo &Bádminton & Basquetbol & Beisbol\\
				\hline
				Boxeo& Ciclismo & Clavados & Equitación & Frontenis\\
				\hline
				Fronton & Futbol & Futbol 7 & Futbol Americano & Futbol de Salón\\
				\hline
			        Futbol Rápido& Gimnasia & Golf  & Halterofilia & Hockey\\
				\hline
				Judo & Karate & Kickboxing & Motociclismo & Natación\\
				\hline
				Ping Pong & Paracaidismo & Polo & Remo &  Rugby \\
				\hline
				Skateboard & Softbol  & Squash  & Sumo& Taekwondo \\
				\hline
				Tenis &Tiro con Arco & Voleibol  & Waterpolo & \[vacio\]\\
				\hline
	 		\end{longtable}
\end{TipoDeDato}
	