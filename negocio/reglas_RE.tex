\subsection{Reglas de Negocio de Evaluaciones}

%======================================================================
\begin{BusinessRule}{BR-RE-N001}{Validación de acta de calificaciones completa}%
%	
	{\bcCondition}%    % Clase: \bcCondition,   \bcIntegridad, \bcAutorization, \bcDerivation.
	{\btExecutive}%     % Tipo:  \btEnabler,     \btTimer,      \btExecutive.
	{\blControlling}%    % Nivel: \blControlling, \blInfluencing.
	\BRItem[Versión] 1.0.
	\BRItem[Estado] Propuesta.
	\BRItem[Propuesta por] Elsi
	\BRItem[Revisada por] Pendiente.
	\BRItem[Aprobada por] Pendiente.
	\BRItem[Descripción] 
	
		Para poder cerrar un acta de calificaciones, es necesario que todos los alumnos registrados, tengan asignada una calificación. 	
			
	\BRItem[Motivación] 1

\end{BusinessRule}
%======================================================================
\begin{BusinessRule}{BR-RE-N002}{Generación de firma del sistema}%
%	
	{\bcAutorization}% Clase: \bcCondition,   \bcIntegridad, \bcAutorization, \bcDerivation.
	{\btExecutive}% Tipo:  \btEnabler,     \btTimer,      \btExecutive.
	{\blControlling}% Nivel: \blControlling, \blInfluencing.
	\BRItem[Versión] 1.0.
	\BRItem[Estado] Propuesta.
	\BRItem[Propuesta por] Elsi
	\BRItem[Revisada por] Pendiente.
	\BRItem[Aprobada por] Pendiente.
	\BRItem[Descripción] 
	
	Cuando las calificaciones de los alumnos son guardadas o el acta de calificaciones es cerrada, el sistema generará una firma que se concatenará al archivo XML del acta de calificaciones correspondiente. Dicha firma está conformada por X caracteres alfanuméricos y es generada mediante el algoritmo Generación de firma de sistma... 	
	
	\BRItem[Motivación] 2
	
\end{BusinessRule}
%======================================================================
\begin{BusinessRule}{BR-RE-N003}{Generación de folio de acta de califiaciones}%
%	
	{\bcAutorization}% Clase: \bcCondition,   \bcIntegridad, \bcAutorization, \bcDerivation.
	{\btExecutive}% Tipo:  \btEnabler,     \btTimer,      \btExecutive.
	{\blControlling}% Nivel: \blControlling, \blInfluencing.
	\BRItem[Versión] 1.0.
	\BRItem[Estado] Propuesta.
	\BRItem[Propuesta por] Elsi
	\BRItem[Revisada por] Pendiente.
	\BRItem[Aprobada por] Pendiente.
	\BRItem[Descripción] 
	
	Cada acta de calificaciones debe tener un folio de identificación único, el cual es generado cuando el acta de calificaciones ha sido cerrada y firmada por el sistema. El folio digital es una cadena de 30 caracteres que es generada por el algoritmo X.  	
	
	\BRItem[Motivación] 3
	
\end{BusinessRule}
%======================================================================
\begin{BusinessRule}{BR-RE-N004}{Registro de calificaciones}%
%	
	{\bcDerivation}% Clase: \bcCondition,   \bcIntegridad, \bcAutorization, \bcDerivation.
	{\btTimer}% Tipo:  \btEnabler,     \btTimer,      \btExecutive.
	{\blControlling}% Nivel: \blControlling, \blInfluencing.
	\BRItem[Versión] 1.0.
	\BRItem[Estado] Propuesta.
	\BRItem[Propuesta por] Elsi
	\BRItem[Revisada por] Pendiente.
	\BRItem[Aprobada por] Pendiente.
	\BRItem[Descripción] 
	  	
	
	\BRItem[Motivación] 4
	
\end{BusinessRule}
%======================================================================
\begin{BusinessRule}{BR-RE-N005}{Consulta de calificaciones}%
%	
	{\bcCondition}% Clase: \bcCondition,   \bcIntegridad, \bcAutorization, \bcDerivation.
	{\btExecutive}% Tipo:  \btEnabler,     \btTimer,      \btExecutive.
	{\blControlling}% Nivel: \blControlling, \blInfluencing.
	\BRItem[Versión] 1.0.
	\BRItem[Estado] Propuesta.
	\BRItem[Propuesta por] Oscar.
	\BRItem[Revisada por] Pendiente.
	\BRItem[Aprobada por] Pendiente.
	\BRItem[Descripción] 
	
	Cuando se solicitan las calificaciones previamente registradas de un acta de califiaciones para su consulta, el sistema las obtendrá del archivo xml y de las tablas de la base de datos que las contienen, posteriormente verifica que no exista inconsitencia en el archivo xml mediante la verificacion del hash, y hace una comparación con las calificaciones registradas en la base de datos. Si las calificaciones coinciden, las muestra al usuario. 
	
	\BRItem[Motivación] 5
	
\end{BusinessRule}

